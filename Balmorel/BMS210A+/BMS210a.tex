% huskeseddel:
% parameter ihelp?r
% sets is, it ej dokumenterede
%  IGHORHERUR  ?

% i QEEQ: skal IRRE og IRRI ombyttes?????



%husk:
% search-replace p� modeling / modelling



\documentclass[twoside,10pt]{article}
 \makeindex

\newcommand{\gmssec}{the BALMOREL.GMS file section [...]}
\newcommand{\MONEY}{Money}
\newcommand{\checks}{The input is checket for reasonable values. See file errors.inc.}
\newcommand{\inputdata}{Comment on input data: }
\newcommand{\naming}{Comment on naming conventions: }
\newcommand{\Versionname}{2.10 Alpha (October 2002) }
\newcommand{\VersionGAMS}{2.25}

\usepackage{graphicx}


 \setlength{\textheight}{240mm}
 \setlength{\textwidth}{125mm}
 \setlength{\oddsidemargin}{15mm}
 \setlength{\evensidemargin}{15mm}
\setlength{\topmargin}{-15mm}


\begin{document}


%\section*{The Balmorel Model Structure}


%\newpage


\section*{{\large Hans F. Ravn}}

\vspace{5cm}
%\section*{}\section*{}\section*{}

\section*{{\LARGE The Balmorel Model Structure}}

\section*{}\section*{}\section*{}



\vspace{12cm}

\subsubsection*{Version \Versionname}
\subsubsection*{(this document revised 2003.01.31)}

%{\tiny Filename: b\_strdoc.tex  Author: Hans F. Ravn  Date: 2001.02.26 }
%\setcounter{page}{??}

\newpage

{\tiny .}
\newpage




\tableofcontents

\newpage

{\tiny .}
\newpage





% dette flyttes senere:



Rettelsesforslag: at ogs� X3VQim X3VQEX TO BE DEPENDENT ON yyy,
and define  IX3VPIM\_Y etc.   At ogs� X3FX bliver koblet til
X3VSUBSTI (der s� vel hedder X3FXSUBSTI   --- eller det er den
samme der s� hedder X3RSUBSTI?)

SETS:

\subsection{X3VSTEP0}
\label{DOC-SS-X3VSTEP0}\index{X3VSTEP0}\label{X3VSTEP0}

Price dependent electricity exchange with places outside the
simulated geographical scope ('third countries') may be used
together with the fixed electricity exchange with third countries.
The price-quantity relationships are given as a piecewise step
function. There are card(X3VSTEP) (Section \ref{DOC-SS-X3VSTEP})
steps applied in simulation, and in the data card(X3VSTEP0) steps
may be given. The length (MW) of each import step is X3VIMQ and
X3VEXQ (Section \ref{X3VQIM}) of each export step. The associated
prices are X3VIMP and X3VIMP (Section \ref{X3VQIM}), respectively.
The prices are given on a yearly basis, the value for the
currently simulated year are held in IX3VPIM\_Y and IX3VPEX\_Y
(Section \ref{IX3VPIM-Y}). The exchange is assumed to be lossless
and without transmission cost.

Potential places with which there may be price dependent
electricity exchange  are given in the set X3VPLACE (Section
\ref{DOC-SS-X3VPLACE}). The simulated price dependent electricity
exchange transmission connections are specified in the set X3VX
(Section \ref{DOC-SS-X3VX}). The set RX3VSUBSTI (Section
\ref{RX3VSUBSTI}) may be used to reduce the risk of user errors.


Internal sets related to price dependent electricity exchange  are
IX3V\_Y, Section \ref{IX3V-Y},  internal parameters are
IX3VPIM\_Y, IX3VPEX\_Y, Section \ref{IX3VPIM-Y}, and associated
variables are VX3VIM\_T and VX3VEX\_T, Section \ref{VX3VIM-T}.


The set X3VSTEP0 holds the steps of the piecewise constant
function giving the relationships between quantity and price for
the price dependent electricity exchange with third countries. The
definitions is like /X3VSTEP01*X3VSTEP03/ if three steps are used.
X3VSTEP0 is used to hold data in the data base, while the subset
X3VSTEP (Section \ref{DOC-SS-X3VSTEP}) is used to indicate the
steps used in simulation.


\subsection{X3VSTEP}
\label{DOC-SS-X3VSTEP}\index{X3VSTEP}\label{X3VSTEP}

The set X3VSTEP holds the steps of the piecewise constant function
giving the relationships between quantity and price for the price
dependent electricity exchange with third countries. The set is a
subset of X3VSTEP  (Section \ref{DOC-SS-X3VSTEP0}) and the
definition is like  SET X3VSTEP(X3VSTEP0) /X3VSTEP01*X3VSTEP02/ if
two steps are used. If no exchange is wanted, then use
"IX3VSTEP(X3VSTEP)=NO".


\subsection{X3VPLACE}
\label{DOC-SS-X3VPLACE}\index{X3VPLACE}\label{X3VPLACE}


The set of regions with which there can be price dependent
electricity exchange, Section \ref{DOC-SS-X3VSTEP0}, is given by
SET X3VPLACE, defined e.g. like /X3VFARAWAY, X3VGERMAN,
X3VPOLAND/. The set is used for holding data, while the set
actually simulated is specified in the set X3VX (Section
\ref{DOC-SS-X3VX}).

\subsection{X3VX}
\label{DOC-SS-X3VX}\index{X3VX}\label{X3VX}

The combinations of RRR and X3VPLACE that are to be simulated for
price dependent electricity exchange, Section
\ref{DOC-SS-X3VSTEP0},  is given by SET X3VX(RRR,X3VPLACE). This
set may be interpreted to specify the transmission lines that are
assumed to be in operation for between regions in the simulated
geographical scope (set IR) and third countries (set X3VPLACE). If
e.g. the region 'DK\_W' is in IR and 'X3VGERMAN' is in X3VPLACE
then  X3VX('DK\_W','X3VGERMAN')=YES will specify that a
transmission line is assumed to be in operation between the two
places, and X3VX('DK\_W','X3VGERMAN')=NO that it is not. Observe
that price dependent electricity exchange only will be possible
for regions in IR, i.e. regions that are in the set C of simulated
countries.


\subsection{RX3VSUBSTI}
\label{DOC-SS-RX3VSUBSTI}\index{RX3VSUBSTI} \label{RX3VSUBSTI}


The  set RX3VSUBSTI is used in relation to the price dependent
electricity exchange with third countries, Section
\ref{DOC-SS-X3VSTEP0}. It indicates (by assigning YES) if elements
in X3VPLACE (Section \ref{DOC-SS-X3VPLACE}) is a substitute for a
region in RRR. If it is, the price dependent exchange should by
assumption only be used if the region is NOT included in a country
in set C, i.e. the set RX3VSUBSTI(IR,X3VPLACE) (where IR is a
region in C) should be empty. Observe that the only function of
the set RX3VSUBSTI is to help the user to avoid errors by printing
an error message.

The declaration is SET RX3VSUBSTI(RRR,X3VPLACE). It there are no
substitutes then define RX3VSUBSTI(RRR,X3VPLACE)=NO, otherwise
give the real information, if any, e.g.
RX3VSUBSTI('DE\_R','X3VGERMAN')=YES;


INTERNAL SETS

\subsection{IX3V\_Y}
\label{DOC-SS-IX3V-Y}\index{IX3V\_Y}\label{IX3V-Y}

The internal set IX3V\_Y(RRR,X3VPLACE,X3VSTEP0,S,T) is used to
hold the combinations (RRR,X3VPLACE,X3VSTEP0,S,T) for which price
dependent electricity exchange with third countries  (Section
\ref{DOC-SS-X3VSTEP0}) the currently simulated year  are relevant.



PARAMeTERS:

\subsubsection{X3VQIM, X3VQEX, X3VPIM, X3VPEX}
\label{X3VQIM}\label{X3VQEX}\label{X3VPIM}\label{X3VPEX}
\index{X3VQIM}\index{X3VQEX}\index{X3VPIM}\index{X3VPEX}


The parameters X3VQIM and X3VPIM hold the quantity-price
relationship for import in relation to price dependent electricity
exchange, Section \ref{DOC-SS-X3VSTEP0}, and the parameters X3VQEX
and X3VPEX hold the quantity-price relationship for export in
relation to price dependent electricity exchange.
 The declarations are: \\
 X3VPIM(YYY,RRR,X3VPLACE,X3VSTEP0,SSS,TTT) \\
 X3VQIM( RRR,X3VPLACE,X3VSTEP0,SSS,TTT) \\
 X3VPEX(YYY,RRR,X3VPLACE,X3VSTEP0,SSS,TTT) \\
 X3VQEX( RRR,X3VPLACE,X3VSTEP0,SSS,TTT)
OBS: YOU SHULD CHAGNE X3VQim X3VQEX TO BE DEPENDENT ON yyy, (and
define then IX3VQIM\_Y etc. )  AND YOU SHOULD SPECIFY THAT THEY
ARE PLaCED IN X3.INC


Unit: for X3VQIM and X3VQEX: MW, for X3VPIM and X3VPEX: Money/MWh.

X3VQIM(RRR,X3VPLACE,X3VSTEP0,SSS,TTT) holds the limit (upper
bound) on import and X3VQEX(RRR,X3VPLACE,X3VSTEP0,SSS,TTT) on
export.

\inputdata It will be assumed that prices should be positive.
For import the prices should be increasing with ord(X3VSTEP0), for
export the prices should be decreasing with ord(X3VSTEP0).



INTERNAL PARAMeTERS:

\subsubsection{IX3VPIM\_Y, IX3VPEX\_Y}
\label{IX3VPIM-Y}\label{IX3VPEX-Y}\index{IX3VPIM\_Y}\index{IX3VPEX\_Y}

The internal parameters IX3VPIM\_Y(RRR,X3VPLACE,X3VSTEP0,SSS,TTT)
and IX3VPIM\_Y(RRR,X3VPLACE,X3VSTEP0,SSS,TTT) hold the  prices for
price dependent electricity  exchange with third countries,
Section \ref{DOC-SS-X3VSTEP0},   the currently  simulated year.






VARIABLES:

\begin{itemize}
 \item []  VX3VIM\_T(RRR,X3VPLACE,X3VSTEP0,S,T) "Imported third country price dependent electricity
 (MW)"\label{VX3VIM-T}\index{VX3VIM\_T}
 \item []     VX3VEX\_T(RRR,X3VPLACE,X3VSTEP0,S,T) "Exported third country price dependent electricity
 (MW)"\label{VX3VEX-T}\index{VX3VEX\_T}

\end{itemize}





%\tiny
{\footnotesize
%{\scriptsize
\begin{center}
\begin{tabular}{|l|l|l|l|l|c|} \hline
Name           &    Domain               & Type            & Unit              & Defined in       & Page  \\
 \hline
 X3VSTEP0      & -                       & set             & -                 & SETS.INC          &  \pageref{X3VSTEP0} \\
 X3VSTEP       & X3VSTEP0                & set             & -                 & SETS.INC          &  \pageref{X3VSTEP} \\
 X3VPLACE      & -                       & set             & -                 & SETS.INC          &  \pageref{X3VPLACE} \\
 X3VX          & (RRR,X3VPLACE)          & set             & -                 & SETS.INC          &  \pageref{X3VSTEP} \\
 RX3VSUBSTI    & (RRR,X3VPLACE)          & set             & -                 & SETS.INC          &  \pageref{RX3VSUBSTI} \\
 IX3V\_Y       & (RRR,X3VPLACE,X3VSTEP0,S,T) & set         & -                 & SETS.INC          &  \pageref{IX3V-Y} \\
 X3VQIM        & (???,RRR,X3VPLACE,X3VSTEP0,SSS,TTT) & parameter  & MW         &            &  \pageref{X3VQIM} \\
 X3VQEX        & (???,RRR,X3VPLACE,X3VSTEP0,SSS,TTT) & parameter  & MW         &            &  \pageref{X3VQEX} \\
 X3VPIM        & (YYY,RRR,X3VPLACE,X3VSTEP0,SSS,TTT) & parameter  & Money/MWh  &            &  \pageref{X3VPIM} \\
 X3VPEX        & (YYY,RRR,X3VPLACE,X3VSTEP0,SSS,TTT) & parameter  & Money/MWh  &            &  \pageref{X3VPEX} \\
 IX3VPIM\_Y     & (RRR,X3VPLACE,X3VSTEP0,S,T)& parameter          & Money/MWh                & SETS.INC          &  \pageref{IX3VPIM-Y} \\
 IX3VPEX\_Y     & (RRR,X3VPLACE,X3VSTEP0,S,T)& parameter          & Money/MWh                & SETS.INC          &  \pageref{IX3VPEX-Y} \\
 VX3VIM\_T     & RRR,X3VPLACE,X3VSTEP0,S,T) & variable            & MW                 & BALMOREL.GMS          &  \pageref{VX3VIM-T} \\
 VX3VEX\_T     &(RRR,X3VPLACE,X3VSTEP0,S,T) & variable            & MW                 & BALMOREL.GMS          &  \pageref{VX3VEX-T} \\

\hline
\end{tabular}
\end{center}





% ----------------------------------------------------------------------------
%\setcounter{section}{9}
%\section{THE BALMOREL MODEL STRUCTURE}

\section{Introduction}

This paper  documents the Balmorel model structure and describes
some of the technicalities in the  model.


The  Balmorel model was developed for the analysis of the power
and CHP (combined heat and power) sectors in the Baltic
Sea\index{BSR, Baltic Sea Region} Region.  The model is  directed
towards the analysis of  policy questions to the extent that they
contain substantial international aspects.


The model is implemented in the GAMS modelling language. For the
present, we assume that the reader is familiar with this. An ultra
short introduction is given in Section \ref{DOC-SS-GAMS-intro}.
The files that contain the model as specified in the GAMS
language is in fact a good documentation of the model.
 In the present document we have
aimed at presenting a documentation that is structured
differently and which  presents additional information and
overview  relative to that in the GAMS model files.

In particular note that emphasis in the present document is on the
model structure. By this is meant that actual values of
parameters  are not given nor are  the actual members of the sets
used in the model. However, the names of the parameters and sets
are specified and so is their functioning in the model.

Further note that  the present document treats  the GAMS part of
the model. Various facilities are provided that permit working in
a spreadsheet\index{spreadsheet} environment, however, this will
not be treated here.

For the exact documentation of the model, input data and set
members, see the model files.


This  document is part of a series that together documents the
Balmorel model:
\begin{itemize}
  \item[] Balmorel: A  Model for Analyses of the Electricity and
  CHP Markets in the Baltic Sea Region (Main Report)
  \item[]The Balmorel Model: Theoretical Background
  \item[]The Balmorel Model Structure  (this document)
  \item[]Balmorel: Data and Calibration
   \item[]Balmorel: Getting  Started
 % \item[]Balmorel: Observations on modelling and results
\end{itemize}

These  documents and further information, including application
examples, may be found at the Balmorel homepage\index{homepage,
Balmorel}: www.Balmorel.com.

\subsection{This version}

The description given here is for  version\index{version number}
\Versionname   of the model.

There have been a number of changes in the Balmorel model
structure from the first version (2.05 (March 2001))  to the
present one. The changes will not be documented here, however you
may contact us for further information, cf. the Balmorel homepage.

The model is implemented in GAMS, Version \VersionGAMS , see
further Section \ref{DOC-SS-GAMS-intro}.

The model has been developed on
PC/Windows.\index{PC-Windows}\index{Windows}


\subsection{Data structure, model and simulation}
\label{DOC-SS-Structure-Model-Simulation}

We distinguish here between three concepts: that for which
data\index{data structure} structures exist, that which is
modelled \index{modelled} (i.e., that for which a meaningful data
set has been entered into the data files), and that which is
simulated. \index{simulated}

When we refer to that for which data structure exist we have in
mind what the data structures actually allow of data input, this
could be seen as the potentials of the database.  The restriction
on this is the sets, parameters, etc. that are declared. For
instance, the years for which demand may be given could be from
1995 to 2030. This set of years is given by the set YYY. Other
triple letter sets,  AAA, RRR, CCC, SSS, TTT, GGG and FFF have the
same function.

When we refer to that which is simulated we refer to a specific
simulation. Such a simulation will for instance only concern the
subset Y of the above mentioned years, e.g. the years 1995 to
2010.  The GAMS syntax requires that Y be a subset of YYY.

Further, in order to make a meaningful simulation, data must be
available for the simulation. That for which data is available is
referred to as that which is modeled (or that for which a data set
exist).

Hence that which is simulated   must be a subset of that which is
modeled, and that again must be a subset of that for which data
structures exist. Assuming that the user is reasonable, it is
necessary only to distinguish between that which is simulated
and that  for which data structures exist.

Presently data structures cover the period 1995 to 2030, and for
this period there has also been provided data.  Hence, what is
modeled is (as far as years are concerned)  identical to that for
which data structures related to years exist.

Observe that the aim of the present document is the description of
the structure of the model. Therefore the actual parameter values
and set members given here should be considered as examples,
rather than that actually  used. See further Section
\ref{DOC-SS-Version-numbering} for some specifications (and
Sections \ref{DOC-SS-Sets-Exceptions} and
\ref{DOC-SS-Parm-Exceptions} for exceptions).




\subsection{A short introduction to GAMS terminology}
\label{DOC-SS-GAMS-intro}

\index{GAMS}\label{GAMS}

GAMS is the acronym for General Algebraic Modeling System. The
system is suitable for formulation, documentation  and solution
of large  mathematical models.


Generally, we assume that the reader is sufficiently familiar with
the GAMS language. A User's Guide,\index{User's Guide} \index{GAMS
User's Guide} a Tutorial, and other relevant information about the
GAMS modeling system may be found at \index{homepage,
 GAMS}\index{GAMS homepage} www.gams.com.


For the purpose of the following description, we shall only point
out a few basic things. The idea is therefore not to give a
rounded presentation of the GAMS modeling language,  implying
e.g. that subjects that can be relatively easily   understood by
reading the model will not be explained.

\subsubsection*{GAMS version}

The Balmorel model is implemented in version \VersionGAMS .
 \label{DOC-pageref-GAMSversion} This version has (minor)
limitation relative to later versions.
 Version \VersionGAMS \mbox{ }
was chosen to ensure compatibility with existing installations of
GAMS. Later versions of GAMS have backwards compatibility such
that Balmorel may execute on them.

The limitations   observed by us following from restriction to
version \VersionGAMS\mbox{ } as observed by us are (i) the length
of identifiers and labels are restricted to ten characters, (ii)
the use of WHILE is not possible, (iii) do not use
\index{tabulator} tabulator.  To check whether the syntax of
 version \VersionGAMS \mbox{ }
is followed, a '\$use225' \index{\$use225}may be inserted in the
first line of the program, with the \$ in the first position. (But
we have observed that this feature unfortunately does not enforce
limitation to ten characters.)



\subsubsection*{Sets}
\index{SET}

The GAMS language contains among other elements SETS\index{SET},
various parameter values (exogenously\index{exogenous} given)
indicated by SCALAR\index{SCALAR} or PARAMETER\index{PARAMETER}
(and possibly entered in a TABLE\index{TABLE}),
(endogenous\index{endogenous}) VARIABLES\index{VARIABLE}, and
EQUATIONS\index{EQUATION}. A set of EQUATIONS constitute a
MODEL\index{MODEL}.

Sets are the basic building blocks of GAMS, corresponding  to the
indices in an algebraic representation of a model. The   set is
declared by SET or SETS, followed by the name
(identifier)\index{identifier} of the set. The definition of the
set is the  specification of the contents of the set, i.e., the
elements\index{element in set} or the members\index{member of
set}  of the set.  If for example the model contains three
countries, this may be specified as
\begin{itemize} \item[] SET COUNTRIES / DENMARK, NORWAY, SWEDEN
/;  \end{itemize} where it is seen that slashes (/)
 \index{/, slash, delimiter}
are used as delimiters of the definition.

As seen, in the GAMS system the creation of entities like SETS
(but also PARAMETERS etc.) involve two parts: a
declaration\index{declaration} and an
\label{DOC-Setdeclarationdefinition} assignment\index{assignment}
or definition\index{definition}. Declaration means declaring the
existence of something and giving it a name. Assignment or
definition means giving something a specific value or form.
Declaration and definition  may be done in separate statements  or
(except for EQUATIONS) in the same statement (as above).

Sets may be given as subsets of previously defined sets, e.g.,
\begin{itemize} \item[] SET HYDROCOUNT(COUNTRIES)
  / NORWAY, SWEDEN /; \end{itemize}

Sets may have their membership explicitly defined (i.e., the
labels are given between slashes)  at the time the SET itself was
declared (in which case the sets are called
  static sets\index{set, static}\index{static set}),
  or the membership may be defined by assignment
   (dynamic sets\index{set, dynamic}\index{dynamic set}),
 see further Section \ref{DOC-SSS-staticanddynamicsets}.


%The Balmorel model does not use empty sets.\index{empty set}

A shorthand asterisk notation\index{asterisk notation} like SET
/S1*S52/\index{*} may be used to indicate the labels S1, S2, ...,
S52.

The entry order\index{entry order} of the labels is the order in
which the individual labels first appear in the program, either
explicitly or as  a result of using the shorthand asterisk
notation. The entry order has implications for e.g.
LOOP\index{LOOP} and DISPLAY\index{DISPLAY} statements. It also
has implications in relation to ordered sets, see Section
\ref{DOC-SSS-Orderedsets}. Section \ref{DOC-SS-GAMSoutput}
describes how a list describing the entry order may be obtained.

The ALIAS\index{ALIAS} statement is used to define sets that are
identical, but which have different identifiers (names). Hence,
in relation to the above example, ALIAS (COUNTRIES,C) declares
the set C and defines it  to be identical to the set COUNTRIES.

Reference to individual members of sets may be given using
quotation marks, thus in relation to the above set an individual
country may be addressed as "DENMARK" or 'DENMARK'.
 \index{' (quotes)}\index{quotes}

Sets may  be one-dimensional or
 multi-dimensional\index{dimension (sets)}
 and they may be ordered\index{ordered set} or
 unordered\index{unordered set}\index{set, (un)ordered}, see
further Section \ref{DOC-SSS-Orderedsets}.

\subsubsection*{Scalars and Parameters}
\index{scalar} \index{parameter}

The parameters \index{parameter} and scalars \index{scalar} are
used to specify exogenous\index{exogenous} values.

Parameters are specified for some or all elements in a set,  or
for  cartesian products of sets. The parameter DH, for instance,
specifies the annual heat demand in an area (e.g., a city).
Therefore this parameter is declared as DH(YYY,AAA), and hence it
is clear that it refers to all combinations of elements (also
referred to as the set\index{set product} product  or
cartesian\index{cartesian product} product) in the sets YYY (all
years)  and AAA (all areas).


Scalars are also used to  specify exogenous values, however,
scalars are not related to any sets.

Parameter and scalar  values  may be specified
 directly by the user, for parameters
often in a TABLE\index{TABLE}, or they may be calculated in the
model from other values. Parameters and scalars  that are not
explicitly assigned a value are automatically set to the default
value zero.


\subsubsection*{Variables}
\index{VARIABLE}

The variables of the model are the endogenous\index{endogenous}
values, i.e., those entities that are determined internally in
the model by solving the specified model. In the Balmorel model, a
typical examples is the  generation of electricity on a specific
generation unit in a particular time period.

Variables are declared by the VARIABLE\index{VARIABLE} statement,
and  they may be  declared to be e.g.  POSITIVE\index{POSITIVE}
(meaning that they can attain only non-negative values) or
FREE\index{FREE} (meaning that they can attain any real values).

The values of the  variables are to be found according to the
problem type specified, typically by optimisation. However,
variables may have their values fixed (by appending
.FX\index{FX}), or they may be bounded downwards and/or upwards
(by appending .LO\index{LO} and/or .UP\index{UP}, respectively).

The optimal values of variables are referred to by the suffix
.L\index{L, level}.
 Marginal values to equations are referred to by the suffix .M\index{M, MARGINAL}.


\subsubsection*{Naming restrictions}

Identifiers \index{identifier} are the  names given to
%  data types, i.e.
SETS, PARAMETERS, SCALARS, VARIABLES, EQUATIONS and MODELS.
%to 10 characters (letters or
%digits) starting with a letter. [NB: DO WE RESTRICT TO THIS - OR
%DO WE USE A LATER VERSION?]
A label\index{label} is the name of a set element. The types and
number  of characters  of  identifiers and labels are limited
according to the GAMS syntax (page
\pageref{DOC-pageref-GAMSversion}). In addition, conventions are
applied in the Balmorel model  (involving among other things the
restriction to ten characters) (Section
\ref{DOC-SS-Naming-conventions}).
%We use only the
%so called unquoted\index{unquoted form}\index{quoted form} form,
%that can be up to 10 characters long,  consisting of a
%combination of letters, digits, "-", "+" and "\_", starting with a
%letter or a digit.




Obviously, words that have predefined meanings in the GAMS
language
 (reserved words, key words)\index{reserved words}\index{key words}
  can not be used (e.g., MODEL, SET, INF, TABLE, LP).
   \index{linear programming}\index{LP, linear programming}

And finally: GAMS is not case sensitive,\index{case sensitive}
thus e.g. the identifiers balmorel,  Balmorel and BALMOREL are
interpreted to be identical.  (But observe, that the editor that
the user applies  may very well be case sensitive.)


\subsubsection*{Arithmetic expressions}

The language permits the formulation of arithmetic expressions in
a form that is fairly easily understood. Thus,  e.g. the
expression  SUM(T, X(T))\index{SUM}\label{SUM} can be read to mean
the sum over the elements in the set T  of the quantities X(T),
where X is a vector with one element for each member in the set T.
Similarly, \index{PROD}PROD, \index{SMAX}SMAX and \index{SMIN}SMIN
means the product, maximum value and minimum value, respectively,
over the specified set. In contrast, MIN\index{MAX} and
MAX\index{MIN} operate on lists of arguments.


The  interpretation of the arithmetic operators  "+"\index{+,
addition}, "-"\index{-, subtraction}, "*"\index{*, multiplication}
and "/"\index{/, division}  is straightforward. The traditional
relational operators $<$, $\leq$, $=$, $\geq$, $>$, $\neq$ are
specified as such, or as LT, LE,  EQ, GE, GT, NE
 \index{LE, less or equal}
  \index{LT, less than}
  \index{EQ, equal to}
  \index{GE, greater or equal}
  \index{GT, greater than}
   \index{NE, not equal to} respectively, except in
EQUATIONS\index{EQUATION}, where $\leq$, $=$ and $\geq$ are
specified as
 =L=\index{L, =L=}\index{=L=}, =E=\index{E, =E=}\index{=E=}
  and =G=,\index{G, =G=}\index{=G=} respectively. The "="\index{=} is used in
assignments\index{assignment}, e.g. PI=22/7.


\subsubsection*{Extended arithmetic}
\index{INF}\label{INF}

Extended arithmetic\index{extended arithmetic} is allowed to
include the value infinity, denoted INF\index{INF}. Thus, 6/INF is
evaluated to  zero, INF+INF is evaluated to INF,  INF-100 is
evaluated to INF, 8*INF is evaluated to INF and  -INF is minus
infinity. The expressions 0*INF and INF-INF are illegal. Also
related to  the implementation of extended arithmetic are
NA\index{NA} (not available: thus e.g. 7+NA evaluates to NA),
UNDF\index{UNDF} (undefined) and EPS\index{EPS} (a number very
close to but different from zero).



\subsubsection*{Conditional, logical, dollar expressions, exceptions}
 \index{dollar operator}\index{exceptions}
 \index{conditional expression} \index{logical expression}

Various means may be used in order to formulate  conditional
expressions. Constructions using  IF, ELSE and ELSEIF  are similar
to those found in common programming languages. Logical
expressions may be made using NOT, AND,
OR\index{NOT}\index{AND}\index{OR}\index{XOR} and XOR. Numerical
values of parameters and scalars may be interpreted as logical
values using the conventions that the value 0 means NO and other
values means YES.  GAMS further has the dollar (\$) operator to
permit conditional operations, loosely speaking corresponding to a
conventional IF condition. An expression like SUM(X\$MYPARM(X),
...) (where X is a set and MYPARM a parameter) is interpreted as
summation of MYPARM over all those elements in X for which
MYPARM(X) is not 0. (Due to the data representation used in GAMS,
this is very efficient if MYPARM(X) is 0 for most elements in X.)
Consult the GAMS User's Guide.


\subsubsection*{Sequence of statements, flow control}
\index{sequence of statements}

The sequence of the statements in GAMS is important. The
statements of the model are normally executed  sequentially.

However, control of this flow may be performed by using
\index{LOOP}LOOP, \index{IF-ELSE}IF-ELSE (including extensions
using ELSEIF\index{ELSEIF}), \index{FOR}FOR and \index{WHILE}WHILE
statements. The LOOP statement causes the execution  of the
statements within the scope of the loop for each member of the
driving set(s) in turn. Thus e.g. "LOOP(C, ... )"   is similar to
"for all elements in turn in set C do ...". The order of execution
within the loop is the entry order  (Section
\ref{DOC-SSS-Orderedsets}) of the labels.


\subsubsection*{Entry of numerical data}
\index{data entry}

Numerical data may be entered along with declaration of
PARAMETERS\index{PARAMETER} or SCALARS\index{SCALAR} or by
assignment. For multi dimensional parameters, the
TABLE\index{TABLE} is convenient.  The layout of a TABLE is quite
flexible. Thus, if a table has too many columns to fit nicely on
a single line, then the columns that do not fit can be entered
below (using the symbol "+"\index{+, table continuation} for
continuation); thus, row labels, unlike column labels,  may be
duplicated. Data may be entered directly, or they may be
calculated and assigned using "="\index{=}. Observe that
declarations\index{declaration} can not come after
assignments\index{assignment}, and an assignment
overwrites\index{overwrite} previous assignments. The extended
arithmetic symbols INF\index{INF}, NA\index{NA} and
EPS\index{EPS} may be used in input.


\subsubsection*{Default data}
\index{default}\label{default}

It is very useful to note that if data are not entered for
parameters or scalars then by default\index{default} the value
zero is assigned.

However, not all elements can be given by default, at least one
must explicitly be given a value, otherwise it is considered an
error.




\subsubsection*{Comments and explanations}
Comments\index{comments} may be entered in a line, if the line has
a "*"\index{*, comment} in the first column.  In particular, this
may be used for commenting out\index{comment out} a command.
Comments may also be inserted between "/*"
 and "*/"\index{/**/, comment}
 (provided it is preceded by
"\$ONINLINE"\index{ONINLINE}, preferably placed near the top of
the BALMOREL.GMS file), or between "\$ONTEXT" and
"\$OFFTEXT"\index{TEXT} (where the \$'s must be in the first
position of a line).

It is possible to associate explicative text with set element,
parameters etc. For example,   "SET CCC    All Countries"  or
"PARAMETER DE(YYY,RRR) Nominal annual electricity demand". The
explicative text may be given between
  quotes\index{quotes}
 (and must be so, if special characters are used). Such
text may be displayed in output, see Section \ref{DOC-S-Output}.


\subsubsection*{Include files}
\index{include file}\label{include file}\index{dollar include}

In GAMS, the input may be split over several files. This is
handled by include files. This means that the content of a file
(typically with the extension "inc") may during compilation of the
model be included in another file. Thus, for instance the contents
of the file "TRANS.INC" is placed in this other file (e.g.,
BALMOREL.GMS) at the place where the statement "\$INCLUDE
TRANS.INC"\index{INCLUDE} (or "\$INCLUDE "TRANS.INC";", but not
"\$INCLUDE TRANS.INC;") is found.

\subsubsection*{Equations and Model}

See Section \ref{DOC-S-Model}.

\subsubsection*{Solver}

For the solution of the model a solver\index{solver} has to be
used. Thus, the GAMS system passes the model to the solver, which
solves the problem and passes the solution and related information
back to the GAMS system, which in turn permits presentation of the
solution and related information in various forms. Also default
and error information, e.g. if the problem does not have a
solution, will be returned.

\subsubsection*{Output, Errors etc.}

See Section \ref{DOC-S-Output}.



\subsection{Naming conventions}
\label{DOC-SS-Naming-conventions} \index{naming}

We have tried to select names for the various sets, parameters
etc. to facilitate the recognition of the meaning from the name.
Observe that names are limited to no more than ten characters;
this facilitates the printing of output and compatibility with
older versions of GAMS, cf. page
\pageref{DOC-pageref-GAMSversion}. The following conventions for
names are used:

\begin{itemize}
\item[]Single letters:
\begin{itemize}
\item[] D: demand (e.g., DE: demand for electricity, DH: demand for
heat)
\item[] E: electricity
\item[] F: fuel
\item[]F: flexible (e.g., DEF: flexible (i.e. elastic\index{elastic demand}) electricity demand,
   DHF:  flexible\index{flexible demand} heat demand)
\item[] G: generation, or generation   technologies (e.g., GE:
generation of electricity, GDINVCOST: investment cost for
generation technology)
\item[] H: heat
\item[] I: internal (set, scalar or parameter)
\item[] K:  capacity
\item[] M: emission
\item[] N: new
\item[] O: related to output from simulations
\item[] Q: equation
\item[] V: variable (see also VQ)
\item[] X: electricity transmission
%\item[] T: to the name of a set this may indicate that the set contains "all" /"total" elements.
\end{itemize}

%\item[]If the second letter is P, this may indicate price, e.g., EP
%and HP for electricity and heat price, respectively.
\item[]Suffixes: \index{suffix}
\begin{itemize}
  \item[] \_T or T: the finest division of time is the subdivision of the season
   (e.g., one hour; or  night-period, day-period,
  peak-hour) (and  implicitly or explicitly  also contains season and year
  index), cf. Section \ref{DOC-SSS-Timewithinyears}
  \item[] \_S or S: the finest division of time is the subdivision of the year into seasons
    (e.g. summer, winter; or  Jan, ... , Dec)
  (and  implicitly or explicitly  also contains  year index), cf. Section \ref{DOC-SSS-Timewithinyears}
  \item[] \_Y or Y: the year, annual
\end{itemize}
\item[]Further:
\begin{itemize}
 \item[] BPR: back pressure generating technology
 \item[] CAL: calibration \index{CAL}
 \item[] CND: condensing generating technology
 \item[] DIS: distribution \index{DIS}
 \item[] EXT: extraction generating technology
 \item[] FLH: full load hours\index{FLH}\index{full load hours}
 \item[] FX: fixed, given, exogenous \index{FX}
 \item[] HOB: heat only boiler generating technology
 \item[] HY: hydro technology \index{HY} (GHYRS: with seasonal reservoir, GHYRR: run-of-river)
 \item[] INI: initial \index{INI}
 \item[] INV: investment \index{INV}
 \item[] LIM: limit \index{LIM}
 \item[] OM: operation and maintenance   \index{OM} (OMF: fixed,
 OMV: variable)
 \item[] POL: policy (with respect to taxation, emission quota) \index{POL}
 \item[] SOL: solar, sun \index{SOL}
 \item[] STO: storage (typically daily, HSTO: heat, ESTO:
               electricity)
 \item[] VAR:  variation over the time segments  of the day and year \index{VAR}
 \item[] WND: wind \index{WND}
 \item[] WTR: water (energy source) \index{WTR}
 \item[] VQ: variable that ensures feasibility in an equation
\end{itemize}
\end{itemize}




\subsection{User interface}
 \index{user interface}\label{DOC-SS-userinterface}

The GAMS program is contained in ascii files, hence any editor
that can produce, read, modify and save such files may be used.
Special editors suited for GAMS exist. Do not use tabulator, cf.
page \pageref{DOC-pageref-GAMSversion}.

The user must make sure that the GAMS system is set up properly
and in relation to the file  structure described in Section
\ref{DOC-S-Files}.

For the Balmorel project, various spreadsheet\index{spreadsheet}
interface facilities are available, cf. the home page.







\subsection{Limitations and expected modifications}

We are not aware of any errors in the present version, however, it
includes elements that are not yet completely integrated:
\begin{itemize}
\item The set IST(S,T) is introduced, however, advantage can not
be taken of it since the implementation is not complete. (Not yet:
SET ISTT(T); SET ISTS(S); defined
by\index{ISTS}\index{IST}\index{ISTT}
 LOOP((S,T) \$IST(S,T),ISTT(T)= YES);
 LOOP((S,T) \$IST(S,T),ISTS(S)= YES).
ISTS is used to hold the subsets of S for which there is at least
on T such that (S,T) is in  IST(S,T). Introduced to get more
flexibility in definition of time segmentation of the year.
Observe that it can not be used i definition of equations that
involve lead or lags (typically i relation to storage)). Section
\ref{IST}.


\item  GDCH4 emission not fully implemented.\index{GDCH4}
\item FDN2O emission  not fully implemented.\index{FDN2O}
\item GDLIFETIME not used (so for investments, the standard period
implied by ANNUITYC is used).\index{GDLIFETIME}
\item Dual fuels not fully implemented.\index{dual fuels}

\item Minimum and maximum limits for hydro power storage works
only correctly for problems without new investments permitted for
this type.   \index{hydro power storage}
\item Heat storage only implemented with no new investments.\index{heat storage}
\item YVALUE. This parameter holds the numerical value related to the
years in set YYY, Section \ref{YVALUE}. Application not fully
implemented.\index{YVALUE}
\end{itemize}

These limitation will be eliminated after revision.





\section{File  Structure}
\label{DOC-S-Files} \index{files}

The model is distributed over a number of files. In this section
we give an overview.

The files are ascii\index{ascii} files, cf. Section
\ref{DOC-SS-userinterface}.

%(For compatibility, it is advocated that tabulators\index{tabulator} not used:)

In the
%  $\backslash$
Balmorel directory   (e.g. taken from the home page
www.balmorel.com at the Internet or from a diskette) you should
find the following subdirectories\index{subdirectory}:
\begin{itemize}
  \item Model\index{Model subdirectory}
  \item Data-pre-inc\index{Data-pre-inc subdirectory}
  \item PrintInc\index{PrintInc subdirectory}
  \item PrintOut\index{PrintOut subdirectory}
  \item LogError\index{LogError subdirectory}
  \item Documentation\index{Documentation subdirectory}
\end{itemize}
This should be copied to the user's computer.
 %, cf. Figure  ??
 %\ref{DOC-F-filestructure1}.
The user may find it expedient to maintain this version
unaltered, and therefore for an application a copy should be
made, e.g. with the name "Balmorel-MyVersion1", containing the
above mentioned subdirectory structure and the file structure
mentioned in the sequel.
%cf. Figure \ref{DOC-F-filestructure2}.

File\index{file name} and path names\index{path name} should not
include special characters like {\ae}, {\aa}, {\H{o}}, \l, \ss,
\oe, or similar, and should be limited to eight characters.



%\begin{figure}
%\hspace{5cm} [........NOT INCLUDED YET........]

% \caption{Filestructure}
% \label{DOC-F-filestructure1}
%\end{figure}

Of the above mentioned subdirectories, Model, PrintInc, PrintOut
and LogError are mandatory in order to run the model. Other
subdirectories may exist.



\subsubsection*{Subdirectory Model}

 The Balmorel model is located  in the subdirectory
Model.\index{Model subdirectory} Here
  the following  files are found:
\begin{itemize}
 \item  \index{BALMOREL.GMS}BALMOREL.GMS: main model file (running this means running
        the model)
\item  \index{SETS.INC} SETS.INC: contains  the declaration and definition
  of the static  sets (see Section \ref{DOC-SSS-staticanddynamicsets}) in the model
\item  \index{GEOGR.INC}  GEOGR.INC:  contains    most values specific for
        geographical entities
\item  \index{FUEL.INC}  FUEL.INC:  contains   values specific for the different fuel types
\item  \index{TECH.INC}  TECH.INC:  contains    the parameters of the generation technologies in use
\item  \index{TRANS.INC}  TRANS.INC:  contains   the parameters related to electricity
  transmission between regions
\item  \index{VAR.INC}  VAR.INC:  contains   daily and seasonal variations of
        all relevant parameters
\item  \index{DE.INC}  DE.INC: contains annual values of electricity demand
\item  \index{DH.INC}  DH.INC: contains annual values of heat demand
\item  \index{FUELP.INC}  FUELP.INC: contains annual values of fuel prices
\item \index{MPOL.INC}   MPOL.INC: contains annual values of environmental policy
\item  \index{GKFX.INC}  GKFX.INC: contains annual values of exogenously specified generation
capacities
\item \index{X3.INC} X3.INC: contains annual values of electricity
exchange with countries (regions) not explicitly modeled
\end{itemize}

These files together  contain the model. The BALMOREL.GMS file is
the main file, while the others are include files (Section
\ref{DOC-SS-GAMS-intro}) which during compilation of the model
automatically are inserted into the appropriate places in the
BALMOREL.GMS file.

All user specified labels and numerical data are found in the
include files.

The distribution of the different sets, parameters etc. over the
files is done according  to the following  principles,

\begin{itemize}
  \item  SETS.INC contains declaration and definition  of all static
  sets (see Section \ref{DOC-SSS-staticanddynamicsets}).
  Also the scalars STARTYEAR and  YEARINC are in this file.

 \item  TECH.INC contains global information about
generation technologies, i.e., information that is not specific to
geography. Therefore, all data structures that contain the index
pair (GGG,GDATASET), and no other indexes are in this file.

 \item FUEL.INC
contains global information about fuels, viz., information that is
not specific to geography or time. Therefore, all data structures
that contain the index pair (FFF,FDATASET)   and no other indexes
are in this file.

 \item  TRANS.INC contains information about
transmission conditions  between pairs of regions. Therefore, all
data structures that contain the index pair  (IRRRE,IRRRI) and no
other indexes are in this file.

 \item  GEOGR.INC   contains all
information that is specific with respect to geography, except
that which contains information relative to time (i.e. except that
which contains indexes YYY, SSS or TTT), and except that which
contains  information about transmission between regions (i.e.,
except that which contains the index pair  (IRRRE,IRRRI)).


\item All data that depend on the year are placed in separate
files: \begin{itemize}
 \item  DE.INC contains annual electricity demand for each
year.
 \item  DH.INC contains annual heat demand for each  the year.
 \item  GKFX.INC contains user specified installed generation capacity for each
year.
 \item  X3FX.INC contains  annual  electricity exchange with
third regions.
 \item  FUELP.INC contains fuel prices for each year.
 \item  MPOL.INC contains environmental policy data for each year.
 \end{itemize}

 \item  VAR.INC contains
information about the variation of parameters within the year.
Therefore, all data structures that contain  indexes SSS and/or
TTT are in this file.
\end{itemize}


See Section \ref{DOC-S-Overview} for exact locations.

After running the model, GAMS will automatically have created two
additional files that will   be placed in the subdirectory Model:
\begin{itemize}
  \item BALMOREL.LST \index{LST file}
  \item BALMOREL.LOG\index{LOG file}
\end{itemize}
See Section \ref{DOC-S-Output} for more on this.




\subsubsection*{Subdirectory Data-pre-inc}

This\index{Data-pre-inc subdirectory} optional  subdirectory
contains various data and facilities for preparing some of the
include files found in subdirectory Model. The user may  find it
expedient  to create subdirectories. This will not be described
further in the present  document.





\subsubsection*{Subdirectory PrintInc}


In addition to the files mentioned above there are in the
subdirectory print \index{Print-inc subdirectory} auxiliary  files
that are not proper  part of the model, but which provide various
possibilities for generating output from successful model runs:
\begin{itemize}
\item  PRINT1.INC: declares file names for predefined output and
various parameters and sets that may be useful for creating output
\item  PRINT2.INC:
declares parameter names for output that can be written for each
year of the simulation
 \item  PRINT3.INC: calculates the values of the  parameters
 declared in PRINT2.INC
 \item  PRINT4.INC: specifies which output from the most recently simulated year to write to a file,
  by including files found in the subdirectory print
 \item  Several output generating files are found in   the subdirectory
print. They are all auxiliary  include-files. They are controlled
by the above PRINT*.INC  files. See Section
\ref{DOC-S-Interal-Parm-Output}  and Section \ref{DOC-S-Output}.
\end{itemize}

These files may be omitted (commented out)\index{comment out} in
the BALMOREL.GMS file without effecting the model itself.
However, if they are not commented out, they must exist. We refer
to such additional components as auxiliary\index{auxiliary}
parts. See Section \ref{DOC-S-Output}.







%\begin{figure}
%\hspace{5cm}  [........NOT INCLUDED YET........]
% \caption{Filestructure 2}
% \label{DOC-F-filestructure2}
%\end{figure}

\subsubsection*{Subdirectory LogError}
\index{LogError subdirectory}

 This subdirectory contains auxiliary parts for checking
the input data and monitoring the solution of the model.

\begin{itemize}
\item  ERROR1.INC: declares file names for predefined output,
ERROR.OUT and LOG.OUT
\item  ERROR2.INC: makes some simple checks of the input data and prints the
conclusion to the file ERRORS.OUT. A summary is printed in the
file LOG.OUT.
\item  ERROR3.INC: makes some simple checks immediately before optimisation starts
 and prints the conclusion to the file ERRORS.OUT. A summary is printed in the
file LOG.OUT.
\item  ERROR4.INC: makes some simple checks immediately after optimisation starts
 and prints the conclusion to the file ERRORS.OUT. A summary is printed in the
file LOG.OUT.
\item  LOG.INC: prints  in
       the file LOG.OUT a summary of the contents of the file ERRORS.OUT,
       and in addition  a status for the solution of the model.
\end{itemize}

These files may be omitted (commented out)\index{comment out} in
the BALMOREL.GMS file without effecting the model itself. However,
if they are not all commented out, at least ERROR1.INC must be
included. See Section \ref{DOC-S-Output} and Section
\ref{DOC-S-Errors}.

\subsubsection*{Subdirectory Output}

All the output specified by files in the print
 %  and errorlog
subdirectory  is placed in the  output subdirectory.\index{Output
subdirectory}   The output generated  from Error    and Log files
is placed in the LogError subdirectory.


Output generated automatically by the GAMS system  will be placed
in the  Model subdirectory, cf. above.


In the output subdirectory  additional facilities for presentation
of output (e.g. in the form of spreadsheets\index{spreadsheet})
may be located.  See Section \ref{DOC-S-Output}.

Observe that by running GAMS any previous output in existing files
may be overwritten. It is therefore recommended that the user
creates a number of subdirectories (e.g. to the subdirectory
Output) where output can be saved before the next GAMS run.
%See Figure \ref{DOC-F-filestructure2} for inspiration.



\subsubsection*{Subdirectory Documentation}
\index{Documentation subdirectory}

In this  optional  subdirectory  various documentation files may
be placed. Also here the user may  find it expedient  to create
subdirectories.
 %, see Figure  ??
%\ref{DOC-F-filestructure2}  for inspiration.


\section{Sets}
\label{DOC-S-Sets}

In this section we describe the sets in the model. According to
the distinctions in Section
\ref{DOC-SS-Structure-Model-Simulation} we shall describe the
sets  in the data structure, and  the subsets that may be used
for specific simulations. The names of the sets in the first
group usually have triple letters (e.g., CCC, TTT).
%, cf. Section \ref{DOC-SS-Naming-conventions}.


Most of the sets have their members\index{member of set}
(elements, labels)\index{element in set}\index{label} specified by
the user. We refer to those sets as input sets\index{input set}.
Some sets are derived automatically from previously given sets,
we refer to such sets as internal
   sets\index{internal set},
Section \ref{DOC-SS-Internal-Parm-Sets}.  Section
\ref{DOC-SS-Sets-Exceptions} lists some restrictions.



All sets are declared and defined   in the file SETS.INC, see
Section \ref{DOC-S-Files}, except for internal sets (Section
\ref{DOC-SS-Internal-Parm-Sets}) that are declared and defined in
the file BALMOREL.GMS.




\subsection{Geography} \label{DOC-SS-Geography}
\index{geography}



The model permits specification of geographically distinct
entities. The types of geographical entities are Areas, Regions,
and Countries. These entities are in relation to the data
structure specified by the sets AAA, \index{AAA} RRR \index{RRR}
and CCC,\index{CCC} and for the subsets to be used for simulation
by the subset C.\index{C}

Each country is constituted of one or more regions while each
region  contains zero or more areas. Any area must be included in
exactly one region, and any region must be included in exactly one
country, see also Figure \ref{DOC-F-geography} and Section
\ref{DOC-SSS-Relations-CRA}.




\begin{figure}
\includegraphics[clip,width=0.9\textwidth]{geographyeps.eps}
 \caption{The geographical entities.}
\label{DOC-F-geography}
\end{figure}



The areas are the building blocks with respect to the geographical
dimension. Thus, for instance all generation and generation
capacities are  described at the level of areas, and so are all
aspects of heat demand; see the list below.

Areas are classified and grouped in a number of ways. The
collection of subsets of areas into regions was described above,
and further examples are given in Section
\ref{DOC-SSS-IAURBH-IARURH}.





Electricity balances are given on a regional basis. For each
element in RRR electricity generation comes from the elements of
AAA located in RRR. Hence, for each region an electricity balance
must be fulfilled, but unlike heat, electricity may be exchanged
between regions. Such transmission, and their constraints, losses
and costs, are the motivation for the concept of regions. In
contrast to this, transmission of heat between areas is not
possible.

A number of regions (i.e., a nonempty subset of RRR) constitute a
country. The country does not have any generation or consumption
apart from that which follows as the sum over the regions in the
country.  However, a number of characteristics may be  identical
for all entities (e.g. generation units,  demands,  prices and
taxes) in a country. A country is constituted of more than one
region when needed to represent bottlenecks in the electricity
transmission system within the country.

The following entities are related to countries:
%see Sections \ref{DOC-SS-Parm-CCC}, \ref{DOC-SS-Parm-F-CCC},
\begin{itemize}
  \item annuity
  \item taxes
  \item environmental policy
  \item availability of certain  fuels
\end{itemize}

The following entities are related to regions:
% see Sections \ref{DOC-SS-Parm-RRR}, \ref{DOC-SS-Parm-YYY}, \ref{DOC-SS-TRMAX-POT-R}, \ref{DOC-SS-Parm-RRE-RRI},
\begin{itemize}
\item related to electricity demand:
\begin{itemize}
  \item annual nominal  electricity demand\index{nominal}
  \item variation within the year of the  nominal electricity demand
  \item deviation from nominal  electricity demand
  \item consumer price base for electricity
   \item variation within the year of the consumer price base for electricity
\end{itemize}
\item related to electricity transmission and distribution:
\begin{itemize}
  \item losses in electrical distribution
  \item cost of electrical distribution
  \item cost of electrical transmission
  \item losses in electrical transmission
  \item electricity export to third countries
  \item variation within the year of the electricity export to third countries
  \item initial capacity on electrical transmission
  \item investment cost for new electrical transmission capacity
  \end{itemize}
  \item related to  energy and fuels:
\begin{itemize}
  \item availability of certain  fuels
 \end{itemize}
 \end{itemize}

The following entities are related to areas:
% see Sections \ref{DOC-SS-Parm-AAA}, \ref{DOC-SS-Parm-GGG-AAA},
%\ref{DOC-SS-Parm-AAA-SSS-TTT}, \ref{DOC-SS-Parm-YYY-AAA-F},
%\ref{DOC-SSS-Internal-Parm-AAA}, \ref{DOC-SSS-Internal-Parm-AAA-SSS-TTT},
\begin{itemize}
\item related to heat demand:
\begin{itemize}
 \item annual  nominal  heat consumption
 \item variation within the year of the  nominal heat demand
  \item deviation from nominal  heat demand
 \item price base for heat
 \item  variation within the year of the consumer price base for
 heat
 \end{itemize}
\item related to heat distribution:
\begin{itemize}
  \item  losses in heat distribution
  \item cost of heat distribution
\end{itemize}
\item related to technologies:
\begin{itemize}
 \item operation and maintenance cost for technologies
 \item capacity reduction factor for technologies
 \item efficiency reduction factor for technologies
 \item initial capacities of generation technologies
 \item investment cost for new technology
 \end{itemize}
 \item related to fuels:
\begin{itemize}
 \item fuel price
  \item availability of certain  fuels

  \item annual quantity and variation between the seasons  of water availability for dispatchable   hydro generation
 \item annual quantity and variation within the year of non-dispatchable hydro generation
  \item annual quantity and variation within the year of wind power generation
  \item annual quantity and variation within the year of   solar voltaic generation





   \end{itemize}
\end{itemize}

The specification in Section \ref{DOC-S-Parms-and-Scalars} is
structured according to the sets on which parameters are defined,
hence refer to the Section \ref{DOC-S-Parms-and-Scalars} part of
the  table of contents    to get an overview of the precise
dependencies on geographical entities.


\subsubsection{Countries: C, CCC}
\label{DOC-SSS-Countries} \index{country}

SET CCC contains the countries in the data structure (cf. Section
\ref{DOC-SS-Structure-Model-Simulation}), e.g.:
 \begin{itemize} \item[]
SET CCC \index{CCC}\label{CCC} /DENMARK,
%ESTONIA, FINLAND,
%GERMANY, LATVIA
%      LITHUANIA,
NORWAY,
%POLAND, RUSSIA,
SWEDEN /
 ; \end{itemize}

SET C(CCC) \index{C}\label{C} is the subset used to define those
countries that are simulated. Observe, that if C is a proper
subset of CCC then automatically the regions in the countries not
included in C are excluded from the model, see Section
\ref{DOC-SSS-Regions} (and similarly with the areas not in
regions in the countries in C, see Section \ref{DOC-SSS-Areas}).
(An obvious implication of exclusion of a region is that
electricity exchange with that region is not possible (therefore
variables VX\_T (Section \ref{VX-T}) relative to the excluded
region will not be included in the model).)



\subsubsection{Regions: RRR}
\label{DOC-SSS-Regions} \index{region}

SET RRR contains the set of regions in the data structure, e.g.:
 \begin{itemize} \item[]
 SET  RRR \index{RRR}\label{RRR}           / DK\_E, DK\_W,  NO\_R,
                                          SE\_R / ;
 \end{itemize}

%Here the first two letters identify the country while the postfix
%is used to identify the region. For this N means North, W means
%single region. A special case is the Russian Kaliningrad region,
%which is specified as RU\_K.
As the choice of names indicates, Denmark is considered to consist
of two regions, while Norway and Sweden each consists of one
region.


If a restructuring of a country is desired, so that the number of
regions is changed, this  will involve restructuring of the
associated data as well, see  Section \ref{DOC-S-Modifications}.

The simulated subset IR of RRR is described in Section \ref{IR}
along with other variants of RRR.



\subsubsection{Areas:  AAA} \label{DOC-SSS-Areas}
\index{city} \index{urban area} \index{rural area}


SET AAA contains the set of all areas in the structure, e.g.:
 \begin{itemize} \item[] SET AAA
\index{AAA}\label{AAA} \\  / DK\_E\_Copnh, DK\_E\_Other,
                              DK\_W\_Odens, DK\_W\_Arhus, DK\_W\_Other,
                            %  EE\_R\_Talln, EE\_R\_Other,
%                              FI\_R\_Urban, FI\_R\_Other,
%                              GE\_N\_Urban, GE\_N\_Other,
%                              LT\_R\_Urban, LT\_R\_Other,
%                              LV\_R\_Urban, LV\_R\_Other,
                              NO\_R\_Oslo, NO\_R\_Other,
%                              PO\_R\_Urban, PO\_R\_Other,
 %                             RU\_W\_Urban, RU\_W\_Other,
  %                            RU\_K\_Urban, RU\_K\_Other,
                              SE\_R\_Sthlm, SE\_R\_Rural / ;
\end{itemize}



If a restructuring of a region is desired, so that the number of
areas is changed, this will involve restructuring of the
associated data as well, see Section \ref{DOC-S-Modifications}.

The simulated subset of AAA is described in Section \ref{IA} and
other subsets are described in Section
\ref{DOC-SSS-AAAURBH-AAARURH}.


\subsubsection{Urban and rural heat areas: AAAURBH, AAARURH}
\label{DOC-SSS-AAAURBH-AAARURH}

The set AAA of areas is classified  according to a variety of
principles. This is done by definition of a number of subsets
(that may be overlapping) of AAA or subsets of set products that
involve AAA. In the following the classification according to heat
demand is described. Further examples are given in Section
\ref{DOC-SS-AGKN} and Section  \ref{DOC-S-Modifications}.



With respect to satisfaction of heat demand the areas are of two
kinds, urban and rural. In the urban heat areas there may be an
economic  dispatch
 \index{dispatch}\index{economic dispatch}
 (i.e., a distribution of generation among the units
varying over time according to economic principles) of heat
between generation units. This is not the case in the rural heat
areas where heat production is proportional between the different
production units. See also Table \ref{DOC-T-RuralUrbandispatchetc}
page \pageref{DOC-T-RuralUrbandispatchetc}.


The set AAARURH\index{AAARURH}\label{AAARURH} of the rural heat
areas in the data structure is defined as a subset of AAA as in
the following example:
\begin{itemize}
  \item[] SET AAARURH(AAA) / DK\_E\_Other,  DK\_W\_Other,
   SE\_R\_Rural  /;
 \end{itemize}

The set AAAURBH\index{AAAURBH}\label{AAAURBH} of the urban heat
areas in the data structure is defined similarly. Note, that the
two sets AAARURH and AAAURBH should not be overlapping.



The  subsets IARURH and IAURBH used in  simulation are then found
automatically, see Section \ref{DOC-SSS-IAURBH-IARURH}.






\subsubsection{Relations between C, R and A: RRRAAA, CCCRRR}
\label{DOC-SSS-Relations-CRA}


Given the definitions of the sets CCC, RRR, and AAA above  the
sets (mappings)\index{mapping}  RRRAAA and CCCRRR  are defined  in
order to specify the connection between the sets, i.e., RRRAAA
specifies which areas  that belong to which regions, and CCCRRR
specifies which regions that  belong to which countries. Thus,
RRRAAA\index{RRRAAA}\label{RRRAAA}(RRR,AAA) specifies the relation
between RRR and AAA and CCCRRR\label{CCCRRR}(CCC,RRR) specifies
the relation between RRR and CCC, as the following example shows:
\begin{itemize}
\item[]
SET RRRAAA(RRR,AAA) \\  / DK\_E.(DK\_E\_Copnh,DK\_E\_Other)
\\
                  DK\_W.(DK\_W\_Odens, DK\_W\_Arhus,
                  DK\_W\_Other) \\
%                  EE\_R.(EE\_R\_Urban,EE\_R\_Other)
%                  FI\_R.(FI\_R\_Urban,FI\_R\_Other)
%                  GE\_N.(GE\_N\_Urban,GE\_N\_Other)
%                  LV\_R.(LV\_R\_Urban,LV\_R\_Other)
%                  LT\_R.(LT\_R\_Urban,LT\_R\_Other)
                  NO\_R.(NO\_R\_Oslo,NO\_R\_Other) \\
%                  PO\_R.(PO\_R\_Urban,PO\_R\_Other)
%                  RU\_W.(RU\_W\_Urban,RU\_W\_Other)
%                  RU\_K.(RU\_K\_Urban,RU\_K\_Other)
                  SE\_R.(SE\_R\_Sthlm,SE\_R\_Rural) /;
\item[]
 SET CCCRRR(CCC,RRR) \\  / DENMARK  .(DK\_E,DK\_W) \\
%                  ESTONIA  .(EE\_R)
%                  FINLAND  .(FI\_R)
%                  GERMANY  .(GE\_N)
%                  LATVIA   .(LV\_R)
%                  LITHUANIA.(LT\_R)
                  NORWAY   .(NO\_R) \\
%                  POLAND   .(PO\_R)
%                  RUSSIA   .(RU\_W,RU\_K)
                  SWEDEN   .(SE\_R) /;
\end{itemize}
Observe the use of the dot\index{. (dot)}\index{dot (.)} and the
parentheses.

The internal set ICA(C,AAA) specifies the relation between AAA and
C, see Section \ref{ICA}.











\subsection{Time}
\label{DOC-SS-Time}



The description of the time dimension in the model may be divided
into two parts: that which refers to the years and the relations
between them, and that which refers to the aspects of time within
the year.

The following entities are specified (exogenous)\index{exogenous}
or found (endogenous)\index{endogenous} within a subdivision of
the year:
\begin{itemize}
  \item  generation (exogenous and endogenous)
  \item  relative weight of time segment
  \item  capacity derating of generation units
  \item  availability of hydro
  \item  demands for electricity and heat
  \item  calibration parameters relative to demands for electricity and heat
  \item  flexible demands related parameters
  \item  electricity exchange with third regions
%  \item
\end{itemize}

The following entities are the same throughout each year, but may
be different from one year to the next one:
\begin{itemize}
 \item nominal generation capacities
 \item  fuel prices
  \item emission limitations and taxes
%  \item
\end{itemize}

The following entities are the same for all years in the data
structure:
\begin{itemize}
  \item characteristics of generation technologies (except that
     some may be available only  from a certain year, and except for capacity derating)
  \item  annuity
  \item distribution and transmission characteristics
  \item costs (except fuel costs)
  \item fuel potentials, including water availability
  \item taxes (except those related to emissions)
  \item variations within the year
  \item demand elasticities
  \item fuel characteristics (except prices)

\end{itemize}

The specification in Section \ref{DOC-S-Parms-and-Scalars} is
structured according to the sets on which parameters are defined,
hence refer to the Section \ref{DOC-S-Parms-and-Scalars} part of
the  table on contents    to get on overview of the precise
dependencies on geographical entities.



\subsubsection{The  years: YYY, Y}
\label{DOC-SS-Timebetweenyears} \index{year} \index{time}
 The years represented in the data
structures  are given by the SET YYY, e.g.:
 \begin{itemize}
 \item[] SET YYY \index{YYY}\label{YYY} / 1995 * 2030   / ;
 \end{itemize}
\label{YEARS} where the asterisk notation using "*" implies that
the years from  1995\index{1995,...,2030}\label{1995,...,2030} to
2030 are included.


The subset of years simulated is given by\index{Y}\label{Y} the
SET Y(YYY).

\naming\index{naming} The only labels consisting of digits only
are those used for set elements YYY (and therefore also those in
the subset Y(YYY)).

See also YEARINC in Section \ref{YEARINC}.

%Some but not all of the input data are specified individually
%with respect to the year, cf. Section
%\ref{DOC-S-Parms-and-Scalars}. In general terms we could say that
%the   parameters that depend on the year Y are those related to
%demands of electricity, demands of heat, fixed electricity
%exchanges with third regions, prices of fuels, development of
%those capacities that are  exogenously given, introduction year
%for emerging new technologies. [More?]

The sets YYY and Y are ordered, cf. the comments  in Section
\ref{DOC-SSS-Orderedsets} on ordered and unordered sets.


\subsubsection{Time segments within years: SSS, S, TTT, T}
\label{DOC-SSS-Timewithinyears} \index{within the year}
 The subdivision of the year into
seasons is given by SET SSS specified e.g. as the following:
  \begin{itemize} \item[] SET  SSS\index{SSS}\label{SSS}
/  S1 * S4 / ;
 \end{itemize}
and the subdivision of the time within the season\index{within
the season}\index{day} of a season is given by SET TTT, e.g.,
 \begin{itemize} \item[] SET  TTT \index{TTT}\label{TTT} /  T1 * T8 /
  \end{itemize}

These examples mean that the year is divided into four
 seasons,\index{season} and that each season has been subdivided
 into   eight time segments\index{time period} (sub periods).
\index{sub period} We refer to the part of the year specified by
(S,T) as a time segment\index{time segment}\index{segment} (or
more specifically as a time segment of the year) and to the part
of the season specified by T as a time segment of the season.

(It is  tempting to  say that the set TTT represents a subdivision
of the day - and we may actually do so sometimes. However,  is
not in general  correct to say so,  see Section \ref{WEIGHT-T}.)


The extension - weight, duration - of each time segment in  S and
T is held in the parameters WEIGHT\_S  and WEIGHT\_T,
respectively, cf. Sections \ref{WEIGHT-S} and \ref{WEIGHT-T}.


The seasons and time periods used in simulation are specified by
the sets \index{S}\label{S} S(SSS) and   \index{T}\label{T}
T(TTT), respectively.    S and T should be ordered, cf. Section
\ref{DOC-SSS-Orderedsets}.


Observe that all the descriptions of the subdivision of the year
are the same for all the geographical entities (countries,
regions, and areas, i.e., the sets CCC, RRR and AAA) and for all
the years (the set YYY) in the model.




\inputdata It will be assumed that the year has 365\index{365} days and 8760\index{8760}
hours.

See also Section  \ref{IDAYSIN-S}  and Section
 \ref{IHOURSIN24}.



\naming\index{naming} For chronological\index{chronology}
specifications of time segments the naming of the individual
seasons will start from winter (i.e. the first season will
include January 1st), and the naming of the time periods of the
day will start at midnight, see also  Sections \ref{WEIGHT-S} and
\ref{WEIGHT-T}. The labels should therefore be entered in such
sequence, in particular in relation to application of ordered
sets, cf. Section \ref{DOC-SSS-Orderedsets}.

Obviously there are some interdependencies between the subdivision
of the year into seasons and  the further subdivision of the
seasons, and this could expediently be reflected in the naming.
The following convention may be used for naming the seasons: SET
SSS  may be  defined as e.g.
\begin{itemize}
  \item[]  SET SSS  / S1\_1  /; or\\
  SET SSS / S2\_1, S2\_2  /; or\\
   SET  SSS        /   S4\_1  *  S4\_4 /; or \\
   SET   SSS      /   S12\_1  *  S12\_12  /;
\end{itemize}
This gives  the possibilities of representing the year with 1, 2,
4, or 12   seasons, respectively.


Similarly, SET TTT may be defined as
\begin{itemize}
  \item[]  SET TTT  / S1T1\_1 /; or \\
 SET TTT /  S2T2\_1, S2T2\_ 2/; or \\
SET TTT /   S4T8\_1 * S4T8\_8 /;
\end{itemize}
giving the possibilities to represent the subdivision of the
season (the "day")  with 1, 2 or 8 segments, respectively.



It is easy to aggregate time within the year such that the model
uses only annual data, i.e., there is no subdivision into seasons
nor any subdivision of the season into sub-periods. This is
achieved by  specifying  the sets S and T to contain only one
member each, e.g.: "SET S(SSS) /S1/ " and "SET T(TTT)  /T1/". It
is also easy to use other subsets S and T. E.g., if SSS is defined
as "SET /S12\_1 * S12\_12/" to represent the twelve months of the
year, then specifying SET S(SSS) / S12\_1, S12\_7 / means that
only January and July will be used in the simulations to represent
the whole year.

Further refinements are possible via the set IST, Section
\ref{DOC-SSS-IST}.



\inputdata Demand for electricity is specified for each region. If demand is not
synchronous between the regions, this will motivate exchange
between the regions. In particular, this may be relevant for
regions  far apart in the east - west direction, because in this
case there may be a time zone difference. This is more outspoken
the larger the difference in time zone is in relation to the
length of the time segments in TTT. Table \ref{DOC-T-timezones}
 illustrates for the
Baltic Sea Region the local time  zones relative to
GMT.\index{GMT, Greenwich Mean Time}\index{time zones}

\begin{table}
\center{
\begin{tabular}{|l |c| c |c |c| c | c |}
\hline

&       GMT    &    DK&    EE &   FI &   DE  &  LV \\
\hline
Summer &  0    &     2  &     2 &      3   &    2  &     2 \\
Winter &0 & 1&1&2&1&1 \\
\hline \hline
 &          LT &   NO &   PL &   RU (West)  & RU (Kaliningrad) &SE \\
 \hline
Summer  &   2    &   2   &    2 &      4  &     3 &      2 \\
Winter &1 &      1 &      1  &     3  &     2  &     1 \\
\hline
\end{tabular} }
\caption{Time zones in  the Baltic Sea Region relative to GMT
(based on http://time.greenwich2000.com/). Observe that these
conventions seem to be quite unstable.}
 \label{DOC-T-timezones}
\end{table}






\subsubsection{Ordered  and unordered sets}
 \label{DOC-SSS-Orderedsets}

 Sets in GAMS may be
  ordered\index{ordered set}\index{unordered set}\index{set, (un)ordered}
   or unordered. Ordered sets are
static\index{static set}\index{set, static} in the sense that they
are initialised by having their elements specified between "/"
and "/" at the time of declaration, and the sets are never
changed afterwards.
 %GAMS bog side 133
They are ordered in the sense that the order in which the labels
appear in the GAMS program is the same as the order in which they
appear in the initialisation of the set (the entry
 order\index{entry order}). See also Section
 \ref{DOC-SSS-staticanddynamicsets}.

For ordered sets, the elements have a sequence, viz., that given
in the initialisation.  Hence, for such sets it is possible to
know if one element is "before" or "after" another one in the set,
implying in relation to  modelling that
chronological\index{chronology} phenomena may be represented. The
operators '+' and '++' \index{+, ++, next} are used to indicate
"the next" element, the latter further indicates a cyclical
concept where "the first" is the successor to "the last".

The function ORD\index{ORD} applied to an element in a
one-dimensional static and  ordered set returns the number of that
element in the sequence.
 % GAMS side 134
  Thus for SET SEASONS /winter, spring, summer, autumn/,
ORD("summer") attains the value 3. The function CARD\index{CARD}
returns the number of elements in a set (also for an unordered
set), hence e.g. CARD(SEASONS) attains the value 4.

It is essential that the sets YYY, Y, SSS, S, TTT and T are
ordered. For set S this may for instance be used for modelling of
hydro power with reservoirs, where it is desired to represent
that the contents of the reservoir at the beginning of a season
equals the contents at the beginning of the previous season plus
the inflow during the previous season, minus the water used for
generation during the previous season. For  set T this may
similarly be used for modeling hydro power with
reservoirs\index{storage}\index{reservoir} for shorter operation
cycles (e.g. pumped storage suited for leveling of variations
within the day or the week), or similarly for short-term heat
storage. (See also Section \ref{WEIGHT-T}.)





\subsection{Generation technologies: GGG, G, GDATASET}
\label{DOC-SS-Technologies} \index{technology}
 \index{generation technology}

SET GGG\index{GGG}\label{GGG} is the set of generation
technologies (i.e., hardware for transformation of energy) in the
structure, given as e.g.
 \begin{itemize} \item[] SET    GGG \\
 /CC-Cond1, ST-Cond1-G,  ST-Cond1-O,  CC-Co-B95,HO-Pump, HO-W-Old, HO-CHP-G,
  HYDRO, GWIND1, GWIND2
 /; \end{itemize}



SET\index{G}\label{G} G(GGG) is the set of generation technologies
simulated, e.g.
\begin{itemize}
  \item[] SET G(GGG) / ST-Cond1-O,     HO-CHP-G,    GWIND2 /;
\end{itemize}

Subsets of G are described in Section
\ref{DOC-SSS-Generation-technology-types}.



The set    GDATASET\index{GDATASET}\label{GDATASET}  is the set of
attributes of generation  technologies:

\begin{itemize} \item[] SET GDATASET / GDTYPE, GDFUEL, GDCB, GDCV, GDFE, GDESO2, GDNOX, GDCH4, GDAUXIL,
GDINVCOST0, GDOMVCOST0, GDOMFCOST0, GDFROMYEAR, GDLIFETIME,
GDKVARIABL, GDCOMB  / ;
 \end{itemize}
\index{GDTYPE}
\index{GDFUEL}\index{GDCB}\index{GDCV}\index{GDFE}\index{GDESO2}
\index{GDNOX}\index{GDCH4}\index{GDAUXIL}
\index{GDINVCOST0}\index{GDOMVCOST0}\index{GDOMFCOST0}
\index{GDFROMYEAR}\index{GDKVARIABL}\index{GDLIFETIME}\index{GDCOMB}
Descriptions are given in Section
\ref{DOC-SSS-Generation-technology-types} and in Section
\ref{GDATA}.



Observe that the user should  not change this set without proper
knowledge of the functioning of the set. Thus, the set can not be
reduced from that specified  above since data will be needed in
the model for each of the elements, see Section \ref{GDATA}. The
set may be enlarged with new elements, however then the user will
have to specify in the model how these elements are to be used.

The data corresponding to the  elements GDINVCOST0, GDOMVCOST0 and
GDOMFCOST0 are considered as default values that may be
overwritten, see Sections \ref{GDATA}, \ref{GDINVCOST},
\ref{GDOMVCOST} and \ref{GDOMFCOST}.

The data are further discussed in Section \ref{GDATA}.

\naming\index{naming} Observe that to distinguish technology from
fuel (see Section \ref{DOC-SS-Fuels}) where similar labels (names)
are tempting, the following is advocated: hydro power is called
something indicating "hydro"\index{hydro power} as a technology,
i.e., as an element in GGG, and something indicating
"water"\index{water} as a fuel, i.e., as an element in FFF (see
Section \ref{DOC-SS-Fuels}). \label{DOC-namingHYDROWATER} For all
other ambiguous subjects, a prefixed "G" is advocated for elements
in GGG and no prefixed "G" for elements in FFF. E.g., a particular
wind turbine  could be e.g. "GWIND-2300" as an element in G but
not as an element in FFF.

In the Balmorel the specification of a generation unit is done by
referring to its name (according to the technology catalogue given
by set GGG) and its geographical location (according to the area
catalogue given by set AAA). Thus, a specific kind of technology
may be represented in more than one area. The capacity GKFX
(Section \ref{GKFX}) of a particular generation unit must
therefore be specified with  indexes reflection this, i.e.
GKFX(*,AAA,G), where AAA represents geography and G represents
technology kind.

The idea behind this is that for the geographical area considered
(the Baltic Sea Region) it is not possible to get, nor sensible to
use, precise information about all generation units. Therefore a
limited number (approximately 50) of technology kinds have
initially been specified in the set GGG. This moreover facilitates
that aggregation of existing units into fewer but larger ones.

The expectation is, however, that with increased application of
the model, some possibly with a national focus, more data will
become available, and in specific applications there will be a
desire to increase the level of detail in representation of
technologies.

The GAMS syntax (Section \ref{DOC-SS-GAMS-intro}) permits an
explanatory text associated with (some or all) units in G. This
identification is a possibility only, and therefore to use it
systematically a convention is needed, and one such will now be
described.

As illustrated in Table \ref{DOC-Table-Technologiesidentification}
the text associated with the label in the set GGG is used to
describe the technology, and this description indicates how
specific each technology is. The GAMS syntax permits up to 80
characters,  all on the same line, preferable enclosed in (single
or double) quotes. To provide nice single line printouts, at
maximum of 50 characters is advocated.

Observe, though, that in the mechanisms of the  Balmorel model
this text is not used. Therefore the user must make sure that the
data entered for each technology is consistent with the intention
indicated in the text for that technology.

Thus, for a technology which is intended to be located in one area
only, the user must make sure that this technology appears with
positive capacity GKFX(*,AAA,G) only in that area. If new
investments are permitted the user must similarly make sure that
the set AGKN(AAA,G) (Section \ref{AGKN}) specifies that new
capacity of the technology in question can only be established in
the relevant area.


{\small
\begin{center}
\begin{table}
\begin{tabular}{ll}

SET GGG & 'All generation technologies' \\
/ & \\
  ST-Cx-CO   & "Generic steam condensing coal - new" \\
  ST-Cy-CO   & "Generic steam condensing coal - old" \\
  ST-G-NOS    & "The proposed new natural gas - Norway South" \\
  NU-BBackSE   & "Swedish nuclear Barseback" \\
  ST-B8-CO   &  "Generic CHP back pressure coal - old"  \\
  WI-L9      & "Generic wind power - new" \\
  WI-RS-DK & "New off shore wind - Roedsand, Denmark" \\
/;
\end{tabular}
 \caption{Illustration of a convention permitting various levels of identification of generation units.}
 \label{DOC-Table-Technologiesidentification}
\end{table}
\end{center}
 }







\subsection{Fuels: FFF, FDATASET, FKPOTSET}
\label{DOC-SS-Fuels} \index{fuel}

SET FFF\index{FFF}\label{FFF} is the set of   fuels in the
structure, given as e.g.
 \begin{itemize} \item[] SET FFF
       /NUCLEAR, NGAS, COAL-HIGHS, COAL-LOWS, LIGNITE, FUELOIL,
      SHALE, PEAT,  WIND1800,  WIND2300,
     WATER,  BIO,  SUN, ELEC, GARBAGE  /;
 \end{itemize}

Observe that in all simulations the whole set FFF is used. If
therefore a particular fuel is not desired, the technology that
uses it could be excluded from GGG and G, or from G.





The set of fuels is divided into three  subsets by definitions of
SET FKPOTSETC(FFF), set FKPOTSETR(FFF) and  SET FKPOTSETA(FFF).
\index{FKPOTSETC}\index{FKPOTSETR}\index{FKPOTSETA}
\label{FKPOTSETC}\label{FKPOTSETR}\label{FKPOTSETA} The subsets
need not be  mutually exclusive, nor need  they  together
constitute FFF.


The subsets indicate whether the members have their potentials
specified at the level of country, region or area, respectively.


The following could the examples of definitions:
 \begin{itemize} \item[] SET FKPOTSETC(FFF)
       / NUCLEAR,  LIGNITE,  SHALE, PEAT   /;  \end{itemize}
 \begin{itemize} \item[] SET FKPOTSETR(FFF)
       / WIND, WATER, SUN, BIO  /;  \end{itemize}
\begin{itemize} \item[] SET FKPOTSETA(FFF)
       / NGAS, WASTE  /;  \end{itemize}

Thus, if e.g. COAL and FUELOIL are included in the set FFF, no
limit will be placed on the use of these fuels.



SET FDATASET \index{FDATASET}\label{FDATASET} is the set of
attributes of fuels:
\begin{itemize} \item[] SET   FDATASET        /
  FDNB,  FDCO2,  FDSO2 , FDN2O    \index{FDCO2}\label{FDCO2}
        \index{FDSO2}\label{FDSO2}\index{FDNB}\label{FDNB}\index{FDN2O}\label{FDN2O}
 / ;  \end{itemize}

%(Here: description)

The FDNB contributes to the   coupling between generation
technology and fuel. In GDATASET (Section
\ref{DOC-SS-Technologies}) the elements GDFUEL for each technology
contains an integer that points to the FDNB for  the   fuel that
the technology uses, cf. also Sections \ref{FDATA} and
\ref{GDATA}.



Observe that the user should   not  change the  set  FDATASET
without proper knowledge of the functioning of the set. Thus, the
set can not be reduced from that specified  above since data will
be needed in the model for each of the elements, see Section
\ref{FDATA}. The set may be enlarged with new elements, however
then the user will have to specify in the model how these
elements are to be used.


\naming\index{naming} See page \pageref{DOC-namingHYDROWATER}.






\subsection{New generation technology and area: AGKN}
\label{DOC-SS-AGKN}\index{investment}\index{new generation tech.}


Investment in new generation capacity is determined endogenously
in the model. The specification of where   new technology capacity
of a particular type  can be placed  must therefore be determined.
This is done by specifying the product set
AGKN(AAA,G)\index{AGKN}\label{AGKN} that hold those combinations
of areas and technologies that permit new investment.


A default definition of AGKN is the following. All technologies
that can produce heat (except that with GDAUXIL=3) can be placed
in urban areas; all technologies that can produce heat (except
that with GDAUXIL=1) can be placed in rural areas; all
technologies that produce only electricity heat can for every
region be placed in one and only one of the  areas in that region.

See Section \ref{DOC-SSS-Generation-technology-types} concerning
specification of capacity according to heat or electricity side.




Other possibilities are described in Section
\ref{DOC-S-Modifications}.




\subsection{Demand: DF\_QP, DEF\_..., DHF\_...}
 \label{DOC-SS-Demand} \index{demand}

Demand for electricity is  specified for each region and demand
for heat is specified for each area.

The specification may be considered to consist of three elements,
see also Figure \ref{DOC-F-elasticdemands1eps}  and Figure
\ref{DOC-F-elasticdemands2eps}:
\begin{itemize}
  \item A nominal\index{nominal} value, specified for each year in the simulation period
as an annual quantity, parameters DE and DH, Sections \ref{DE} and
\ref{DH}.
 \item A nominal profile, i.e., a distribution of the annual quantity
  over the time segments of the year, specified in DE\_VAR\_T and
  DH\_VAR\_T, see Sections \ref{DE-VAR-T} and \ref{DH-VAR-T}.
  \item An elasticity\index{elastic demand} function which specifies the relationship
between quantity and price for deviations from the nominal
profile. Parameter  values for this are given as described  in
Sections  \ref{DEFP-BASE},  \ref{DHFP-BASE}, \ref{DHFP-CALIB},
\ref{DEFP-CALIB}, \ref{DHFP-T}, \ref{IDEFP-T},  while the related
sets are specified in the following.
\end{itemize}








\begin{figure}
\includegraphics[clip,width=0.9\textwidth]{elasticdemands1eps.eps}
 \caption{Elastic demand, illustration of price elasticities.}
\label{DOC-F-elasticdemands1eps}
\end{figure}



\begin{figure}
\includegraphics[clip,width=0.9\textwidth]{elasticdemands2eps.eps}
 \caption{Elastic demand, illustration of development over time.}
\label{DOC-F-elasticdemands2eps}
\end{figure}









The sets related to elastic demands  specify  steps relating
quantities and prices:
\begin{itemize}
  \item[] SET DF\_QP /DF\_QUANT, DF\_PRICE   /;
 \index{DF\_QP} \index{DF\_PRICE}\label{DF-PRICE}
 \label{DF-QP} \index{DF\_QUANT}\label{DF-QUANT}
\end{itemize}

Observe that the user should  not  change this set without proper
knowledge of the functioning of the set. Thus, the set can not be
reduced from that specified  above since data will be needed in
the model for each of the elements, see below.


The  individual steps in the electricity demand function are
specified by SET DEF given as e.g.:
\begin{itemize}
  \item[]SET DEF / DEF\_DINF, DEF\_D3, DEF\_D2, DEF\_D1,
  DEF\_U1, DEF\_U2, DEF\_UINF / ; \label{DEF}\index{DEF}
\end{itemize}
This example shows 7 steps.

The entry order (Section \ref{DOC-SSS-Orderedsets}) is important
in relation to these sets.

Observe that the elements  DEF\_DINF and DEF\_UINF must  be
included in the set DEF.
\index{DEF\_DINF}\index{DEF\_UINF}\label{DEF-DINF}\label{DEF-UINF}

SET DEF\_D(DEF)  and SET DEF\_U(DEF) are subsets  used to
distinguish between steps for regulation downwards (decreased
demand, in this example 4 steps) and upwards (increased demands,
in this example 3 steps) of electricity demand relative to
nominal\index{nominal} demand:
\begin{itemize}
  \item[] SET DEF\_D(DEF)   / DEF\_DINF, DEF\_D3, DEF\_D2,
  DEF\_D1/ ; \index{DEF\_D}\label{DEF-D}
\end{itemize}
\begin{itemize}
  \item[] SET DEF\_U(DEF) \index{DEF\_U}\label{DEF-U}
 /  DEF\_U1, DEF\_U2, DEF\_UINF/ ;
\end{itemize}

Observe that DEFP\_D must contain  DEF\_DINF and that DEFP\_U
must  contain DEF\_UINF.

 If the above two sets are specified as
\begin{itemize}
  \item[]    DEF\_D(DEF)  / DEF\_DINF / ;
\end{itemize}
\begin{itemize}
  \item[]     DEF\_U(DEF)  / DEF\_UINF / ;
\end{itemize}
then the intention is that demand is   inelastic,\index{inelastic
demand } according to the interpretation relative to  DEF\_STEPS,
see Section \ref{DEF-STEPS}.

Similarly for the steps in the heat demand function (the example
has  5 steps, of which 3 are down and 2 are up):

\begin{itemize}
  \item[]   SET DHF    / DHF\_DINF, DEF\_D2, DEF\_D1, DEF\_U1,  DHF\_UINF
  /;
  \index{DHF}\label{DHF}
\item[] SET  DHF\_D(DHF)  / DHF\_DINF, DEF\_D2, DEF\_D1/ ;  \index{DHF\_D}\label{DHF-D}
\item[] SET  DHF\_U(DHF)  / DEF\_U1, DHF\_UINF / ;  \index{DHF\_U}\label{DHF-U}
\end{itemize}
Similar comments apply to DHF\_DINF,
DHF\_UINF\index{DHF\_DINF}\index{DHF\_UINF}, DHF\_D and DHF\_U as
to DEF\_DINF, DEF\_UINF,   DEF\_D and DEF\_U above.
\label{DHF-DINF}\label{DHF-UINF}

\subsection{Emission policies: MPOLSET}
 \label{DOC-SS-Emission-Pol}
\index{emission policy}


SET MPOLSET\index{MPOLSET}\label{MPOLSET} contains elements for
specification of environmental policies,
\begin{itemize}
  \item[] TAX\_CO2   "CO2 emission tax (\MONEY/t CO2)"\index{TAX\_CO2}\label{TAX-CO2}
  \item[] TAX\_SO2   "SO2 emission tax (\MONEY/t SO2)"\index{TAX\_SO2}\label{TAX-SO2}
  \item[] TAX\_NOx   "NOx emission tax (\MONEY/kg NOx)"\index{TAX\_NOx}\label{TAX-NOX}
  \item[] LIM\_CO2   "Annual CO2 limit (t CO2/year)"\index{LIM\_CO2}\label{LIM-CO2}
  \item[] LIM\_SO2   "Annual SO2 limit (t SO2/year)"\index{LIM\_SO2}\label{LIM-SO2}
  \item[] LIM\_NOx   "Annual NOx limit(kg NOx/year)"\index{LIM\_NOx}\label{LIM-NOX}
\end{itemize}



Observe that the user should not change this set without proper
knowledge of the functioning of the set. Thus, the set can not be
reduced from that specified  above since data will be needed in
the model for each of the elements, see Section \ref{M-POL}. The
set may be enlarged with new elements, however then the user will
have to specify in the model how these elements are to be used.







\subsection{Internal sets} \label{DOC-SS-Internal-Parm-Sets}
\index{internal set} \index{SET}

A number of sets are defined and their members  defined
automatically, i.e.,  they are not specified explicitly by the
user. We refer to these as internal sets, in contrast to input
 sets\index{input set}. The names of these sets start with I,
 Section \ref{DOC-SS-Naming-conventions}.
The internal sets are dynamic sets in the sense explained next.

\subsubsection{Static and dynamic sets}
\label{DOC-SSS-staticanddynamicsets}

In the GAMS terminology,
 static sets\index{set, static}\index{static set}
  are sets that have their membership
declared as the SET itself was declared and the membership was
never changed.
  % GAMS-bog p 127
 In contrast,
  dynamic sets\index{set, dynamic}\index{dynamic set}
have their membership changed because  of assignments. Hence,
membership of dynamic sets may change during the execution of the
program.

In assignments,\index{assignment, sets}  constructions of sets may
be done using the symbols
 "+",\index{+, set union}
  "-",\index{-, set difference}
   "*"\index{*, set intersection}\index{set intersection}
and "NOT"\index{NOT, set complement}  to provide the set
operations\index{set operations}
 union\index{union (set)},
  difference\index{difference (set)},
 intersection\index{intersection (set)} and
 complement\index{complement (set)},
 respectively. Constructions
using YES and NO\index{YES}\index{NO} may be used, see below for
examples.

Dynamic sets are not ordered, Section \ref{DOC-SSS-Orderedsets}.
They can not be used in declarations but can be used in
definitions, see page
 \pageref{DOC-Setdeclarationdefinition}.






\subsubsection{Areas simulated: IA}

SET  IA(AAA) \index{IA}\label{IA} is the subset  used to define
those areas that are simulated. This subset is derived
automatically as that subset of AAA that is relevant for the
simulated countries C:
\begin{itemize}
  \item[] SET IA(AAA) = YES\$(SUM(C,ICA(AAA,C)));
 \end{itemize}



\subsubsection{Rural and urban heat areas simulated: IARURH, IAURBH}
\label{DOC-SSS-IAURBH-IARURH}

 The  subset
IARURH\index{IARURH}\label{IARURH}  used in simulation is found
automatically as that subset of AAARURH, Section \ref{IARURH},
that is relevant for the simulated countries C:
\begin{itemize}
  \item[] SET IARURH(AAARURH) = YES\$(SUM(C,ICA(AAARURH,C)));
 \end{itemize}

The subset IAURBH\index{IAURBH} \label{IAURBH} of urban heat areas
is similarly automatically defined    as the set of the urban heat
areas.





\subsubsection{Country to area mapping: ICA}
\label{DOC-SSS-ICA}


The internal set ICA(C,AAA)\index{ICA}\label{ICA} specifies the
relation between AAA and C. It is  derived automatically from the
sets RRRAAA(RRR,AAA) and CCCRRR(C,RRR), to assign consistently the
areas in AAA to the countries in C:
 \begin{itemize}
\item[] ICA(C,AAA)=YES\$(SUM(RRR, (RRRAAA(RRR,AAA) AND CCCRRR(C,RRR))));
 \end{itemize}



\subsubsection{Regions simulated: IR}


SET IR(RRR) \index{IR}\label{IR} is the subset  of regions that
are simulated.     This subset is derived automatically for the
simulated countries C as:
\begin{itemize}
  \item[] SET IR(RRR) =     YES\$(SUM(C,CCCRRR(C,RRR)));
\end{itemize}


\subsubsection{Electricity import-export: IRRRI, IRRRE, IRI, IRE}
 \index{IRRRE}\index{IRRRI}\index{IRE}\index{IRI}
 \label{IRRRE}\label{IRRRI}\label{IRE}\label{IRI}

 For description of transmission relations between pairs
of regions copies of the  sets are necessary. They are obtained
by the ALIAS statement as:
 \begin{itemize} \item[]
\index{ALIAS}ALIAS(RRR,IRRRE), ALIAS(RRR,IRRRI), ALIAS(IR,IRE),
ALIAS(IR,IRI)
 \end{itemize}

This permits the reference to pairs of regions, e.g., (IRI,IRE).
As seen, IRRRE and IRRRE are the sets of regions in the data
structure, and IRE and IRI are the subsets of regions in the
simulation. The final E and I are used to indicate exporting and
importing regions, respectively.


\subsubsection{Season and time duplication: ISALIAS, ITALIAS}
\label{DOC-SSS-ISALIAS-ITALIAS}

Copies of the sets S and T are obtained as "ALIAS(S,ISALIAS)" and
"ALIAS(T,ITALIAS)".\index{ALIAS}\index{ISALIAS}\index{ITALIAS}
\label{ISALIAS}\label{ITALIAS}

\subsubsection{Season and time refinements: IST, ISTS, ISTT}
 \label{DOC-SSS-IST}\index{IST}\label{IST}
The set IST(S,T) identifies the time segments selected. In the
standard version IST(S,T) holds all combinations of elements in S
and T, i.e., IST is defined as IST(S,T)=YES, but a subset of this
may be defined. Combined with overwriting of the standard
definition of IHOURSINST(S,T) (Section \ref{IHOURSINST}) this
permits maximal flexibility with respect to the time structure.

The set ISTS(S)\index{ISTS}\label{ISTS} holds the time segments S
for which there is at least one element T.

The set ISTT(T)\index{ISTT}\label{ISTT} holds the time segments T
appearing in at least one seasons S.



\subsubsection{Generation technology types}
\label{DOC-SSS-Generation-technology-types}

A number of convenient subsets of generation technology types have
been defined. There are  11 basic types:
 Condensing\index{condensing unit},
 Back pressure
 \index{back pressure unit} (sometimes also called intermediate take out and
condensing),
  Extraction  \index{extraction unit},
 Heat-only\index{heat only} boilers,
 Electric heaters/heatpumps\index{heatpump}
 \index{electric heating},
   Heat storage technologies,
   \index{heat storage}\index{storage}
   Electricity storage technologies,
   \index{electricity storage}
  Hydro power with
storage\index{storage}\index{reservoir},
 Hydro power without storage
 \index{hydro power}
(run-of-river\index{run-of-river}\index{hydro run-of-river}),
 Wind power,
 \index{wind power}
 Solar power   \index{solar power}.


The names of these sets are, respectively,
  \index{IGCND}\label{IGCND}IGCND(G),
  \index{IGBPR}\label{IGBPR}IGBPR(G),
  \index{IGEXT}\label{IGEXT}IGEXT(G),
  \index{IGHOB}\label{IGHOB}IGHOB(G),
  \index{IGETOH}\label{IGETOH}IGETOH(G),
  \index{IGHSTO}\label{IGHSTO}IGHSTO(G),
  \index{IGESTO}\label{IGESTO}IGESTO(G),
  \index{IGHYRS}\label{IGHYRS}IGHYRS(G),
  \index{IGHYRR}\label{IGHYRR}IGHYRR(G),
  \index{IGWND}\label{IGWND}IGWND(G),
  \index{IGSOL}\label{IGSOL} IGSOL(G).
Each technology is automatically allocated to one and only one of
the sets according to the integer given in GDATA(GGG,'GDTYPE') as
follows:
\begin{itemize}
  \item[]     IGCND(G)    = YES\$(GDATA(G,'GDTYPE') EQ 1);
  \item[]     IGBPR(G)    = YES\$(GDATA(G,'GDTYPE') EQ 2);
  \item[]     IGEXT(G)    = YES\$(GDATA(G,'GDTYPE') EQ 3);
  \item[]     IGHOB(G)    = YES\$(GDATA(G,'GDTYPE') EQ 4);
  \item[]     IGETOH(G)   = YES\$(GDATA(G,'GDTYPE') EQ 5);
  \item[]     IGHSTO(G)   = YES\$(GDATA(G,'GDTYPE') EQ 6);
  \item[]     IGESTO(G)   = YES\$(GDATA(G,'GDTYPE') EQ 7);
  \item[]     IGHYRS(G)   = YES\$(GDATA(G,'GDTYPE') EQ 8);
  \item[]     IGHYRR(G)   = YES\$(GDATA(G,'GDTYPE') EQ 9);
  \item[]     IGWND(G)    = YES\$(GDATA(G,'GDTYPE') EQ 10);
  \item[]     IGSOL(G)    = YES\$(GDATA(G,'GDTYPE') EQ 11);
\end{itemize}



Observe that the  sets are defined as  subsets of G, not GGG.

The technologies in each set  has specific properties in the
model.  Since the allocation of the technologies  to the sets is
done according to the value of the integer in GDATA(GGG,'GDTYPE'),
the user should  not  change the meaning associated with these
integers without proper understanding of the model.  These
properties are expressed as equations and/or  lower and/or upper
bounds  and/or fixed values on the individual variables describing
the generation from the technology. See BALMOREL.GMS for details.




In addition, the technologies may be grouped into sets according
to various characteristics. Some examples are: all technologies
excluding electric heating, technologies for
 which the generation may be dispatched,\index{dispatch}  technologies that produce
electricity only i.e., not  heat.

The identification  of these technology sets are:
\begin{itemize}
 \item[]   IGHH(G):\index{IGHH}\index{IGHH}  producing only heat
 \item[]     IGEE(G):\index{IGEE}\index{IGEE}  producing only  electricity
 \item[]     IGHHNOSTO(G):\index{IGHHNOSTO}\index{IGHHNOSTO}
 type IGHH except heat storage
 \item[]     IGEENOSTO(G):\index{IGEENOSTO}\index{IGEENOSTO}  type IGEE except electricity.
storage
 \item[]           \index{IGKH}\label{IGKH}IGKH(G):
 capacity specified with respect to heat \index{capacity}
 \item[]           \index{IGKE}\label{IGKE}IGKE(G):   capacity specified with respect to
electricity
 \item[]     IGKENOSTO(G):\index{IGKENOSTO}\index{IGKENOSTO}
 capacity given on electricity side, except el. storage
 \item[]        IGKHNOSTO(G):\index{IGKHNOSTO}\index{IGKHNOSTO}
 capacity given on heat side, except heat storage
 \item[]           \index{IGNOTETOH}\label{IGNOTETOH}IGNOTETOH(G): all
 except heat pumps
 \item[]           \index{IGDISPATCH}\label{IGDISPATCH}IGDISPATCH(G):
   dispatch may be made
 \item[]          \index{IGEOREH}\label{IGEOREH}IGEOREH(G):
  producing electricity (with or without heat)
 \item[]          \index{IGKKNOWN}\label{IGKKNOWN}IGKKNOWN(G):
 capacity can not be found endogeneously
 \item[]         \index{IGKFIND}\label{IGKFIND}IGKFIND(G):   capacity can be found endogeneously
% \item[]  \index{IGURBANONLY}\label{IGURBANONLY}IGURBANONLY(G):  % ud?
\item[]     \index{IGHORHERUR}\label{IGHORHERUR}IGHORHERUR(G): producing heat (with or without
electricity), rural areas
\end{itemize}

Each technology is automatically allocated to these dynamic sets.
For some of them, this is done according to the integer given in
GDATA(GGG,'GDTYPE'), and/or using previously defined sets. Table
\ref{DOC-T-dynamic-G-sets} specifies the  dynamic sets that are
defined  this way.  The sets IGKKNOWN and IGKFIND are defined
according to the values given in GDATA(G,'GDKVARIABL'). Other ways
may be used. See BALMOREL.GMS for details.

Table \ref{DOC-T-RuralUrbandispatchetc} indicates characteristics
of  urban and rural areas with respect to heat  dispatch.



\begin{table}
\center{
{\small
\begin{tabular}{ |l|l|l|  } \hline
 &IARURH    & IAURBH    \\
\hline
  Heat dispatch &  Fixed  &  Free    \\

                &          & (in combination with el. dispatch)   \\
 \hline
  El. dispatch  &  Fixed &Free \\

%                &except hydro with reservoir
  &                &(in combination with heat dispatch)\\
%\hline
% El.-only units & Yes  &  Existing, exog.: Yes,   \\
%  capacity      &      & New,   exog. capacity: Yes \\
%                &      & New,  endog.  capacity: No \\
%\hline
%  Heat-only units & Existing, exog.:  Yes  & Yes  \\
%capacity          &  New,   exog. capacity: Yes  & \\
%                  & New,  endog.  capacity: No  &\\
%\hline
%GDAUXIL=1   & No & Yes \\
%GDAUXIL=2   & Yes & Yes \\
%GDAUXIL=3   & Yes & No \\
\hline
\end{tabular}
} % small
} %center
\caption{Differences between urban and rural areas with respect
to  treatment of capacities and dispatch.}
\label{DOC-T-RuralUrbandispatchetc}
\end{table}



\begin{table}
\center{
 {\small
\begin{tabular}{ |l|l|l|l|l|l|l| } \hline
Tech. type: & IGCND &  IGBPR& IGEXT &IGHONLY&IGETOH &IGHSTO  \\
GDTYPE no.: & 1     & 2     & 3     & 4     & 5     & 6      \\
 \hline
 IGEE       & yes   &       &       &       &       &        \\
 IGHH       &       &       &       & yes   &       & yes    \\
 IGHHNOSTO  &       &       &       & yes   &       &        \\
 IGEENOSTO  & yes   &       &       &       &       &        \\
 IGKH       &       &       &       & yes   &       &  yes   \\
 IGKE       & yes   &yes    & yes   &       & yes   &        \\
 IGKENOSTO  & yes   &yes    & yes   &       & yes   &        \\
 IGKHNOSTO  &       &       &       & yes   &       &        \\
 IGEOREH    & yes   & yes   & yes   &       & yes   &        \\
 IGNOTETOH  & yes   & yes   & yes   & yes   &       & yes    \\
 IGDISPATCH & yes   & yes   &   yes &  yes  &  yes  & yes    \\
 IGHORHERUR &       & yes   &  yes  & yes   &yes    &        \\
\hline \hline

Tech. type: & IGESTO& IGHYRS&IGHYRR &IGWND  &IGSOL  &        \\
GDTYPE no.: &  7    & 8     &  9    & 10    & 11    &        \\
 \hline
 IGEE       &      & yes    &yes    & yes   &yes    &        \\
 IGHH       &      &        &       &       &       &        \\
 IGHHNOSTO  &      &        &       &       &       &        \\
 IGEENOSTO  &      & yes    &yes    & yes   & yes   &        \\
 IGKH       &      &        &       &       &       &        \\
 IGKE       & yes  & yes    &yes    & yes   & yes   &        \\
 IGKENOSTO  &      & yes    &yes    & yes   & yes   &        \\
 IGKHNOSTO  &      &        &       &       &       &        \\
 IGEOREH    & yes  & yes    &yes    & yes   &  yes  &        \\
 IGNOTETOH  & yes  & yes    &yes    & yes   & yes   &        \\
 IGDISPATCH & yes  & yes    &       &       &       &        \\
 IGHORHERUR &      &        &       &       &       &        \\
\hline
\end{tabular}
} % end small
} % center
\caption{Dynamic sets with relation to generation technology
types, i.e., depending on GDTYPE.} \label{DOC-T-dynamic-G-sets}
\end{table}








\subsubsection{Equation feasibility: IPLUSMINUS}
 \label{DOC-SSS-IPLUSMINUS}\index{IPLUSMINUS}\label{IPLUSMINUS}


It is not practically possible to ensure that a   model will have
a feasible solution. And if it does not, it may be difficult to
find the explanation why. Therefore a mechanism is introduced to
ensure that a model will always be feasible, and to provide some
kind of indication that may help in searching for a reason. The
SET IPLUSMINUS /IPLUS, IMINUS/ is part of this, see further
Section \ref{IPENALTY}, Section \ref{DOC-S-Variables}.


\subsection{Restrictions on sets}
 \label{DOC-SS-Sets-Exceptions}


Most of the sets have their members\index{member of set}
(elements, labels)\index{element in set}\index{label} specified by
the user. The exceptions are:
\begin{itemize}
  \item  Some sets have their members derived automatically
  (they are the dynamic sets, or internal sets,
   Section \ref{DOC-SSS-staticanddynamicsets}); these
sets are described in Section \ref{DOC-SS-Internal-Parm-Sets}.
  \item  For the sets GDATASET, FDATASET, DF\_QP, MPOLSET,
    see Sections \ref{GDATASET},
\ref{FDATASET},  \ref{DF-QP}, \ref{MPOLSET}, the members should
not be changed without proper understanding  of the functioning
of these sets.
 \item For the sets DEF,
DEF\_DINF, DEF\_UINF, DHF, DHF\_DINF and DHF\_UINF there are
restriction such that some specific elements must be present, see
Section \ref{DOC-SS-Demand}.
\end{itemize}

The set members (labels) in relation to the two last items will be
referred to as obligatory set members. \label{obligatorysetmember}
 \index{obligatory set member}

It is permitted that an  element is   member of more than one
set. This is used e.g. in the declaration of subsets, and other
examples are given in Section
\ref{DOC-SSS-Generation-technology-types}.  However, apart from
such  intentional use, it should be avoided that elements are
members of more than one set. Thus, it may   be tempting to have
e.g. a technology that is called HYDRO and also a fuel that is
called HYDRO, but this should be avoided, see the discussion on
naming conventions on page \pageref{DOC-namingHYDROWATER} in
Section \ref{DOC-SS-Technologies}.



For similar reasons, the  only\index{label} labels consisting of
digits only are those used for set elements indicating the years
(i.e., in labels the sets YYY and Y).




Finally, as previously noted, the user should not change the
internal sets, cf. Section \ref{DOC-SS-Internal-Parm-Sets}.

\section{Parameters and scalars}
\label{DOC-S-Parms-and-Scalars}




In this section we describe parameters and scalars that must be
specified by the   user while in Section
\ref{DOC-SS-Internal-Parm} parameters that are automatically
calculated will be treated. The former type will be referred to as
input parameters\index{input parameter} while the latter type will
be referred to as internal parameters.\index{internal parameter}

Recall from Section \ref{DOC-SS-GAMS-intro}
 that parameters\index{parameter} and scalars \index{scalar} are used to
specify exogenous \index{exogenous} values, and that parameters,
unlike scalars, relate to sets.

Recall that the focus in the present document is on model
structure and therefore the actual input data to be used is not in
focus. However, occationally  some comments on input data will be
given. To avoid confusion between what is logically required
within the model structure and what may reasonable be expected
concerning numerical values of input the comments on input data
will be clearly identified as such.

The units\index{unit (data)}\label{DOC-page-unitsdiscussion} used
in parameters are: MW (megawatt)\index{MW}\index{megawatt}, MWh
(megawatthour)\index{MWh}, GJ\index{GJ}\index{Gigajoule}, hours,
days, kg (kilogram)\index{k, kg, kilogram}\index{kilogram}, t
(ton)\index{t (ton)}\index{ton},   and Money\index{Money} where
the latter may indicate e.g. Euro\index{Euro}\index{USD} or USD.
Prefixes like M (million)\index{M (million)},
 k (kilo)\index{k (kilo)}  or milli (m)\index{milli}\index{m (milli)} will also be used.

For entities like 'loss in electricity distribution' (which must
be given as a fraction) or 'the number of hours per year' the unit
has been indicated as\index{(none)} '(none)'. For some entities
the important thing is their proportions, in this case also
'(none)' may be specified, however, an indication may be given as
concerns between which entities the proportions should be taken;
e.g. '(none$\sim$MW)' to signal that the proportions are between
MW or similar, see Section \ref{DE-VAR-T}.

The factor 3.6\index{3.6} indicates the usual relations between
units using seconds and hours, respectively, e.g. between MWh and
GJ. The meaning of the numbers 24\index{24}, 365\index{365},
8760\index{8760} are obvious.

Most data are entered using a list\index{list} (for scalars) or a
TABLE (for parameters with two or more dimensions). Observe that
if entries are not given, or entry values are not filled in, the
default\index{default} value zero is automatically used. In some
cases where individual data can not be found, or can not be found
for all relevant entries, user specified default\index{default
(user specified)} data may be entered by first giving a TABLE
with those values that are known, and then filling all other
entries with the user specified default data, see Section
\ref{DOC-SS-Parm-defaultvalues}.

In general, the data in any model must be selected by the user to
be consistent in the sense that model and data are, generally
speaking, mutually dependent, and therefore the individual data
elements are also interdependent. This topic will not be
discussed in general terms here. However, in a few places it is
crucial that there is a logical consistency between parameter
values, this is discussed in  Section
\ref{DOC-SS-Parm-Exceptions}.


\subsection{Scalars}
\label{DOC-SS-Scalars}

\subsubsection{STARTYEAR}

 The  parameter STARTYEAR \index{STARTYEAR}\label{STARTYEAR}  is used  to indicate the
first year in the simulation.  It  must  indicate the first year
in the set Y (Section \ref{DOC-SSS-Timewithinyears}); thus, if
the first year in Y is e.g. 1997, then specify STARTYEAR=1997.


See also YEARINC in Section \ref{YEARINC}.


\subsubsection{YEARINC}

In order to simplify experimental calculations the scalar
YEARINC\index{YEARINC}\label{YEARINC} has been introduced. If
this is initialised to e.g. 5  this  means that only every 5th
year in the set Y is to be simulated, starting with STARTYEAR,
see Section \ref{STARTYEAR}.  YEARINC must be a positive integer.

Observe that skipping every fifth  year may also be achieved by
specification in the set Y, e.g. "SET Y / 1995, 2000, 2005 /;" for
simulation of these three years only. In some cases this is an
attractive alternative to giving YEARINC the values 5.


\subsection{Parameters on the set YYY}
 \label{DOC-SS-Parm-YYY}
\subsubsection{YVALUE}
 \index{YVALUE}\label{YVALUE}
The parameter YVALUE(YYY) holds the numerical values related to
the years in set YYY. Unit: (none).


\subsection{Parameters on the set SSS}
\label{DOC-SS-Parm-SSS}

\subsubsection{WEIGHT\_S}

PARAMETER WEIGHT\_S   \index{WEIGHT\_S}\label{WEIGHT-S} reflects
how much of the year each season represents expressed relatively
between the seasons. Unit: (none), see next.

One way of doing it is to state the number of days in each season
(could sum up to 365\index{365}). Another is to give percentages
(summing up to 100), cf. Section \ref{WEIGHT-T}.


The parameter WEIGHT\_S is used only in the calculation of the
parameters  IWEIGHSUMS and DAYSIN\_S, see Sections
\ref{IWEIGHSUMS} and  \ref{IDAYSIN-S}.

\inputdata The calculation of IDAYSIN\_S will involve   a division.
This will be unproblematic if  the data entered in WEIGHT\_S
contain  no negative values and at least one positive value
(which is natural).



(It  is quite possible to specify the year to have 366\index{365,
366, 365.24} (or even 365.24) days. Just take the editor and
replace 365 by 366 (or 365.24) in the BALMOREL.GMS file (and
8760\index{8760}
 should be changed accordingly to 8784\index{8784} etc.). This will make the
numerical values change slightly while the interpretations will
be harder.)

\naming See Section \ref{DOC-SSS-Timewithinyears}.

\inputdata See the comments on input data in Section
\ref{WEIGHT-T}.





\subsection{Parameters on the set TTT }
\label{DOC-SS-Parm-TTT}

\subsubsection{WEIGHT\_T}

PARAMETER WEIGHT\_T    \index{WEIGHT\_T}\label{WEIGHT-T} reflects
how much of the season  each time segment represents expressed
relatively between the time segments. Unit: (none), see next.

One way of doing it is to state the number of hours  that each
period represents,  another is to state it as percentages (summing
up to 100), cf. Section \ref{WEIGHT-S}.

The parameter WEIGHT\_T  is used only in the calculation of the
parameters IWEIGHSUMT and IHOURSIN24, see  Sections
\ref{IWEIGHSUMT} and \ref{IHOURSIN24}.


It  tempting to  say that the set TTT represents a subdivision of
the day - and we may actually do so sometimes. However,  is  not
in general  correct to say so.  Thus, if for instance SSS contains
four elements (wither, spring, summer, autumn) and TTT contains 24
segments, the time segments in  TTT need not represent a typical
day. Rather,  TTT should represent not only the "typical day" but
also the week-end days. See also Section \ref{IHOURSIN24}


\inputdata The calculation of  IHOURSIN24 will involve   a division.
This will be unproblematic if  the data entered in  WEIGHT\_T
contain  no negative values and at least one positive value
(which is natural).



\inputdata The set T is ordered, cf. Section  \ref{DOC-SSS-Orderedsets}.
If this is used in the model to represent
chronological\index{chronology} aspects then the sequence of
numbers entered in TABLE WEIGHT\_T matters. This is the case e.g.
if pumped hydro reservoirs\index{storage}\index{reservoir} or heat
storage shall be modelled. However, if this is not so, WEIGHT\_T
may be used to represent only the duration\index{duration}
(weight) of the individual time segments. Then, if  e.g. the
electricity loads (demand)  as expressed in PARAMETER DE\_VAR\_T
appear in descending  magnitude the  load duration\index{load
duration} idea is applied. Similarly it  is important that the
set S is ordered in e.g. a hydro reservoir  with seasonal storage
capacity is to be modelled.

\naming See Section\ref{DOC-SSS-Timewithinyears}.



\subsection{Parameters on the set CCC }
\label{DOC-SS-Parm-CCC}

\subsubsection{ANNUITYC}

PARAMETER ANNUITYC   \index{ANNUITYC}\label{ANNUITYC}  indicates
the transformation of an investment to an annual payment. Unit:
(none).

Thus, for instance, an investment of 100, paid over 20 years,
with payment at the end of each year, assuming an interest rate of
5\%, will imply an annual payment of 8.02, hence, ANNUITYC should
in this case have the value 0.0802.

For electrical transmission investments between regions in two
different countries, the average annuity between the  annuities
for  the two countries in question will be used.





Table \ref{DOC-T-Annuities} illustrates the dependence of the
annuity on  various combinations of interest rates, 5, 6, 7, 8, 9,
10, 15, 20\% and number of years, 5, 10, 15, 20, 25, 30 years.
See also  Section \ref{DOC-S-Modifications}.




\begin{table}
\center{ {\small
\begin{tabular}{|r|l|l|l|l|l|l|l|l|} \hline
    &  5\%  &    6\%  &   7\%   &    8\%  &     9\% &    10\% &    15\% &     20\% \\
\hline
  5& 0.2310 &  0.2374 &  0.2439 &  0.2505 &  0.2571 &  0.2638 &  0.2983 &  0.3344 \\
 10& 0.1295 &  0.1359 &  0.1424 &  0.1490 &  0.1558 &  0.1627 &  0.1993 &  0.2385 \\
 15& 0.0963 &  0.1030 &  0.1098 &  0.1168 &  0.1241 &  0.1315 &  0.1710 &  0.2139 \\
 20& 0.0802 &  0.0872 &  0.0944 &  0.1019 &  0.1095 &  0.1175 &  0.1598 &  0.2054 \\
 25& 0.0710 &  0.0782 &  0.0858 &  0.0937 &  0.1018 &  0.1102 &  0.1547 &  0.2021 \\
 30& 0.0651 &  0.0726 &  0.0806 &  0.0888 &  0.0973 &  0.1061 &  0.1523 &  0.2008\\
\hline
\end{tabular}
}% small
}% center
\caption{Annuity as depending on interest rates and number of
years.} \label{DOC-T-Annuities}
\end{table}






\subsubsection{TAX\_DE}

PARAMETER TAX\_DE \index{TAX\_DE}\label{TAX-DE} holds consumers'
tax on electricity consumption. Unit: \MONEY/MWh.

\inputdata  Observe that the tax must be specified as the weighted average
over  all consumer groups.



\subsubsection{TAX\_DH}

PARAMETER TAX\_DH \index{TAX\_DH}\label{TAX-DH} holds consumers'
tax on heat consumption. Unit: \MONEY/MWh.

\inputdata  Observe that the tax must be specified as the weighted average
over  all consumer groups.





\subsection{Parameters on the set RRR }
\label{DOC-SS-Parm-RRR}




\subsubsection{DISLOSS\_E}


PARAMETER DISLOSS\_E
\index{DISLOSS\_E}\label{DISLOSS-E}\index{loss, el. distribution}
holds the loss in electricity distribution, as a fraction of the
electricity entering the distribution network. Unit: (none).



\subsubsection{DISCOST\_E}

PARAMETER  DISCOST\_E \index{DISCOST\_E}\label{DISCOST-E} holds
the    cost of electricity distribution,
 given relative to end consumption.  Unit: \MONEY/MWh.




\subsubsection{DEFP\_BASE}


PARAMETER DEFP\_BASE \index{DEFP\_BASE}\label{DEFP-BASE} holds
the annual average consumer price of electricity  (including
taxes) in the base year. Unit: Money/MWh.

\inputdata  Observe that the average is to be taken over the whole year and
over all consumer groups, and that the price is including taxes
(that may differ between the different consumer groups).

\inputdata  Also in the case of inelastic demand a reasonable value must be
given, as the value in DEFP\_BASE will be also in this case be
taken as starting point for calculation of a "very high" price,
used in case the demand can not be satisfied, see also Section
\ref{DEF-STEPS}.




\subsection{Parameters on the set AAA }
\label{DOC-SS-Parm-AAA}

\subsubsection{DISLOSS\_H}

PARAMETER HDIS\_LOSS
\index{DISLOSS\_H}\label{DISLOSS-H}\index{loss, heat distribution}
holds the loss in heat distribution, as a fraction of heat
generated (identical to the heat entering the distribution
network). Unit: (none).


\subsubsection{DISCOST\_H}

PARAMETER  DISCOST\_H \index{DISCOST\_H}\label{DISCOST-H} holds
the cost of heat distribution,  given relative to end consumption.
Unit: \MONEY/MWh.






\subsubsection{DHFP\_BASE}


PARAMETER DHFP\_BASE \index{DHFP\_BASE}\label{DHFP-BASE} holds the
annual average consumer price of heat (including taxes) in the
base year.    Unit: Money/MWh.

\inputdata  Similar comments as in Section \ref{DEFP-BASE} apply.



\subsection{Parameters on the set product (FFF,FDATASET)}
\label{DOC-SS-Parm-F-FDATASET}

\subsubsection{FDATA}

PARAMETER FDATA \index{FDATA}\label{FDATA} contains information
about emission characteristics of fuels. In addition it contains
an integer code FDNB identifying the individual fuels.
 Units: kg/GJ (for FDCO2),  kg/GJ (for FDSO2), (none) for FDNB.

The FDNB contributes to the   coupling between generation
technology and fuel. In GDATASET (Section
\ref{DOC-SS-Technologies}) the elements GDFUEL (Section
\ref{GDFUEL})  for each technology contains an integer that points
to the FDNB for  the   fuel that the technology uses. Therefore
two different fuels should not have identical FDNB.







\subsection{Parameters on the set product  (FFF,CCC)}
\label{DOC-SS-Parm-FFF-CCC}


\subsubsection{TAX\_F  }

PARAMETER TAX\_F   \index{TAX\_F}\label{TAX-F}  specifies fuel
taxes on primary fuel types (i.e. neither electricity nor heat).
This tax is applied on the fuel, IRRREspective of whether
electricity, heat or both is produced. Unit: \MONEY/GJ.





\subsection{Parameters on the set product (GGG,AAA)}
\label{DOC-SS-Parm-GGG-AAA}


\subsubsection{GKINI}

PARAMETER GKINI \index{GKINI}\label{GKINI}\index{capacity} holds
the generation capacity in the initial year. Unit: MW.

For electricity generation plants and
co-generation\index{co-generation}\index{chp} plants this must be
specified with respect to electricity generation. For heat only
boilers and electrical heating units the capacity must be
specified with respect to  heat generation. See Section
\ref{DOC-SSS-Generation-technology-types}.

This way of specifying capacity is used also in all other
contexts,    with implications for parameters that associate with
capacity, e.g. GDINVCOST Section \ref{GDINVCOST}.

The initial year is the first year in the simulation, i.e. the
first year in the set Y.


\subsubsection{GDINVCOST}

 PARAMETER GDINVCOST \index{GDINVCOST}\label{GDINVCOST}  holds the investment cost for new technology.
 Unit: M\MONEY/MW.

Observe the definition of the capacity (MW), Section
\ref{DOC-SSS-Generation-technology-types}.

Observe that if a zero or if  nothing is specified for GDINVCOST
in TABLE GDINVCOST  (and therefore the default\index{default}
value zero is automatically assigned) then the value in table
GDATA, Section \ref{GDATA}, is used.

\subsubsection{GDOMVCOST}
PARAMETER GDOMVCOST \index{GDOMVCOST}\label{GDOMVCOST} holds the
variable operating and maintenance costs.
 Unit: \MONEY/MWh.

Observe that if a zero or if  nothing is specified for GDOMVCOST
in TABLE  GDOMVCOST (and therefore the default\index{default}
value zero is automatically assigned) then the value in table
GDATA, Section \ref{GDATA}, is used.


\subsubsection{GDOMFCOST}
 PARAMETER GDOMFCOST \index{GDOMFCOST}\label{GDOMFCOST} holds the annual fixed operating and maintenance costs.
 Unit:  k\MONEY/MW.

Observe the definition of the capacity (MW), Section
\ref{DOC-SSS-Generation-technology-types}.

Observe that if a zero or if  nothing  is specified for GDOMFCOST
in TABLE  GDOMFCOST (and therefore the default\index{default}
value zero is automatically assigned) then the value in table
GDATA, Section \ref{GDATA}, is used.











































\subsubsection{GEFFDERATE}

PARAMETER GEFFDERATE \index{GEFFDERATE}\label{GEFFDERATE}
represents an adjustment of  efficiency. Unit: (none).

\inputdata This parameter is intended for catching some of  the
shortcomings in the modeling of the individual  units. Thus,  the
information on efficiency given in GDATA may be seen as general
information with validity irrespective of where the unit is
located, this is then made geographically specific through
GEFFDERATE.  The value of  GEFFDERATE  will probably be slightly
below unity (but despite the name it may also be above unity).
 See also Section \ref{DOC-SS-Calib-GEFFDERATE}.






\subsection{Parameters on the set product (YYY,RRR) }
\label{DOC-Parm-YYY-RRR}


\subsubsection{X3FX }
PARAMETER X3FX \index{X3FX}\label{X3FX} contains the annual net
electricity export to third regions. Unit: MWh.

\inputdata Observe that the values in X3FX must be specified to be
consistent with the values in  X3FX\_VAR\_T, Section
\ref{X3FX-VAR-T}. X3FX is only used to calculate IX3FX\_T\_Y, see
Section \ref{IX3FX-T-Y}.

Observe that this exchange (intended to be positive for export,
negative for import, but from Section \ref{IX3FX-T-Y} it follows
that some care is necessary to get consistency) is specified by
the user, and that there is no other exchange possibilities with
regions or countries not in the model (i.e., not in the sets C or
R). Also observe that no payment is associated with this exchange.
No capacity is to be given for the transmission lines supposed to
carry the exchange (and hence no interaction with e.g. XKINI,
Section \ref{XKINI}).

Exchange between regions in the model (i.e., between members in
the set R) will be found during the simulation as the endogenous
value of variable VX\_T, see Section \ref{DOC-S-Variables}.



\subsubsection{DE}

PARAMETER DE \index{DE}\label{DE} contains the
nominal\index{nominal} annual electricity consumption.  Unit: MWh.

The value should be the end consumption, since distribution and
transmission losses are accounted for separately.

The nominal annual consumption will be distributed over the time
segments over the year, see Section \ref{DOC-SS-Time}. If demand
is elastic, there may be deviation from this nominal value, see
Section \ref{DOC-SS-Demand}.






\subsection{Parameters on the set product (YYY,AAA) }
\label{DOC-SS-Parm-YYY-AAA}

\subsubsection{DH}

PARAMETER DH \index{DH}\label{DH} contains the
nominal\index{nominal} annual heat consumption in those areas that
are heat areas, Section \ref{DOC-SSS-AAAURBH-AAARURH}. Unit: MWh.


The value should be the end consumption, since distribution
losses are accounted for separately.

The nominal annual consumption will be distributed over the time
segments over the year, see Section \ref{DOC-SS-Demand}. If
demand is elastic, there may be deviation from this nominal
value, see Section \ref{DOC-SS-Demand}.




\subsection{Parameters on the set product (GGG,GDATASET)}
\label{DOC-Parm-GGG-GDATASET}

\subsubsection{GDATA}

PARAMETER GDATA\index{GDATA}\label{GDATA} contains information
about the individual  generation technologies.



\begin{itemize}
 \item[]   GDTYPE: \index{GDTYPE}\label{GDTYPE} This is an integer. According to the
value of this integer, the technology is uniquely placed in one of
the internal sets specified in Section
\ref{DOC-SSS-Generation-technology-types}. According to the value
of the integer (and hence according to which of those sets the
technology belongs), the technology has specific properties, see
 Section \ref{DOC-SSS-Generation-technology-types}.

 \item[]      GDFUEL: \index{GDFUEL}\label{GDFUEL} This is an integer indicating which
fuel the technology uses, corresponding to the fuel number FDNB
given in FDATA (Section \ref{FDATA}).

 \item[]      GDCB:       \index{GDCB}\label{GDCB} This value specifies the Cb-value
for back pressure and  extraction type technologies.
% It   is used in
%equations G\_CMBP(A,IGBPR,S,T),   GN\_CMBP(A,IGBPR,S,T),
%G\_CMEX(A,IGEXT,S,T),   GN\_CMEX(A,IGEXT,S,T), and the
%specifications of bounds   VGH\_T.UP(A,IGBPR,S,T),
%VGH\_T.UP(A,IGEXT,S,T).


 \item[]     GDCV:   This value specifies the isofuel constant
  Cv\index{GDCV}\label{GDCV} for  extraction type technologies
 \item[]     GDFE:       Fuel efficiency\index{GDFE}\label{GDFE}
 \item[]     GDESO2:      Degree of desulphuring\index{GDESO2}\label{GDESO2}
 \item[]     GDNOX:    NOx-factor (mg/MJ)\index{GDNOX}\label{GDNOX}
 \item[]     GDAUXIL: Used for various additional information.
            For CHP it denotes central (urban) or decentral (rural) technology\index{GDAUXIL}\label{GDAUXIL}
 \item[]     GDINVCOST0:  Default investment cost (M\MONEY/MW) (default
             value).\index{GDINVCOST0}\label{GDINVCOST0}
              Will be used, if nothing is specified for GDINVCOST,  Section
              \ref{GDINVCOST}.
              Observe the
             definition of capacity, Section \ref{DOC-SSS-Generation-technology-types}
 \item[]     GDOMVCOST0:  Default variable operating and maintenance costs (\MONEY/MWh)
          (default value).\index{GDOMVCOST0}\label{GDOMVCOST0} The cost is specified with respect to
          the total energy (electricity plus heat)
           Will be used, if nothing is specified for GDOMVCOST,  Section \ref{GDOMVCOST}.
 \item[]     GDOMFCOST0:  Default annual operating and maintenance costs (k\MONEY/MW)
           (default value).\index{GDOMFCOST0}\label{GDOMFCOST0} Observe the
             definition of capacity, Section \ref{DOC-SSS-Generation-technology-types}
              Will be used, if nothing is specified for GDOMFCOST,  Section \ref{GDOMFCOST}.
 \item[]     GDFROMYEAR:  technology available from the beginning of this year\index{GDFROMYEAR }\label{GDFROMYEAR}
 \item[]     GDKVARIABL:  Capacity is  a variable to be found for each year
    (0: no, 1: yes)\index{GDKVARIABL}\label{GDKVARIABL}
\end{itemize}


The data corresponding to the  elements GDINVCOST0, GDOMVCOST0 and
GDOMFCOST0 are considered as default values that may be
overwritten, see Sections   \ref{GDINVCOST}, \ref{GDOMVCOST} and
\ref{GDOMFCOST}.

Observe that the following will be specified automatically in the
model:
\begin{itemize}
  \item[]  GDATA(IGBPR,'GDCV')=1;
  \item [] GDATA(IGHOB,'GDCV')=1;
\end{itemize}
This ensures homogeneous ways of calculating fuel consumption.
Therefore these obligatory\index{obligatory value} values should
not be given by the user (i.e., the corresponding  entries in
TABLE GDATA should be left blank).


\subsection{Parameters on the set product (IRRRE,IRRRI) }
\label{DOC-Parm-IRRRE-IRRRI}

\subsubsection{XKINI}

PARAMETER XKINI \index{XKINI}\label{XKINI}\index{capacity}
contains the initial electrical transmission capacities between
pairs of regions. Unit: MW.

The electrical transmission
 capacity\index{capacity (transmission)}
  is the capacity disregarding an eventual loss
(see the table XLOSS). Thus, if there is a loss XLOSS, a maximum
of XKINI MW may be sent into the transmission line, but at most
(XKINI*(1-XLOSS)) MW may be extracted.

Observe that the  initial transmission capacity between two
regions need not be the same in both directions. (But  new
transmission capacity will be symmetric.)


\subsubsection{XINVCOST}

PARAMETER XINVCOST \index{XINVCOST}\label{XINVCOST} contains
information about the investment cost in new electrical
transmission capacity between pairs of regions. Unit: \MONEY/MW.

Observe the definition of transmission capacity, Section
\ref{XKINI}.

In contrast to tables providing the other information relating to
transmission (XKINI, XCOST, XLOSS), this table must be symmetric.

\subsubsection{XCOST}

PARAMETER XCOST \index{XCOST}\label{XCOST} contains information
about the electrical transmission cost between pairs of regions.
Unit: \MONEY/MWh.


The electrical transmission cost is applied to the electricity
entering the transmission line, cf. Section \ref{XKINI}.

Observe that the cost need not be the same in both directions.

\inputdata  Unreasonable results may be found if there are
neither cost nor loss associated with electrical transmission.
Therefore for all non-diagonal entries the user must enter a
positive number in either TABLE XCOST or in TABLE XLOSS, Section
\ref{XLOSS}.




\subsubsection{XLOSS}

PARAMETER\index{XLOSS}\label{XLOSS}
 XLOSS\index{loss, el. transmission}
  contains the loss in transmission  expressed as a
fraction of the electricity entering the transmission line. Unit:
(none).

Observe that the loss need not be the same in both directions,
Section \ref{XKINI}.

\inputdata  Unreasonable results may be found if there are
neither cost nor loss associated with electrical transmission.
Therefore for all non-diagonal entries the user must enter a
positive number in either   XLOSS or in   XCOST, Section
\ref{XCOST}.



\subsection{Parameters on the set product (FKPOTSETA,AAA) }
\label{DOC-SS-Parm-FKPOTSETA-AAA}

\subsubsection{FKPOTA}
PARAMETER FKPOTA\index{FKPOTA}\label{FKPOTA} holds the fuel
potentials specified at area level. Unit: MW.

The potential is specified as an upper limit on the areas'
installed  generation  capacity relative to the individual fuel
type.

\inputdata The specification of the fuel potential as an upper
limit on generation capacity is obviously a simplification.
However, in many cases the fuel potential, defined in energy
terms, is not known with precision and this motivates the
simplification. See Section \ref{DOC-S-Modifications}.

\subsection{Parameters on the set product (FKPOTSETR,RRR) }
\label{DOC-SS-Parm-FKPOTSETR-RRR}

\subsubsection{FKPOTR}
PARAMETER FKPOTR\index{FKPOTR}\label{FKPOTR} holds the fuel
potentials specified at regional level. Unit: MW.

The potential is specified as an upper limit on the regions'
installed  generation   capacity relative to the individual fuel
type.

\inputdata The same comments as in Section \ref{FKPOTA} apply.


\subsection{Parameters on the set product (FKPOTSETC,CCC) }
\label{DOC-SS-Parm-FKPOTSETC-CCC}

\subsubsection{FKPOTC}
PARAMETER FKPOTC\index{FKPOTC}\label{FKPOTC} holds the fuel
potentials specified at country level. Unit: MW.

The potential is specified as an upper limit on the countries'
installed  generation   capacity relative to the individual fuel
type.

\inputdata The same comments as in Section \ref{FKPOTA} apply.

\subsection{Parameters on the set product (AAA,SSS)}
 \label{DOC-Parm-RRR-SSS}

\subsubsection{WTRRSVARS}
 \index{WTRRSVARS} \label{WTRRSVARS}
 PARAMETER WTRRSVARS contains the description of the seasonal
variation of the  amount of water inflow to the hydro reservoirs
 with storage. Unit: (none$\sim$MW).

The water is assumed available at the beginning of each season.


\subsubsection{HYPPROFILS}
\index{HYPPROFILS} \label{HYPPROFILS}

PARAMETER HYPPROFILS contains the description of the seasonal
variation of prices in relation to production of electricity from
hydro power with storage. Unit: Money/MWh.




\subsection{Parameters on the set product (AAA,SSS,TTT)}
\label{DOC-SS-Parm-AAA-SSS-TTT}

\subsubsection{DH\_VAR\_T}

PARAMETER DH\_VAR\_T \index{DH\_VAR\_T}\label{DH-VAR-T} contains
the description of seasonal and daily variation of the heat demand
Unit: (none$\sim$MW) (see description in relation to  DE\_VAR\_T,
Section \ref{DE-VAR-T}).


\subsubsection{SOL\_VAR\_T}

PARAMETER SOL\_VAR\_T \index{SOL\_VAR\_T}\label{SOL-VAR-T}
contains the description of seasonal and daily variation of the
solar power generation. Unit: (none$\sim$MW) (see description in
relation to DE\_VAR\_T, Section \ref{DE-VAR-T}).


\subsubsection{WND\_VAR\_T}

PARAMETER WND\_VAR\_T \index{WND\_VAR\_T}\label{WND-VAR-T}
contains the description of seasonal and daily variation of the
wind power generation. Unit: (none$\sim$MW) (see description in
relation to DE\_VAR\_T, Section \ref{DE-VAR-T}).

\subsubsection{WTRRRVAR\_T}\index{WTRRRVAR\_T}\label{WTRRRVAR-T}

PARAMETER WTRRRVAR\_T contains the description of the seasonal and
daily variation of the amount of water inflow to the hydro
electricity  generation without storage (run of river). Unit:
(none$\sim$MW) (see description in relation to DE\_VAR\_T, Section
\ref{DE-VAR-T}).



\subsubsection{DHFP\_CALIB}
\index{DHFP\_CALIB}\label{DHFP-CALIB}


PARAMETER DHFP\_CALIB is used to calibrate the price side of the
demand function  for heat in order to get demand consistent for a
base year. Unit: \MONEY/MWh.

The intention with this parameter is the following.  Balance
between supply and demand is obtained as a consequence of the
costs on the supply side  and the demand function. The demand
function is exogenously specified, and the supply function is
found in the model (in the sense that supply costs may be
calculated). However, it may be unlikely that the model and data
will reproduce accurately the situation in  a base years, such
that the simulated demand need not correspond to that observed in
the base year. The parameter DHFP\_CALIB may then be given a value
different from zero to obtain such correspondence.

 The parameter is used in the calculation of IDHFP\_T, Section
\ref{DHFP-T} and Section \ref{DOC-S-Calibration}.













\subsection{Parameters on the set product (RRR,SSS,TTT) }
\label{DOC-SS-Parm-RRR-SSS-TTT}

\subsubsection{DE\_VAR\_T}

PARAMETER DE\_VAR\_T \index{DE\_VAR\_T}\label{DE-VAR-T} contains
the description of seasonal and daily variation of the
electricity demand. Unit: (none) (see below).

DE\_VAR\_T is used to calculate IDE\_SUMST (Section
\ref{IDE-SUMST}) and DE\_VAR\_T and IDE\_SUMST in combination with
DE are used to calculate IDE\_T\_Y (Section \ref{IDE-T-Y}).


The values in DE\_VAR\_T are interpreted to be  specified
relatively (i.e. the values for each day  or for all time segments
do not have to sum up to something specific, only the relative
values are important) within each region. One way to do this is to
specify each season/time period value as a percentage of the
yearly maximum power load. Another option is to specify the
MW-loads for each combinations.

In any case it is important to note that the values must be
derived from data given with the dimension of power, i.e., energy
per time unit, e.g. MW, GW, J/s MJ/s, and not with dimension of
energy, e.g. MHh or MJ.


\inputdata The calculation of parameter IDE\_T\_Y, Section
\ref{IDE-T-Y}, involves a division. This will be unproblematic if
 the data entered in DE\_VAR\_T  contain  no negative values and at least one
positive value (which is natural).

Observe that parameters  DE\_VAR\_T, DH\_VAR\_T,  WND\_VAR\_T,
SOL\_VAR\_T,  X3FX\_VAR\_T   will be handled in similar ways as
DE\_VAR\_T (special care should be taken with X3FX\_VAR\_T, see
Section \ref{X3FX-VAR-T}).



\subsubsection{DEFP\_CALIB}
\index{DEFP\_CALIB}\label{DEFP-CALIB}

 PARAMETER DEFP\_CALIB is used to calibrate the price side of
the demand function  for electricity  in order to get demand
consistent for a base year. Unit: \MONEY/MWh.

See the explanation in relation to DHFP\_CALIB, Section
\ref{DHFP-CALIB}.

The parameter is used in the calculation of IDEFP\_T, Section
\ref{IDEFP-T} and Section \ref{DOC-S-Calibration}.





\subsubsection{X3FX\_VAR\_T}

PARAMETER X3FX\_VAR\_T \index{X3FX\_VAR\_T}\label{X3FX-VAR-T}
contains the description of seasonal and daily variation in the
fixed exchange with third regions. Unit:  (none$\sim$MW) (see
description in relation to DE\_VAR\_T, Section \ref{DE-VAR-T}).

A positive number is intended to indicate net export from the
region in the set RRR, a negative number indicates net import to
the region in the set RRR for the given time segment (S,T),
however see Section \ref{IX3FX-T-Y} about consistency.  The values
are seen from the country in the set CCC, i.e., any losses are
disregarded.

X3FX\_VAR\_T is only used to calculate X3FX\_T\_Y, see Section
\ref{IX3FX-T-Y}.


\inputdata Observe that the values in X3FX\_VAR\_T must be specified to be
consistent with the values in  X3FX, Section \ref{X3FX}, cf.
Section \ref{IX3FX-T-Y}.


\subsection{Parameters on the set product (GGG,AAA,SSS)}
\label{DOC-SS-Parm-GGG-AAA-SSS}


\subsubsection{GKDERATE }

PARAMETER GKDERATE \index{GKDERATE}\label{GKDERATE} represents a
reduction in capacity. Unit: (none).

This reduction may represent e.g. forced and scheduled outages.
It is used to reduce the capacity of each unit type in each area.

Observe that for all technology types the specification of
GKDERATE has implication for capacity of both electricity and
heat (if both are relevant).


\inputdata  If GKDERATE is set ot zero for some paricular
combination (G,A,S) then this represents an outage for the whole
season, typically a planned outage.

To represent stochastic outages for thermal units a reasonable
value will probably be close to but smaller than 1. For other
types of units (e.g., wind, hydro, solar, heatpumps) the value
should probably be equal to 1; however, this will depend on a
number of factors, e.g. the data source. See also Section
\ref{DOC-SS-Calib-GKDERATE}.



\subsection{Parameters on the set product (YYY,AAA,GGG)}
\label{DOC-SS-Parm-YYY-AAA-GGG}

\subsubsection{GKFX}

PARAMETER GKFX\index{GKFX}\label{GKFX} holds the exogenously
specified generation capacities. Unit: MW.

Observe that generation capacities\index{capacity} are considered
specified  as net capacity, i.e., the generation unit is assumed
to be able to deliver such amount to the network (distribution or
transmission network for electricity, distribution network for
heat). On the other hand, the delivery may be modified by
GKDERATE, see  Section \ref{GKDERATE}.

For electricity generation plants and
co-generation\index{co-generation}\index{chp} plants this must be
specified with respect to electricity generation. For heat only
boilers and electrical heating units the capacity must be
specified with respect to  heat generation. See Section
\ref{DOC-SSS-Generation-technology-types}.



\subsection{Parameters on the set product (YYY,AAA,FFF)}
\label{DOC-SS-Parm-YYY-AAA-FFF}


\subsubsection{FUELPRICE}
PARAMETER FUELPRICE \index{FUELPRICE}\label{FUELPRICE} contains
fuel prices. Unit: \MONEY/GJ.

\inputdata  Fuels  like wind, water,  sun  or electricity are not
expected to  be given a positive value (and hence if no value is
assigned the default value zero will be used).






\subsection{Parameters on the set product (YYY,MPOLSET,CCC)}
\label{DOC-SS-Parm-YYY-M-POL-CCC}

\subsubsection{M\_POL}
PARAMETER M\_POL \index{M\_POL}\label{M-POL} contains emissions
policy data. Unit: See Section \ref{MPOLSET}.

For each year in the simulation the data is transferred to
internal parameters, see Sections \ref{ITAX-CO2-Y} ff.




\subsection{Parameters on the set product (RRR,SSS,TTT,DF\_QP,DEF)}
\label{DOC-SS-Parm-RRR-SSS-TTT-DF-QP-DEF}

\subsubsection{DEF\_STEPS}

PARAMETER DEF\_STEPS\index{DEF\_STEPS}\label{DEF-STEPS} describes
the elastic electricity demands in relative terms, by quantifying
the steps. Units: (none).

Quantities are specified relative to 1.0. Thus, with e.g. quantity
values of 0.97 and 0.96 for DEF\_D2 and DEF\_D1, respectively,
this means that the first step in decrease of demand  has a
magnitude of 3\% of demand (relative to demand DE) and the second
step has a magnitude of 1\%, adding up to 4\%. See Table
\ref{DOC-T-DEF-STEPS}. Corresponding to these quantity steps the
price steps have to be specified, e.g. as 1.18 and  1.10,
respectively. Thus the 3\% decrease in consumption is the result
of a 10\% increase in price (relative to DEFP\_BASE). Similar
ideas apply to increasing demand and decreasing price.

Observe that the sequence of quantities should be increasing and
the  sequence of prices   should be decreasing.


\begin{table}
\center{ {\small
\begin{tabular}{|l|l|l|l|l|l|l| } \hline
             & DEF\_DINF &  DEF\_D2 & DEF\_D1 & DEF\_U1 &  .. & DEF\_UINF \\
   \hline
 ..DF\_QUANT &  -INF     &    0.96  &   0.97  &   1.02  & .. &  INF  \\
 ..DF\_PRICE &    10     &    1.18  &   1.10  &   0.90  & .. &   -10  \\
\hline
\end{tabular} } %end of small
} % center
\caption{Illustration of TABLE  DEF\_STEPS}
\label{DOC-T-DEF-STEPS}
\end{table}

Do not change the values associated with DEF\_DINF. They are
intended to state that this  step downwards may be infinitely
large, and that the associated price is "very high" (this is
interpreted to be ten times the value entered in DEFP\_BASE,
Section \ref{DEFP-BASE}, which for this reason must have a
reasonable value). Similarly for the values in DEF\_UINF.




\subsection{Parameters on the set product (AAA,SSS,TTT,DF\_QP,DHF)}

\subsubsection{DHF\_STEPS}
 PARAMETER DHF\_STEPS\index{DHF\_STEPS}\label{DHF-STEPS}
describes the elastic heat demands  in relative terms, by
quantifying the steps.  Units: (none).

See the description of the similar construction for electricity
demand, Section \ref{DEF-STEPS}.

For rural areas only the elasticity for the first time segment of
the year (ORD(S)=1 and ORD(T)=1) will be used.


\subsection{Default values}
\label{DOC-SS-Parm-defaultvalues}


If no data is assigned to a parameter or  scalar the default value
zero is automatically assigned. Unless zero incidentally is a
suitable value another method of assigning default values
therefore has to be used.


The following construction may be applied, indicated by an
example. As seen, a
 default\index{default (user specified)} value 0.9 is used and assigned unless where
non-zero values have been assigned in the TABLE:
\begin{itemize}
  \item[]
 TABLE GEFFDERATE(GGG,AAA)  \\
\begin{tabular}{lcccccl}
   &  DK\_E\_Urban  &  LT\_R\_Urban  \\
 CC-Cond1 &  0.94   &  0.99   \\
ST-BP-2-C  &  & 0.87 & ;
\end{tabular}
  \item[] GEFFDERATE(G,A)\$(GEFFDERATE(G,A) EQ 0)=0.9;
\end{itemize}

(Observe that not all elements in a parameter  can be given by
default, at least one must explicitly be given a value (in a TABLE
or by assignment), otherwise it is considered an error.)


\subsection{Restrictions on parameter values.}
\label{DOC-SS-Parm-Exceptions}

To ensure  the basic functioning of the model, some parameter
values must be consistent or attain some
obligatory\index{obligatory value} values:
\begin{itemize}
  \item The pointer GDFUEL from technology type to fuel type FDNB
  must be correct, cf. Section \ref{FDATA}. Hence, if the user
  changes the set of fuels, and therefore also the FDNB to some or
  all of the individual fuels, this must be carefully checked.
 \item For some  parameters undesirable consequences will be encountered for certain values.
   Thus, leaving certain values unspecified, such that the default
   values zero is assigned, will result in a division-by-zero
   error, eg. Sections \ref{IDE-T-Y}, \ref{IDH-T-Y}, \ref{X3FX-VAR-T}.
 \item The values in DEF\_STEPS relative to DEF\_DINF and
 DEF\_UINF, and the values in DHF\_STEPS relative to DHF\_DINF and
 DHF\_UINF are subject to restrictions, see Section
 \ref{DEF-STEPS}  and Section \ref{DHF-STEPS}. Associted with this
 are requirements to DEFP\_BASE and DHFP\_BASE, Sections
 \ref{DEFP-BASE}, \ref{DHFP-BASE}.
  \item   GDATA(IGBPR,'GDCV') and
  GDATA(IGHONLY,'GDCV')  must
  attain unity value (assigned automatically), Section \ref{GDATA}.



\end{itemize}




\subsection{Internal parameters} \label{DOC-SS-Internal-Parm}
\index{internal parameter} \index{parameter}

A number of scalars and parameters have been defined as part of
the model.  In this section we describe those scalars and
parameters, called internal parameters,   for which the values
are derived from other input data, i.e. the user is not supposed
to specify the values of these internal parameters. The names of
these internal parameters all start with the letter I.

Another group of internal parameters is used to hold values for
printing  output, see Section \ref{DOC-S-Interal-Parm-Output}.
These   parameters are not part of the model in the sense that
they influence model results, hence they are referred to as
auxiliary\index{auxiliary} parameters. The names of these
auxiliary parameters all start with the letter O.


\subsubsection{IWEIGHSUMS}


The  internal parameter
IWEIGHSUMS\index{IWEIGHSUMS}\label{IWEIGHSUMS}
  is  used to hold
the total weight of the time of each season  in S, hence it is
calculated from WEIGHT\_S(S), Section \ref{WEIGHT-S},  as
\begin{itemize}
  \item[] IWEIGHSUMS = SUM(S,WEIGHT\_S(S))
\end{itemize}

Observe that the sum must be over set S, not SSS. Unit: (none),
cf. Section \ref{WEIGHT-S}.

\subsubsection{IWEIGHSUMT}

The  internal parameter IWEIGHSUMT
\index{IWEIGHSUMT}\label{IWEIGHSUMT} is used to hold the total
weight of the time of each time period in T, hence it is
calculated from WEIGHT\_T(T), Section \ref{WEIGHT-T}, as
\begin{itemize}
    \item[]  IWEIGHSUMT = SUM(T, WEIGHT\_T(T))
\end{itemize}

Observe that the sum must be  over set T, not TTT. Unit: (none),
cf. Section \ref{WEIGHT-T}.

\subsubsection{ITHISYEAR}

The  internal  scalar  ITHISYEAR
\index{ITHISYEAR}\label{ITHISYEAR} is used as bookkeeping to keep
track of the current simulation year. Its initial value will be
equal to STARTYEAR, Section \ref{STARTYEAR}, and it is incremented
by one for each year in the simulation.



\subsubsection{IHOURSINST}
 \index{IHOURSINST}\label{IHOURSINST}
The internal parameter IHOURSINST(S,T) holds the length of the
time segment (S,T) measured in hours. Unit: (none). It is defined
as
\begin{itemize}
 \item[] IHOURSINST(S,T)=IDAYSIN\_S(S)*IHOURSIN24(T)
\end{itemize}

For advanced use, it may be defined differently, cf. Section
\ref{IST}.



\subsubsection{IDAYSIN\_S}

The time structure within one year of the model is indicated by
the division of the year into seasons SSS and the division of the
day in any season into time periods TTT, cf. Section
\ref{DOC-SS-Time}.

The  internal parameter  IDAYSIN\_S(S)
\index{IDAYSIN\_S}\label{IDAYSIN-S} is defined  from  the weights,
Sections \ref{WEIGHT-S} and \ref{IWEIGHSUMS}, as
\begin{itemize}
  \item[] IDAYSIN\_S(S) = 365*WEIGHT\_S(S) / IWEIGHSUMS
\end{itemize}

It indicates the length of each season given as the  number of
days that are in each season. Unit: (none).  As seen, it is
assumed that there are 365\index{365} days in the year. Observe
that the sum is over the  set S, not SSS. See Section
\ref{IHOURSIN24}.


\subsubsection{IHOURSIN24}


The  internal  parameter IHOURSIN24(T)
 \index{IHOURSIN24}\label{IHOURSIN24} is defined  from  the
weights, Sections \ref{IHOURSIN24} and \ref{IWEIGHSUMT}, as
\begin{itemize}
  \item[]  IHOURSIN24(T) = 24*WEIGHT\_T(T) / IWEIGHSUMT
\end{itemize}

It indicates the length of each period of the season given as the
number of hours that would be  in each time segment, if the length
of the season were 24\index{24} hours. Unit: (none). Observe that
the sum is over the set T, not TTT, similarly to the case in the
following parameters.






\subsubsection{IDE\_SUMST}

The  internal PARAMETER
IDE\_SUMST\index{IDE\_SUMST}\label{IDE-SUMST} holds the annual
amount of electricity demand  as expressed in the units of the
weights and demands used in IDAYSIN\_S, IHOURSIN24 and DE\_VAR\_T,
\begin{itemize}
  \item[]  IDE\_SUMST(IR) = SUM(S, IDAYSIN\_S(S)* \\ SUM(T,
IHOURSIN24(T)*DE\_VAR\_T(IR,S,T)))
\end{itemize}

Unit: (none$\sim$MWh). See also  Section \ref{DE-VAR-T} and
Section \ref{IDE-T-Y}. The use is described in Section
\ref{IDE-T-Y}.

\subsubsection{IDH\_SUMST}
The  internal PARAMETER
IDH\_SUMST\index{IDH\_SUMST}\label{IDH-SUMST} holds the annual
amount of  heat demand   as expressed in the units of the weights
and demands used in IDAYSIN\_S, IHOURSIN24 and DH\_VAR\_T,
\begin{itemize}
  \item[] IDH\_SUMST(IA) = SUM(S, IDAYSIN\_S(S)* \\ SUM(T,
IHOURSIN24(T)*DH\_VAR\_T(IA,S,T)))
\end{itemize}

Unit: (none$\sim$MWh). See also  Section \ref{DE-VAR-T} and
Section \ref{IDE-T-Y}. The use is described  in Section
\ref{IDH-T-Y}.


\subsubsection{IWND\_SUMST}

The  internal PARAMETER  IWND\_SUMST
\index{IWND\_SUMST}\label{IWND-SUMST} holds the annual   amount of
wind generated electricity   as expressed in the units of the
weights and demands used in IDAYSIN\_S, IHOURSIN24 and
WND\_VAR\_T,
\begin{itemize}
  \item[]  IWND\_SUMST(IA) = SUM(S, IDAYSIN\_S(S)* \\ SUM(T,
IHOURSIN24(T)*WND\_VAR\_T(IA,S,T)))
\end{itemize}

Unit: (none$\sim$MWh). See also  Section \ref{DE-VAR-T} and
Section \ref{IDE-T-Y}.


\subsubsection{ISOL\_SUMST }

The  internal PARAMETER ISOL\_SUMST
\index{ISOL\_SUMST}\label{ISOL-SUMST} holds the annual   amount of
solar generated electricity   as expressed in the units of the
weights and demands used in IDAYSIN\_S, IHOURSIN24 and
SOL\_VAR\_T,
\begin{itemize}
  \item[]   ISOL\_SUMST(IA) = SUM(S, IDAYSIN\_S(S)* \\
  SUM(T, IHOURSIN24(T)*SOL\_VAR\_T(IA,S,T)))
\end{itemize}

Unit: (none$\sim$MWh). See also  Section \ref{DE-VAR-T} and
Section \ref{IDE-T-Y}.


\subsubsection{IX3FXSUMST}

The  internal PARAMETER IX3FXSUMST
\index{IX3FXSUMST}\label{IX3FXSUMST} holds the annual   amount of
electricity exported to third countries  as expressed in the
units of the weights and demands used in IDAYSIN\_S, IHOURSIN24
and X3FX\_VAR\_T,
\begin{itemize}
  \item[]   ISOL\_SUMST(IR) = SUM(S, IDAYSIN\_S(S)* \\
  SUM(T, IHOURSIN24(T)*X3FX\_VAR\_T(IR,S,T)))
\end{itemize}

Unit: (none$\sim$MWh). See also  Section \ref{DE-VAR-T} and
Section \ref{IDE-T-Y}.The use is described  in Section
\ref{IX3FX-T-Y}.


\subsubsection{IM\_CO2}

The internal parameter IM\_CO2 \index{IM\_CO2}\label{IM-CO2}
attaches  the CO2 emission coefficient for the fuel to the
technology using that fuel. Unit: kg/GJ. See BALMOREL.GMS.

\subsubsection{IM\_SO2}

The internal parameter IM\_SO2\index{IM\_SO2}\label{IM-SO2}
combines   the SO2 emission coefficient for the fuel with  the
efficiency for technology using that fuel. Unit: kg/GJ. See
BALMOREL.GMS.





\subsubsection{IGKVACCTOY}

The  internal PARAMETER
IGKVACCTOY\index{IGKVACCTOY}\label{IGKVACCTOY} holds the
internally found generation capacity
 at the beginning of the year simulated (i.e., ITHISYEAR). Unit:
 MW.

The value of IGKVACCTOY is equal to the sum of the generation
capacities found endogenously by simulation in the years previous
to ITHISYEAR. Total capacity at the beginning of ITHISYEAR is
equal to (IGKVACCTOY+IGKFX\_Y), see Section \ref{IGKFX-Y}. Total
capacity throughout ITHISYEAR is VGKN.L (where VGKN.L is the level
(value) that VGKN attains) larger than (IGKVACCTOY+IGKFX\_Y).
Compare IXKINI\_Y in Section \ref{IXKINI-Y}.


\subsubsection{IGKFX\_Y}

The  internal PARAMETER IGKFX\_Y\index{IGKFX\_Y}\label{IGKFX-Y}
holds the externally given (parameter GKFX, see Section\ref{GKFX})
generation capacity  at the beginning of the year simulated
(i.e., ITHISYEAR).   Unit: MW.


Total capacity at the beginning of ITHISYEAR and throughout
ITHISYEAR are described in   Section \ref{IGKVACCTOY}.




\subsubsection{IXKINI\_Y}
The  internal PARAMETER IXKINI\_Y
\index{IXKINI\_Y}\label{IXKINI-Y} holds the electrical
transmission  capacity  at the beginning of the year simulated
(i.e., ITHISYEAR). Unit:  MW.

The capacity throughout ITHISYEAR  is VXKN.L  (where VXKN.L is the
level (value) that VXKN attains) larger than IXKINI\_Y. Compare
IGKVACCTOY in  Section \ref{IGKVACCTOY}.




\subsubsection{IFUELP\_Y}

The  internal parameter IFUELP\_Y\index{IFUELP\_Y}\label{IFUELP-Y}
holds the fuel price in the year simulated, transferred from
parameter FUELPRICE, Section \ref{FUELPRICE}. Unit: \MONEY/GJ.




\subsubsection{ITAX\_CO2\_Y}
The  internal PARAMETER  ITAX\_CO2\_Y
\index{ITAX\_CO2\_Y}\label{ITAX-CO2-Y} indicates environmental
policy parameter for a given year and country. Unit: Money/ton.

During simulation the relevant values in MPOLSET, Section
\ref{M-POL}, will be transferred to this internal parameter.




\subsubsection{ITAX\_NOX\_Y}
The  internal  PARAMETER ITAX\_NOX\_Y
\index{ITAX\_NOX\_Y}\label{ITAX-NOX-Y} indicates environmental
policy parameter for a given year  and country. Unit: Money/kg.

During simulation the relevant values in MPOLSET, Section
\ref{M-POL}, will be transferred to this internal parameter.

\subsubsection{ITAX\_SO2\_Y}
The  internal PARAMETER ITAX\_SO2\_Y
\index{ITAX\_SO2\_Y}\label{ITAX-SO2-Y} indicates environmental
policy parameter for a given year  and country.  Unit: Money/ton.

During simulation the relevant values in MPOLSET, Section
\ref{M-POL}, will be transferred to this internal parameter.


\subsubsection{ILIM\_CO2\_Y}
The  internal PARAMETER ILIM\_CO2\_Y
\index{ILIM\_CO2\_Y}\label{ILIM-CO2-Y} indicates environmental
policy parameter for a given year  and country. Unit: ton.

During simulation the relevant values in MPOLSET, Section
\ref{M-POL}, will be transferred to this internal parameter.

\subsubsection{ILIM\_SO2\_Y}
The  internal PARAMETER ILIM\_SO2\_Y
\index{ILIM\_SO2\_Y}\label{ILIM-SO2-Y} indicates environmental
policy parameter, for a given year  and country. Unit: ton.

During simulation the relevant values in MPOLSET, Section
\ref{M-POL}, will be transferred to this internal parameter.

\subsubsection{ILIM\_NOX\_Y}
The  internal PARAMETER ILIM\_NOX\_Y
\index{ILIM\_NOX\_Y}\label{ILIM-NOX-Y} indicates environmental
policy parameter for a given year  and country. Unit: kg.

During simulation the relevant values in MPOLSET, Section
\ref{M-POL}, will be transferred to this internal parameter.




\subsubsection{IDE\_T\_Y}

The internal  PARAMETER IDE\_T\_Y \index{IDE\_T\_Y}\label{IDE-T-Y}
holds the nominal\index{nominal} electricity demand for each time
segment  in the current simulation year. Unit: MW.

It is calculated using the input parameters DE\_VAR\_T and DE and
the internal parameter IDE\_SUMST as
\begin{itemize}
  \item [] IDE\_T\_Y(IR,S,T) = (DE(Y,IR) * DE\_VAR\_T(IR,S,T)) / IDE\_SUMST(IR);
\end{itemize}

See Section \ref{DOC-SS-Demand}.

\inputdata The calculation of   IDE\_T\_Y  involves a division. This will be unproblematic if
 the data entered in DE\_VAR\_T  contain  no negative values and at least one
positive value (which is natural).

Observe that parameters  DE\_VAR\_T, DH\_VAR\_T,  WND\_VAR\_T,
SOL\_VAR\_T,  X3FX\_VAR\_T   will be handled in similar ways as
DE\_VAR\_T (special care should be taken with X3FX\_VAR\_T, see
Section \ref{X3FX-VAR-T}).


\subsubsection{IDH\_T\_Y}

The internal  PARAMETER IDH\_T\_Y \index{IDH\_T\_Y}\label{IDH-T-Y}
holds the nominal\index{nominal} heat demand for each time segment
in the current simulation year. Unit: MW.

It is calculated using the input parameters DH\_VAR\_T and DH and
the internal parameter IDH\_SUMST as
\begin{itemize}
  \item [] IDH\_T\_Y(IA,S,T) = (DH(Y,IA) * DH\_VAR\_T(IH,S,T)) / IDH\_SUMST(IA);
\end{itemize}

See Section \ref{DOC-SS-Demand}.



\subsubsection{IDHFP\_T}
PARAMETER IDHFP\_T \index{IDHFP\_T}\label{DHFP-T}
 holds the price levels of individual steps
 in the electricity demand function, transformed to be comparable with
 generation costs, taxes and distribution costs.
  Unit: \MONEY/MWh.

Observe that the magnitudes of the  quantity measure (MW) of the
corresponding  steps will be specified as  upper bounds on the
variables VDHF\_T, cf. Section \ref{DOC-S-Model}.

\subsubsection{IDEFP\_T}
PARAMETER IDEFP\_T  \index{IDEFP\_T}\label{IDEFP-T} holds the
price levels of individual steps
 in the electricity demand function, transformed to be comparable with
 generation costs, taxes and distribution costs.
  Unit: \MONEY/MWh.


Observe that the magnitudes of the  quantity measure (MW) of the
corresponding steps will be specified as  upper bounds on the
variables VDEF\_T, cf. Section \ref{DOC-S-Model}.






\subsubsection{IX3FX\_T\_Y}
 The internal parameter
IX3FX\_T\_Y\index{IX3FX\_T\_Y}\label{IX3FX-T-Y} holds the export
to third countries for each time segment. It is calculated using
the input parameters X3FX\_VAR\_T and X3FX and the internal
parameter IX3FXSUMST as
\begin{itemize}
  \item [] IX3FX\_T\_Y(IR,S,T)= \\ (X3FX(Y,IR)*X3FX\_VAR\_T(IR,S,T))
   /IX3FXSUMST(IR)
\end{itemize}

Hence the sign of IX3FX\_T\_Y will depend on X3FX(Y,IR),
X3FX\_VAR\_T(IR,S,T)) and IX3FXSUMST(IR).

\inputdata Observe that the calculation will result in an error, if
division by zero is attempted. This puts restrictions on
IX3FXSUMST and in turn in X3FX\_VAR \_T from which it is derived.
- If IX3FXSUMST(IR)  is either strictly positive or strictly
negative for all R this is not a problem. However, it could  make
sense to specify values of X3FX\_VAR \_T such that there is
export in some time segments and import in others, but such that
there a zero net annual export. This situation is not handled in
the model.



\subsubsection{IPENALTY}
 \index{IPENALTY} \label{IPENALTY}
 The internal scalar IPENALTY is the penalty used
in relation to securing feasibility in equations, it enters the
equation QOBJ as coefficient to variables VQ..., cf. Section
\ref{DOC-S-Variables}, Section \ref{IPLUSMINUS}.



\section{Auxiliary parameters for outputs} \label{DOC-S-Interal-Parm-Output}
\index{output}

We have found that it may be convenient to have some internal
parameters to hold various results. For this purpose we have
defined a number of parameters, called auxiliary\index{auxiliary}
parameters. These parameters are only used to hold intermediate
results for printing output, in contrast to those internal
parameters described in Section \ref{DOC-SS-Internal-Parm}   that
are used as proper parts of the model. Such parameters are
located in those  files that are not proper part of the model
(i.e., they are not found in files located in the subdirectory
Model, cf. Section \ref{DOC-S-Files}). The names of such
auxiliary entities start with the letter O.

Error and logging are described in Section
\ref{DOC-SS-BALMORELERRORS}, and model output is described in
Section \ref{DOC-SS-Balmoreloutput}.




%\subsection{Internal output scalars ERRORS}

%SCALAR ERRORS\index{ERRORS}\label{ERRORS} will contain the number
%of errors encountered in the Balmorel error detection, see
%Section \ref{DOC-S-Errors}.






%\section{Data Entry and Control}
%\label{DOC-S-DataEntry}

%Principles. Files. Tables. Default values. Error checking.




\section{Variables}
\label{DOC-S-Variables}\index{VARIABLE}

The following are the main
variables\index{endogenous}\index{VARIABLE} (endogenously
determined values) of the model.

\begin{itemize}
 \item[] VOBJ \index{VOBJ}\label{VOBJ}
      "Objective function value (M\MONEY)"
 \item[]  VGE\_T(AAA,G,S,T) \index{VGE\_T}\label{VGE-T(U,G,S,T)}
     "Electricity generation (MW), existing units"
 \item[]     VGH\_T(AAA,G,S,T) \index{VGH\_T}\label{VGH-T(U,G,S,T)}
  "Heat generation (MW), existing units"
 \item []    VX\_T(IRRRE,IRRRI,S,T) \index{VX\_T}\label{VX-T}
   "Electricity export from region IRRRE to IRRRI (MW)"
 \item []   VGEN\_T(AAA,G,S,T) \index{VGEN\_T}\label{VGEN-T(U,G,S,T)}
    "Electricity generation (MW), new units"
 \item[]     VGHN\_T(AAA,G,S,T) \index{VGHN\_T}\label{VGHN-T(U,G,S,T)}
    "Heat generation (MW), new units"
 \item[]     VGKN(AAA,G) \index{VGKN}\label{VGKN}
   "New generation capacity (MW)"
 \item []    VXKN(IRRRE,IRRRI) \index{VXKN}\label{VXKN}
   "New electricity transmission capacity (MW)"
 \item[]   VDEF\_T(RRR,S,T,DEF\_STEPS) \index{VDEF\_T}\label{VDEF-T}
  "Flexible electricity demands (MW)"
 \item[]    VDHF\_T(AAA,S,T,DHF\_STEPS) \index{VDHF\_T}\label{VDHF-T}
  "Flexible heat demands (MW)"  ;

\item[]  VGHYPMS\_T(AAA,S,T) \index{VGHYPMS\_T}\label{VGHYPMS-T}
  'Contents of pumped hydro storage (MWh)'
\item[]   VHYRS\_S(AAA,S)     \index{VHYRS\_S}\label{VHYRS-S}
   'Hydro energy equivalent at the start of the season (MWh)'
%\item[]   NU_AVERAGE(AAA,G,S)  \index{}\label{}   'Average load of nuclear production in season S (MW)'
\item[]   VESTOLOADT(AAA,S,T)   \index{VESTOLOADT}\label{VESTOLOADT}
 'Loading of electricity storage (MW)'
\item[]   VHSTOLOADT(AAA,S,T)   \index{VHSTOLOADT}\label{VHSTOLOADT}
 'Loading of heat storage (MW)'
\item[]   VESTOVOLT(AAA,S,T)    \index{VESTOVOLT}\label{VESTOVOLT}
 'Electricity storage contents at beginning of time segment (MWh)'
\item[]   VHSTOVOLT(AAA,S,T)   \index{VHSTOVOLT}\label{VHSTOVOLT}
 'Heat storage contents at beginning of time segment (MWh)'


\end{itemize}



In the GAMS language, the restriction  of a variable  to
non-negativity is specified by declaring the variables as
POSITIVE\index{POSITIVE} VARIABLE as the following indicates:
\begin{itemize}
  \item[] FREE     VARIABLE VOBJ;
  \item [] FREE     VARIABLE VGE\_T;
  \item []   POSITIVE VARIABLE VGH\_T;
 \item[] POSITIVE VARIABLE VX\_T;
 \item[] POSITIVE VARIABLE VGHN\_T;
  \item[] POSITIVE VARIABLE VXKN;
  \item[] POSITIVE VARIABLE VGKN;
   \item[]POSITIVE  VARIABLE VDEF\_T;
 \item[] POSITIVE VARIABLE VDHF\_T;
   \item[]POSITIVE  VARIABLE  VHYRS\_S;
   \item[]POSITIVE  VARIABLE   VESTOLOADT;
   \item[]POSITIVE  VARIABLE   VHSTOLOADT;
   \item[]POSITIVE  VARIABLE   VESTOVOLT;
   \item[]POSITIVE  VARIABLE   VHSTOVOLT;
\end{itemize}

Most variables are declared to be positive.  One exception is VOBJ
which cannot be constrained since it expresses the objective
function value, which is unknown. In the GAMS language this is
indicated by the specification FREE\index{FREE} VARIABLE. Also
VGE\_T is unconstrained. The reason is that  VGE\_T also contains
the specification of generation on heat pumps, and this is by
convention assumed to be non-positive. All other generations
covered by VGE\_T are by convention assumed to be non-negative.
The appropriate signs on the components of VGE\_T are enforced as
lower or upper bounds on the individual variables, by assignment
of .LO\index{LO} and/or .UP\index{UP}, respectively. Hence, for
components corresponding to heat pump generation the upper bound
is zero, while  all other components of VGE\_T are given the lower
bound zero. For certain constructions variables may have they
values fixed, this may be done by assignment of .FX\index{FX} (or
implicitly by assigning identical values for .LO and .UP).

Also the units in which the variables are measured are specified
in the list above. There are only two kinds of units, related to
money or power, respectively. The objective function variable VOBJ
is in millions of {\MONEY}    terms (e.g., MEuro or MUSD) while
all others are in millions of power (MW) terms.


In addition to the variables listed above, there are the
following POSITIVE VARIABLES:
\begin{itemize}
\item[]   VQESTOVOLT(AAA,S,T,IPLUSMINUS) \index{VQESTOVOLT}\label{VQESTOVOLT}
 'Feasibility in electricity storage equation'
\item[]   VQHSTOVOLT(AAA,S,T,IPLUSMINUS)  \index{VQHSTOVOLT}\label{VQHSTOVOLT}
 'Feasibility in heat storage equation'
\item[]   VQHYRSSEQ(AAA,S,IPLUSMINUS)  \index{VQHYRSSEQ}\label{VQHYRSSEQ}
 'Feasibility of QHYRSSEQ';
\end{itemize}


All these variables are declared  also on\index{IPLUSMINUS}
IPLUSMINUS, see Section \ref{IPLUSMINUS}, i.e. there is one
corresponding to 'plus' and 'minus'. These sign refer to sign in
front of the variable when it is placed on the right hand side of
the relational operator (=L=, =E= or =G=) in the corresponding
equation. Observe the naming convention with the initial\index{VQ}
'VQ'.

The purpose of these variables is to secure feasibility in the
equations, even unfortunate values are entered for some of the
energy system input. The variables enter the objective function
(specified in QOBJ, Section \ref{DOC-S-Eqs-and-Constr}) with a
large coefficient IPENALTY\index{IPENALTY} and they will therefore
only be positive if there  will not otherwise be a feasible
solution.

To each variable a number of attributes\index{attribute} are
associated, Section \ref{DOC-SS-GAMSoutput}.



\section{Equations and constraints} \label{DOC-S-Eqs-and-Constr}
\index{EQUATION} \index{constraint}\index{EQUATION}

The constraints in the GAMS model are called equations. This
refers to both equality constraints (indicated by =E=) and
inequality constraints (indicated by =L= for 'less than or equal'
and =G= for 'greater than or equal').

The model contains the following equations:
\begin{itemize}
 \item[]    QOBJ  \index{QOBJ} \index{}\label{QOBJ}
                   'Objective function (M\MONEY)'
 \item[]    QEEQ(RRR,S,T)  \index{QEEQ} \label{QEEQ}
         'Electricity generation equals demand'

 \item[]    QHEQURBAN(AAA,S,T)  \index{QHEQURBAN} \label{QHEQURBAN}
      'Heat generation equals demand, urban areas'
\item[]   QHEQRURAL(AAA,S,T)\index{QHEQRURAL} \label{QHEQRURAL}
   'Heat generation equals demand, rural areas'

 \item[]    QGCBGBPR(AAA,G,S,T)  \index{QGCBGBPR} \label{QGCBGBPR}
     'CHP generation (back pressure) limited by Cb-line'
 \item[]    QGCBGEXT(AAA,G,S,T)  \index{QGCBGEXT} \label{QGCBGEXT}
     'CHP generation (extraction) limited by Cb-line'
 \item[]    QGCVGEXT(AAA,G,S,T)  \index{QGCVGEXT} \label{QGCVGEXT}
       'CHP generation (extraction) limited by Cv-line'
 \item[]    QGGETOH(AAA,G,S,T)  \index{QGGETOH} \label{QGGETOH}
      'Electric heat generation'
 \item[]    QGNCBGBPR(AAA,G,S,T)  \index{QGNCBGBPR} \label{QGNCBGBPR}
     'CHP generation (back pressure) Cb-line, new'
 \item[]    QGNCBGEXT(AAA,G,S,T)  \index{QGNCBGEXT} \label{QGNCBGEXT}
    'CHP generation (extraction) Cb-line, new'
 \item[]    QGNCVGEXT(AAA,G,S,T)  \index{QGNCVGEXT} \label{QGNCVGEXT}
       'CHP generation (extraction) Cv-line, new'
 \item[]    QGNGETOH(AAA,G,S,T)  \index{QGNGETOH} \label{QGNGETOH}
     'Electric heat generation, new'
 \item[]    QGEKNT(AAA,G,S,T)   \index{QGEKNT} \label{QGEKNT}
   'Generation on new electricity cap, limited by cap'
 \item[]    QGHKNT(AAA,G,S,T)  \index{QGHKNT} \label{QGHKNT}
     'Generation on new IGHONLY cap, limited by cap'
 % \item[]       QGHKNRURAL(AAA,G,S,T)\index{QGHKNRURAL} \label{QGHKNRURAL}
%    'Generation of heat on new units in rual areas'


% \item[]    QHRURPROP(AAA,G,S,T)\index{QHRURPROP}\label{QHRURPROP}
%    'Proportional generation of heat in rural areas, old'
\item[]    QHNRURPROP(AAA,G,S,T)\index{QHNRURPROP}\label{QHNRURPROP}
    'Proportional generation of heat in rural areas, new'
\item[]     QGKNHYRR(AAA,G,S,T)  \index{QGKNHYRR}\label{QGKNHYRR}
 'Generation on new hydro-ror limited by cap and water'


 \item[]    QGKNWND(RRR,AAA,G,S,T)   \index{QGKNWND} \label{QGKNWND}
   'Generation on new windpower limited by cap and wind'
 \item[]    QGKNSOL(RRR,AAA,G,S,T)  \index{QGKNSOL} \label{QGKNSOL}
   'Generation on new solarpower limited by cap and sun'

 \item[]    QHYRSSEQ(AAA,S)  \index{QHYRSSEQ} \label{QHYRSSEQ}
  'Hydropower with reservoir seasonal energy constraint'           QGKNHYRR
 \item[]    QHYMINRS(AAA,G,S)   \index{QHYMINRS} \label{QHYMINRS}
   'Hydropower reservoir - minimum level'
 \item[]    QHYMAXRS(AAA,G,S)  \index{QHYMAXRS} \label{QHYMAXRS}
   'Hydropower reservoir - maximum level'
 \item[]    QHYMING(AAA,G,S,T)  \index{QHYMING} \label{QHYMING}
  'Hydropower reservoir - minimum generation'
 \item[]    QESTOVOLT(AAA,S,T)  \index{QESTOVOLT} \label{QESTOVOLT}
  'Electricty storage dynamic equation (MWh)'
 \item[]    QESTOLOADT(AAA,S,T) \index{QESTOLOADT} \label{QESTOLOADT}
   'Electricity storage loading less than heat production (MW)'
 \item[]    QHSTOVOLT(AAA,S,T)  \index{QHSTOVOLT} \label{QHSTOVOLT}
   'Heat storage dynamic equation (MWh)'
 \item[]    QHSTOLOADT(AAA,S,T)  \index{QHSTOLOADT} \label{QHSTOLOADT}
   'Heat storage loading less than heat production (MW)'

% \item[]    QGHYRS0(RRR,AAA,G)    \index{QGHYRS0}  \label{QGHYRS0}
%     'Hydropower energy constraint'
%\item[]    QGHYRR???(RRR,AAA,G)    \index{QGHYRR???}  \label{QGHYRR???}
%     'Hydropower ??????????'
 \item[]    QKFUELC(C,FKPOTSETC)  \index{QKFUELC} \label{QKFUELC}
   'Total capacity using fuel FFF is limited in country'
 \item[]    QKFUELR(RRR,FKPOTSETR)  \index{QKFUELR} \label{QKFUELR}
   'Total capacity using fuel FFF is limited in region'
 \item[]    QKFUELA(AAA,FKPOTSETA)  \index{QKFUELA} \label{QKFUELA}
   'Total capacity using fuel FFF is limited in area'
 \item[]    QXK(IRRRE,IRRRI,S,T)    \index{QXK} \label{QXK}
    'Transmission capacity constraint'
 \item[]    QLIMCO2(C)  \index{QLIMCO2} \label{QLIMCO2}
    'Limit on annual CO2-emission'
 \item[]    QLIMSO2(C)   \index{QLIMSO2} \label{QLIMSO2}
     'Limit on annual SO2 emission'
 \item[]    QLIMNOX(C)   \index{QLIMNOX} \label{QLIMNOX}
     'Limit on annual NOx emission'
\end{itemize}

The specification of an EQUATION  consists of a declaration, as
seen above, and a definition which gives the details. The
definition starts with the name of the previously declared
equation followed by ".."\index{.., delimiter with equations} and
then the algebra. In equations the relational operators $\leq$,
$=$ and $\geq$ are specified as =L=, =E= and =G=, respectively.
\index{L, =L=} \index{E, =E=} \index{G, =G=}
\index{=L=}\index{=E=}\index{=G=} See the BALMOREL.GMS file for
the details, and consult the GAMS User's Guide.


The variables will be constrained by the equations, and in
addition by lower and upper bounds on the individual variables as
described in Section \ref{DOC-S-Variables}.

Most of the  equations are expressed in MW. The exceptions are
noted above.

%QOBJ (M\MONEY),  QHYRSSEQ (MWh), QLIMCO2 (ton),  QLIMSO2 (ton) and
%QLIMNOX (kg).








\section{Model and solve}\label{DOC-S-Model}
\index{MODEL}

 In the GAMS language the word MODEL has the specific meaning of a
collection of previously declared EQUATIONS. Hence, it is possible
to  declare more EQUATIONS than what are actually used in a
specific model, and to  specify  several models from previously
 declared  equations.

The specification  of a model is done by stating MODEL followed by
an identifier\index{identifier} (the name of the model), possibly
a short descriptive text and then, between "/" and "/", the
equations to be included in the model. E.g. a very small model
called "TINY", based on Section \ref{DOC-S-Eqs-and-Constr} could
be
\begin{itemize}
  \item []
 MODEL TINY  Only for  this example   / QOBJ, QEEQ, QHEQURBAN   /;
\end{itemize}
It is possible to use ALL\index{ALL} if all the declared
EQUATIONS are to be used. Thus, the "Balmorel" could be specified
as
\begin{itemize}
  \item []
 MODEL   Balmorel Baltic  Model for Regional Energy Liberalisation   /ALL/;
\end{itemize}

Observe that in some of the auxiliary parts the name of the model
is important, cf. Section \ref{DOC-SS-BALMORELERRORS}.


To specify the solution of the model the SOLVE\index{SOLVE}
statement is used, e.g.
\begin{itemize}
  \item []  SOLVE  Balmorel USING\index{USING} LP\index{LP}
 MINIMIZING\index{MINIMIZING}  VOBJ;
\end{itemize}
In this, VOBJ is the variable that holds the objective function
value, cf. Section \ref{DOC-S-Variables}, and it is to be
minimised (the alternative is to specify
MAXIMIZING\index{MAXIMIZING}). The problem class is specified to
be LP (linear programming\index{linear programming}).

Various further options related to the solution process may be
applied. Some may be included in the SOLVE statement, some
specified by using the OPTION\index{OPTION} statement. Useful
options may be specified in relation to  RESLIM,
ITERLIM,\index{RESLIM}\index{ITERLIM} \index{HOLDFIXED} HOLDFIXED.
See the GAMS User's Guide.




\section{Calibration}
\index{calibration} \label{DOC-S-Calibration}

Most of the numerical values in the Balmorel model will be taken
directly from  data sources. Care should be taken to ensure  that
they are consistent. In general, this is not easy, but a
discussion is outside the scope of the present document.

The model contains a few calibration parameters that may be tuned
in an attempt to attain certain consistency between model
simulation results and historically observed values.

 Basically, there are four calibration parameters: GKDERATE,
GEFFDERATE, DEFP\_CALIB  and DHFP\_CALIB. The basic information
about these parameters is given in their respective sections,
\ref{GKDERATE}, \ref{GEFFDERATE}, \ref{DEFP-CALIB} and
\ref{DHFP-CALIB}.



The first two are used to attempt  consistency between fuel
consumption as determined in the model and as specified in energy
statistics, respectively. GKDERATE limits the generation of the
units, and hence the base\index{base load}load units (the high
 merit\index{merit order}
order units) are not generating at full nominal\index{nominal}
capacity all 8760 hours of the year. This implies a certain
generation on other units (the medium\index{medium load} to low
merit order units) that would otherwise not generate. Thus, this
parameter changes the relative amounts of generation between the
units. GEFFDERATE changes the relation between fuel consumption
and electricity and heat generation of the individual units. (It
may also change the relative merit order of the units.)

The last two parameters are used to attempt  attainment of
consistency between the consumption of electricity and heat  as
determined in the model and as specified in energy statistics,
respectively. The point is that the model in principle has two
methods to determine or specify prices of electricity and heat.
One is to determine them through  the marginal values related to
cost components of generation (including fuel and emission taxes),
distribution cost, correction for losses (in the case of
electricity generation, possibly also transmission), and
consumers' taxes on electricity and heat, respectively. The other
way around   the demand functions for electricity and heat,
respectively, are specified directly through nominal values,
profiles and elasticities, cf. Section \ref{DOC-SS-Demand}. See
Figure \ref{DOC-F-CalibrationSupplyDemand}. Most probably, these
two methods will yield prices that are not consistent. The effect
of an inconsistency is that when a historical year is simulated
(e.g., 1995) then the consumption of electricity and heat,
respectively, as found in the model will not be the same as that
observed historically. Moreover, the model permits identification
of marginal values that   differ between  the time segments of
the year, while historically most consumers have seen a price
that has been constant over long time periods.

The parameters DEFP\_CALIB  and DHFP\_CALIB permit a modification
in the  demand functions for electricity and heat in order to get
such consistency.

The suggested sequence of calibration is first GKDERATE then
GEFFDERATE and finally  DEFP\_CALIB and    DHFP\_CALIB.


\begin{figure}
\includegraphics[clip,width=0.9\textwidth]{demand-supply-calib-eps.eps}

 \caption{Illustration of
elements in the calibration of demand functions}
\label{DOC-F-CalibrationSupplyDemand}
\end{figure}

\subsection{Calibration of fuel consumption}
\label{DOC-SS-Calib-fuel}

\subsubsection{Calibration of GKDERATE}
\label{DOC-SS-Calib-GKDERATE}

 The calibration of GKDERATE may be
done with departure in a variety of sources. The overall purpose
is to ensure a reasonable balance between generation on the
various units. Data sources may be any, e.g. statistics over
planned and forced outages of generation units. For thermal
generation units, typical values for forced outages could be in
the range 0.03 to 0.15. Scheduled outage could maybe be 2 to 8
weeks per year. Hence values of GKDERATE could typically be found
in the range 0.7 to 0.95.

For wind power the apparent efficiency and capacity for a group
of turbines will in general be different from that which may be
immediately derived from the individual turbines.  For instance
for an installed capacity of 1000 MW, dispersed over a certain
areas,  the maximal generation will most probably never reach
1000 MW, due to forced and planned  outages,  and due to the fact
that the wind speed is not the same all over the area. Such
phenomena may be reflected in GKDERATE.



The following explains in more detail the reasoning related to
stochastic outages of dispatchable units.



\subsubsection*{Modeling of stochastic outages}


Consider the problem of modelling stochastic
outages\index{outage}\index{stochastic outage} in the electricity
system, more specifically
 %In J{\o}rgensen and Ravn
 % (19..) the expected costs were analysed. Here we consider
the expected generation of the individual units.


Thus, assume $n$ units with capacities $\overline{x}_i$, sorted in
merit order.  Each has a rate of forced outage of $r_i$. Consider
one time period with demand $d$.

We take three models:
\begin{enumerate}
  \item The "true" stochastic model
  \item No consideration of outages
  \item Capacity reduction model, i.e., the capacity $\overline{x}_i$
is substituted by $\overline{x}_i (1- r_i)$, and the problem is
then solved deterministically.
\end{enumerate}

\subsubsection*{Example }

Consider an example where $\overline{x}_i = 100$MW and $r_i = 0.1$
for all $i$.  Let $d=300$MW  and consider a time period of one
hour.

Model 1.

\begin{table}[h]
\centering
\begin{tabular}{|c|c|c|c|c|c|c|c|c|} \hline
Unit 1 & on & on&on&off&on&off&off&off \\
 Unit  2 &on&on&off&on&off&on&off&off \\
  Unit  3 &on&off&on&on&off&off&on&off \\
 \hline
Prob.: & 0.729 & 0.081 & 0.081 & 0.081 &0.009 &0.009& 0.009 &
0.001 \\  \hline
\end{tabular}
\caption{Probability of on-off combinations for the first three
units} \label{Forcedoutage-T1}
\end{table}


With the specified capacities and load the first three units
should always be on. Table  \ref{Forcedoutage-T1} gives the
probabilities for the on-off combinations of these first three
units. As seen, there is a probability of 0.729 that they are all
three on, and consequently a probability of 0.271 that at least
one will be forced out. The expected generation of the units are
then 90MWh, see the Table \ref{Forcedoutage-T2}.

\begin{table}[h]
\centering
\begin{tabular}{|l|c|c|c|c|c|c|} \hline
 Unit: & 1 & 2 & 3 & 4 &5 &6   \\  \hline
 Prob. of attempted  on: & 1.0 & 1.0 & 1.0 & 0.271 &0.0271 &0.00271  \\  \hline
 Prob. of actually  on: & 0.9 & 0.9 & 0.9 & 0.2439 &0.02439 &0.002439  \\  \hline
 Expected prod., MWh:  & 90 & 90 & 90& 24.39 &2.439&0.2439 \\
 \hline
\end{tabular}
\caption{Results for Model 1}  \label{Forcedoutage-T2}
\end{table}


When at least one of the first three units is unit off,  unit 4
will be attempted applied. This happens with a probability of
0.271.  Unit 4 will in these situations produce at 100 MW with
probability 0.9 and hence its expected energy generation will be
0.2439 times 100MWh, i.e., 24.39MWh.

If unit 4 fails when attempted turned on then unit 5 will be
attempted turned on. This attempt will happen with probability
0.0271. With probability 0.9 unit 5 will then turn on, and its
expected energy generation will then be 2.439MWh. Continuation
of  reasoning will lead to the figures given in Table
\ref{Forcedoutage-T2}.


Model 2:

The similar table for model 2 is shown in Table
\ref{Forcedoutage-T3}. Observe, that as the model is not actually
stochastic the terms "probability" and "expected" are somewhat
misleading.

\begin{table}[h]
\centering
\begin{tabular}{|l|c|c|c|c|c|c|} \hline
 Unit: & 1 & 2 & 3 & 4 &5 &6   \\  \hline
 Prob. of attempted  on: & 1.0 & 1.0 & 1.0 & 0.0 &0.0 &0.0  \\  \hline
 Prob. of actually  on: & 1.0 & 1.0 & 1.0 & 0.0 & 0.0&0.0  \\  \hline
 Expected generation, MWh:  & 100 & 100 & 100& 0 &0&0 \\
 \hline
\end{tabular}
\caption{Results for Model 2}  \label{Forcedoutage-T3}
\end{table}



Model 3:

In this model, each of the units has a capacity of 90MW. Hence,
the fourth unit will be applied with a generation of 30MW, and the
figures are given in Table \ref{Forcedoutage-T4}.

\begin{table}[h]
\centering
\begin{tabular}{|l|c|c|c|c|c|c|} \hline
 Unit: & 1 & 2 & 3 & 4 &5 &6   \\  \hline
 Prob. of attempted  on: & 1.0 & 1.0 & 1.0 & 1.0 &0.0 &0.0  \\  \hline
 Prob. of actually  on: & 1.0 & 1.0 & 1.0 & 1.0 &0.0 & 0.0 \\  \hline
 Expected generation, MWh:  & 90 & 90 & 90& 30 &0&0 \\
 \hline
\end{tabular}
\caption{Results for Model 3}  \label{Forcedoutage-T4}
\end{table}

The graph in Figure \ref{Forcedoutage-F}  shows the expected
energy generation for each unit in each of the three models.
Taking Model 1 as the "true"  model it is  seen that Model 2
overestimates the energy generation for the first units (the good
ones), and underestimates it for the last units. Model 3 comes
closer to Model 1. Thus  it  has a correct representation of the
expected energy generation for the first units, while it
overestimates for the next unit and underestimates for the last
units.

For Model 3 this is brought out more clearly in the last graph on
Figure \ref{Forcedoutage-F} which for each unit shows the expected
energy generation of Model 3 in relation to  that of Model 1.




\begin{figure}
\centering
\includegraphics[clip,width=0.9\textwidth]{forcedoutage_eps.eps}
\caption{Expected generation (Models 1, 2 and 3) and comparison of
results for  Model 1 and Model 3 } \label{Forcedoutage-F}
\end{figure}




\subsubsection*{Generalisations}

We may generalise the results as follows. Define indexes $i_1$ and
$i_2$ such that $\sum_{i=1}^{i_1 } \overline{x}_i  \leq d <
\sum_{i=1}^{i_1 +1} \overline{x}_i $
 and
  $\sum_{i=1}^{i_2 -1}  \overline{x}_i (1-r_i) < d \leq \sum_{i=1}^{i_2}
\overline{x}_i (1-r_i)$. As seen, for Model 3 all units with
$i\leq i_1$ will be on with capacity $\overline{x}_i(1-r_i)$, and
all units with $i_2<i$ will be off. We define $e_i^{M1}$ and
$e_i^{M3}$ to be the expected generation of unit $i$ under Model 1
and Model 3, respectively.  We assume for simplicity that
$0<r_i<1$ for all $i$, and give presentation of the results with a
short argumentation.

Units with $i\leq i_1$  will be on all time in Model 3 with power
$\overline{x}_i(1-r_i)$, while in Model 1 they will be attempted
on and therefore have expected power $\overline{x}_i(1-r_i)$.
Hence $e_i^{M1}=e_i^{M3}$ for $i\leq i_1$.

For $i_1 < i < i_2$ the power in Model 3 will be
$\overline{x}_i(1-r_i)$. In relation to Model 1 we see that all
units with $i<i_1$ will be attempted run, and since $1-r_i>0$ they
will in fact be running some of the time. Hence for units with
$i_1 < i \leq i_2$ their expected power under Model 1 will
necessarily be less than $\overline{x}_i(1-r_i)$ (which is the
expected power when attempted on at full capacity), and therefore
$e_i^{M1} \leq e_i^{M3}$ for $i_1 <i< i_2$.

For unit $i_2$, which is the "marginal unit" in Model 3, it may be
shown that   $e_{i_2}^{M1} <  e_{i_2}^{M3}$ (if $ d$ is close to
$\sum_{i=1}^{i_2 -1} \overline{x}_i (1-r_i)  $), $e_{i_2}^{M1} =
e_{i_2}^{M3}$, or $e_{i_2}^{M1}
> e_{i_2}^{M3}$ (if $ d$ is close
to $\sum_{i=1}^{i_2 } \overline{x}_i (1-r_i)  $).



For units $i_2 < i$,  $e_i^{M3}=0$. Further,   $e_i^{M1}>0$ since
with positive probability all units with $i\leq i_2$ will fail
when attempted run.  Therefore $e_i^{M3} <  e_i^{M1}$ for $i_2
<i$.


As seen, Model 3 provides an approximation to Model 1 such that
"base load" units have the correct  expected generations, "first
reserve" units have overestimated generations,  "second reserve"
units have underestimated generations, while for the "marginal
unit" the expected generation may be correct, overestimated or
underestimated.

In terms of total costs it may be  shown that the cost of Model 3
is less than or equal to that of Model 1.



\subsubsection{Calibration of GEFFDERATE}
\label{DOC-SS-Calib-GEFFDERATE}



Once GKDERATE has been given its  values, GEFFDERATE may be
specified. The aim could be that in the calibration year (e.g.,
1995) there is, for the observed generation of electricity and
heat,  consistency between the fuel consumption as determined by
the model and the fuel consumption observed historically.

The deviations between the two measures may be due to  a variety
of reasons. The following description assumes that the need for
calibration  can be meaningfully ascribed to the generation side,
although also e.g.  consumption, subdivision of the year into time
segments, and unrealistic distribution losses could be
responsible.

The reasons for deviations between the two measures may be many,
even  in relation to generation alone.  For thermal units for
instance, there will be a certain fuel consumption related to the
start up of units;   the efficiency for thermal units is not
constant, but depends on the load; a certain additional loss may
also be accredited to up- and down-regulation, relative to the
loss encountered at steady level generation. Hence, the efficiency
depends not only on technical factors, but also on the actual use
of a unit.

For wind power the apparent efficiency and capacity for a group of
turbines will in general be different from that which may be
immediately derived from the individual turbines as discussed
above. Such phenomena may be reflected in GKDERATE, but note that
also for wind power there is inter-dependency with GEFFDERATE and
GDFE, where the latter may be found in various ways. The specific
way chosen to calibrate for wind power will therefore in part
depend on the data sources. Similar considerations hold for hydro
power.

The following description exemplifies the calibration for some
thermal units, assuming that historical data are available at a
regional level. The same value of GEFFDERATE will be assumed for
all units participating in the calibration.




\begin{enumerate}
 \item Define the set of generation technologies for which the
 calibration is to be performed. For instance, if backpressure and extration units will be considered, then
 specify  "SET GEFFSETCAL(G); GEFFSETCAL(G)=NO; GEFFSETCAL(IGEXT)=YES;
   GEFFSETCAL(IGBPR)=YES;".
\item Define the set of fuels for which the
 calibration is to be performed. For instance, "SET FEFFSETCAL(F);
   FEFFSETCAL(F)=NO;   \\  FEFFSETCAL("NGAS")=YES;
   FEFFSETCAL("COAL")=YES;".

  \item Select  calibration year (e.g., 1995), and place the significant
historical values for the calibration year in the relevant
tables; essential are  DE, DH, X3FX (and, if the relations between
the fuel prices have changed significantly, also FUELP). Here,
X3FX should be net electricity export in the calibration year to
all other regions, IRRREspective of these being part of the model
or not. DE should be that part of electricity consumption  that is
covered by the fuels and technologies selected above, multiplied
by (1-DISLOSS\_E(IR)). Similarly for heat demand.
 \item Specify the set Y to contain only the calibration year.
 \item Specify FIRSTYEAR to be the calibration year.
 \item Exclude the possibility of electricity transmission, see
 Section \ref{DOC-SS-Modifications-Notransmission}.
 \item Exclude the possibility of new investments, see
 Section \ref{DOC-SS-Modifications-Nonewinvestments}.
\item  Specify  in  SETS.INC the following sets: "SET DEF\_D(DEF)  /
DEF\_DINF/;",  "SET DEF\_U(DEF)  / DEF\_UINF/;",  " SET
DHF\_D(DHF) / DHF\_DINF /;" and   "SET DHF\_U(DHF)  / DHF\_UINF
/;". This ensures inelastic demand, cf. Section
\ref{DOC-SS-Demand}.

\item For the fuels and technologies selected above let  GEFFDERATE be 1.
\item Make sure that  the files  PRINT1.INC, PRINT2.INC, PRINT3.INC,
and PRINT4.INC are included in the model by a \$INCLUDE, and that
there is a possibility to print out the fuel consumption for the
fuels in question.

\item Run the model.
\item Calculate the quotient between the  fuel consumption as found in the model and the
historically observed fuel consumption.
\item If the values look reasonable (in particular they should be
positive and not too far from 1) then let GEFFDERATE attain the
value of the quotient for the generation technologies in question.
\item (The procedure may at this point be checked as follows:
Run the model again. Now the fuel consumption in the calibration
year should be close to that observed historically.)
\item  Observe that the sets defined in Step 1 and Step 2 need not be the same for all
countries.  Therefore it may be necessary to run the calibration
several times, one for each county, with changes between the sets
specified in Step 1 and Step 2.  Therefore the final table
GEFFDERATE may have to be constructed by combining results from
the individual runs.

\item Delete the sets  constructed in Step 1 and Step 2, and specify
  the tables and sets  that were modified above to have their
desired contents, i.e., bring back the model to original  status.
 \end{enumerate}

Since the values specified in Step  3 may differ between various
runs,  it may be convenient to  keep the data separate. The
auxiliary  files  GEFF\_CAL1.INC\index{GEFF\_CAL1.INC} and
GEFF\_CAL2.INC\index{GEFF\_CAL2.INC} have been prepared for this
purpose, and also in other ways there  are deviations; however,
the idea is the same.  See the instructions in those files.



\subsection{Calibration of demand functions}
\label{DOC-SS-Calib-demand}

\subsubsection{Calibration of   DEFP\_CALIB and DHFP\_CALIB}
\label{DOC-SSS-Calib-DEFP-CALIB-DHFP-CALIB}


 The  calibration of the demand function  is only  necessary if the
demand is elastic. The purpose of the  calibration is to get the
model's demand to coincide with the observed values. This is done
by modifying the base price IDEFP\_T by DEFP\_CALIB  and IDHFP\_T
by DHFP\_CALIB.

Calibration makes sense only if the geographical entity chosen
contains those generation units that are price
setting\index{price setting}\index{marginal unit} (marginal).
Hence, if a region is heavy net exporter or importer of
electricity, calibration can not meaningfully be made  on this
region alone.





In the following the calibration will be explained, assuming that
the auxiliary file DFP\_CALIB.INC\index{DFP\_CALIB.INC}, located
in subdirectory PRINT-INC,  is used.

\begin{enumerate}
\item Select the geographical entity that will be used for
calibration (specify the set C).
  \item Select a calibration year (e.g., 1995), and
place the significant historical values for the calibration year
in the relevant tables; essential are  DE, DH, X3FX and  FUELP.
 %Here, X3FX should be net electricity
 %export in the calibration year to all other regions, irrspective
 %of these being part of the model or not.
 \item Specify the set Y to contain only the calibration year.
 \item Specify FIRSTYEAR to be the calibration year.
% \item Exclude the possibility of electricity transmission, see
% Section \ref{DOC-SS-Modifications-Notransmission}.
  \item Exclude the possibility of new investments, see
 Section \ref{DOC-SS-Modifications-Nonewinvestments}. Exclude the
 possibility of transmission, see Section \ref{DOC-SS-Modifications-Notransmission}.

 \item Make sure that the files  PRINT1.INC, PRINT2.INC, PRINT3.INC,
 and PRINT4.INC are included in the model by a \$INCLUDE.
  \item Let all the values in DEFP\_CALIB  and   DHFP\_CALIB be
  zero (e.g., place the statements "LOOP((IR,S,T),
  DEFP\_CALIB(IR,S,T)=0);"  and "LOOP((IR,S,T),
  DEFP\_CALIB(IR,S,T)=0);" after DEFP\_CALIB and  \\ DHFP\_CALIB,
  respectively).

 \item Include the statement
 "FILE DFP\_CALIB /..$\backslash$OUTPUT $\backslash$DFP\_CALIB.OUT/;"
 in  file PRINT1.INC.
\item Include the statement  "\$INCLUDE "DFP\_CALIB.INC";" in    file PRINT4.INC.

\item  Specify  in  SETS.INC the following sets: "SET DEF\_D(DEF)  /
DEF\_DINF/;",  "SET DEF\_U(DEF)  / DEF\_UINF/;",  " SET
DHF\_D(DHF) / DHF\_DINF /;" and   "SET DHF\_U(DHF)  / DHF\_UINF
/;". This ensures inelastic demand.
\item Run the model.

\item Take the table  DEFP\_CALIB in the file DFP\_CALIB.OUT and
 let it replace the relevant parts of the existing
 table  DEFP\_CALIB.
  Take the table  DHFP\_CALIB in
the file DFP\_CALIB.OUT and  let it replace the existing
 table  DHFP\_CALIB. (Recall to remove
 the "LOOP .." statements if they were introduced above.)
\item Modify the sets DEF\_D(DEF),   DEF\_U(DEF),  DHF\_D(DEF) and   DHF\_U(DEF)
 to include the desired steps in down- and upwards directions.
\item (The procedure may at this point be checked as follows:
Run the model again. Now the consumption of electricity and heat
in the calibration year should be equal to that specified in DE\_Y
and DH\_Y, respectively, i.e. all  up- and down- regulation steps
should be zero. The values in the tables in the file
DFP\_CALIB.OUT should now all be zero.)
\item Specify  the tables that were modified in Step 1 to have their
desired contents, specify possibilities of investment and
transmission as desired (Step 5), delete or comment out the
statements specified in Step 8 and Step 9, i.e., bring back the
model to original status.


\end{enumerate}



















%\section{Dynamics}
%\index{dynamics} \label{DOC-S-Dynamics} \index{current simulation year}\index{simulation year}

\section{Output}
\index{output} \label{DOC-S-Output}

There are two types of output, that generated automatically by the
GAMS system, and that generated by auxiliary parts of the Balmorel
model.





\subsection{Automatically generated GAMS output}
\label{DOC-SS-GAMSoutput}

GAMS automatically generates  two files after each run, the
Balmorel.lst file and the Balmorel.log  file.
 \index{LST file}\index{LST}\index{LOG file}

The lst file contains a summary of model statistics and solution,
including\index{STATUS} SOLVER STATUS, MODEL STATUS,
OBJECTIVE\index{OBJECTIVE VALUE} VALUE. See the GAMS User's Guide
for further information.

The lst file also contains an echo of the input with line numbers
associated, usefull for identification. Errors detected by GAMS
will be identified, see Section \ref{DOC-SS-GAMSERRORS}.

The lst file may also contain  specific output if wanted by the
user. Details may be controlled by various compiler directives
 \index{compiler directive}
in the form of  OPTIONS\index{OPTION} and \$ON / \$OFF statements.
For variables this will typically be their values.

Some options are OFFLISTING,  OFFSYMXREF,  OFFSYMLIST, OFFUELLIST
OFFUELXREF, with the alternatives OFFLISTING, ONSYMXREF,
ONSYMLIST, ONUELLIST, ONUELXREF
  \index{LISTING}\index{SYMXREF}\index{SYMLIST}
  \index{UELLIST}\index{UELXREF}
are used to control printing of listing and cross references.
ONINLINE makes it possible to comment out parts between /* and */.
LIMROW\index{LIMROW}\index{LIMCOL} and LIMCOL specify the maximum
number of rows and columns  used in equations listing and
inspection of details.  SYSOUT\index{SYSOUT} controls the printing
of the solved status in the list file. SOLPRINT\index{SOLPRINT}
controls the printing of the solution in the list file.


To each EQUATION in a solved model are associated a
level\index{level} and a marginal value\index{marginal value},
that may be referenced for printing, for instance for QOBJ as
QOBJ.L\index{L, level} and QOBJ.M\index{M, MARGINAL},
respectively. See the GAMS User's Guide for further information.

Similarly a number (six) of attributes\index{attribute} are
specified in relation to a variable, viz., the lower bound
(.LO),\index{LO}) the upper bound (.UP),\index{UP}) a fixed value
(.FX),\index{FX}) a level (.L)\index{L}), a marginal or dual value
(.M),\index{M, MARGINAL}) a scale value (.SCALE)\index{SCALE}) and
a branching priority value. (.PRIOR)\index{PRIOR})



\subsection{Balmorel auxiliary output}
\label{DOC-SS-Balmoreloutput}

A number of auxiliary  files have been designed to facilitate
output from the Balmorel model. Four of these files are taken
into the Balmorel model by the \$INCLUDE statement, viz.,
PRINT1.INC, PRINT2.INC, PRINT3.INC, PRINT4.INC.\index{PRINT.INC}
See Section \ref{DOC-S-Files} for location.  Prints are generated
from these, or from other files that are in turn included in the
model by a \$INCLUDE statement in one of the  mentioned files. See
those files for further details.


Error checking  is also reported and a  log maintained,   see
Section \ref{DOC-SS-BALMORELERRORS}.

Observe that quite a number of auxiliary include files for
printing output are available. The user's computer may not be able
to handle so many open files, therefore carefully specify those
files to be produced in PRINT4.INC and comment out the remaining
ones.


It is important to note that the information printed by auxiliary
parts is not reliable if errors occurred during the execution of
the  GAMS program!




A spreadsheet\index{spreadsheet}  environment is provided for
assistance in processing the output.





\section{Errors and Log}
\label{DOC-S-Errors}

We distinguish between errors that are detected automatically by
the GAMS system and errors that are detected by auxiliary parts
of the Balmorel model.

\subsection{Errors automatically detected by GAMS}
\label{DOC-SS-GAMSERRORS}

 If there are errors detected automatically  by
GAMS they will in the Balmorel.lst file be marked by four stars,
hence a convenient way to locate them is to search for the
string\index{****} "****". If there are errors, then at the end of
the Balmorel.lst file there will be a list of errors and a
description of the possible cause of each error.  User errors are
indicated by the statement "**** USER ERROR(S) ENCOUNTERED".
Other error types are marked by e.g.
 "**** EXECUTION ERROR",
 "**** MATRIX ERROR",  or
 "**** PUT ERROR".
  See the GAMS User's Guide for further information.




\subsection{Errors detected by Balmorel auxiliary parts}
\label{DOC-SS-BALMORELERRORS}\index{errors}

A number of error checks have been specified in the files
ERROR1.INC, ERROR2.INC, ERROR3.INC.\index{ERROR.INC} See Section
\ref{DOC-S-Files} for location.



These files are included in the BALMOREL.GMS file by the \$INCLUDE
statement. If any of these files are included (i.e., they are not
all commented out) the file PRINT1.INC must also be included
(Section \ref{DOC-SS-Balmoreloutput}).

The error checking mainly concerns the numerical  values of the
input. The error checking tries to detect if the values specified
are "reasonable". For instance, fuel efficiency would for most
generation  units be expected to take  a value between, say, 0.3
and 0.9. On the other hand, it might be that values outside this
range were relevant for some applications. In order to catch  as
many errors as possible, the range should be as small as possible,
but in order not to indicate an error where there is none, the
range should be large. Thus, a balance has to be achieved.

If an error (which, as just argued, need not be an "error") is
detected, a specification  is written to the file
ERRORS.OUT.\index{ERRORS.OUT} To see the exact reason for the
identification of the error, see the appropriate ERROR*.INC file.
The number of errors encountered is held in the parameter ERRORS.
In any case, a summary is printed in ERRORS.OUT and also in the
file LOG.OUT.

(In a few cases the error check is not reported to ERROR.OUT but
rather results in a deliberate computation error. In this case
follow the instructions given.)


The LOG.INC file\index{log},\index{LOG.INC}  see Section
\ref{DOC-S-Files} for location,  will contain a summary of the
automatically generated  log and lst files, see Section
\ref{DOC-SS-GAMSoutput}, and a summary of the auxiliary error
checking. The LOG.INC file is included in the model by a
\$INCLUDE statement in the file BALMOREL.GMS. If the LOG.INC file
is included (i.e., not commented out) the file PRINT1.INC
(Section \ref{DOC-SS-Balmoreloutput}) must be included in the
BALMOREL.GMS file by the \$INCLUDE statement. The output
generated by the LOG.INC will be placed in the file
LOG.OUT.\index{LOG.OUT}

Finally observe that the information  in the auxiliary files
described in this Section is not valid if there are user error(s)
encountered, cf. Section \ref{DOC-SS-GAMSERRORS}.


\subsection{Sequence of log and error observations}
\label{DOC-SS-SEQLOGERROR}\index{errors}

The very first step is to observe if the attempts to interpret the
input and generate the model were successful. Therefore inspect
the Balmorel.lst file to see if it contains the statement "****
USER ERROR(S) ENCOUNTERED",  if this is the case then this should
be fixed, cf. Section \ref{DOC-SS-GAMSERRORS}. The information in
the output files described in Section \ref{DOC-SS-BALMORELERRORS}
is in this case not valid.


After simulation, the user must first observe if the attempt was
successful. Again, errors will be documented in the Balmorel.lst
file, following "****". This for instance could be
 "**** EXECUTION ERROR". Also during   printout errors may occur, indicated e.g.
 by    "**** PUT ERROR".
 See Section \ref{DOC-SS-GAMSERRORS}.



If the execution of GAMS was successful, further information may
be acquired by using the facilities provided in the auxiliary
parts  described in Section \ref{DOC-SS-BALMORELERRORS}. This is
intended to be more expedient, however, it can of course not be
guaranteed that the auxiliary parts will be free of errors! If
there are errors here, then using the GAMS standard output is the
way to detect them. Moreover, some errors are detrimental to the
intended functioning of the auxiliary parts, and therefore the
information in the auxiliary parts is only reliable if there were
no errors  detected by the GAMS system. Also observe that if for
instance no solution was found (which is also some sort of
 \index{normal completion}
normal completion!) the information in the auxiliary parts may be
unreliable.


If the auxiliary parts are used, the following procedure should
be followed. First check the LOG.OUT file, see Section
\ref{DOC-SS-BALMORELERRORS}. For a successful simulation  there
should be a declaration that there were no errors detected in the
input, and that the solution efforts were successful. (Remember
to check the date and time, because if there were errors detected
by GAMS, the LOG.OUT file may not be updated.)

If errors were detected in the input, then see the output file
from the error detection, cf. Section \ref{DOC-SS-BALMORELERRORS}.

If the solution is not successful then study the error messages
in the LOG.OUT and in the Balmorel.lst files.
 See the GAMS User's Guide for further information.

And finally: there are of course  many modeling errors that
neither GAMS nor the auxiliary parts can detect, but only the
user.


\section{Model variants}
\label{DOC-S-Modifications}

The above description refers to the "standard" version of the
Balmorel model. In the following, a number of obvious
modifications will be described.

\subsection{Why variants}

In any modeling work choices are made as to what to include in the
model and what not. Many objectives are balanced in this process.
Therefore, a model that may be suited for one purpose may be less
appropriate for another. Moreover, if every possible application,
and therefore the most detailed level of representation of the
energy sector was attempted, the model would not be appropriate
to any application.

To obtain maximum flexibility,  the Balmorel model is coded in a
high level modeling language (GAMS) and the code itself is
available to any user. The user therefore has complete control
over the model and therefore also over modifications. This
permits a wider range of potential applications.

In the following a number of obvious modifications will be
described.


\subsection{Types of changes}



Obviously, some modifications are easy while others are more
complicated. The following classification  may be suggested, where
the first modifications are very simple, and the last one more
complicated.

\begin{itemize}
  \item Limit the scope\index{scope} of the model, while maintaining basic
structure. Thus for instance with respect to geography, the model
represents  a number countries, as given by the set CCC. It is
elementary to delete some of the countries from the model by
declaring the set C to be a proper subset of CCC. It is  not much
more complicated to reduce the number of regions or areas within
a country, although some consistency is required, see Section
\ref{DOC-SS-Geography}. Reduction of the number of  years
simulated is elementary and reduction of the number of time
segments is discussed in Section \ref{DOC-SSS-Timewithinyears}.
Reduction in the number generation technologies is elementary,
Section \ref{DOC-SS-Technologies}. Reduction in the number of
fuels is described in Section \ref{DOC-SS-Fuels}.

  \item Change the  values of the data entered.  It is elementary to
change the values of input parameters. Observe that the auxiliary
parts involve some checking for "reasonable" values, see Section
\ref{DOC-SS-BALMORELERRORS}, and if therefore unanticipated values
are entered, error messages may occur; in this case, the user is
advised to revise the data and error checking.

  \item Enlarge the model with elements very similar to those that are
already there. New countries, with their associated regions and
areas may easily be introduced. The model contains a number of
energy transformation technologies  in the set GGG   and more -
provided they are similar to one of the existing technology types,
see Section \ref{DOC-SSS-Generation-technology-types}  - may be
added by copying the ideas in the representations already there
and then filling  in  the required parameter values. Additional
fuel types  may be introduced in the set FFF (and the appropriate
pointers introduced in GDFUEL, see Section
\ref{DOC-SS-Technologies}). Additional years may be introduced
into YYY without difficulties, and the number of time segments may
also be increased, see Section \ref{DOC-SSS-Timewithinyears} and
Section \ref{DOC-SS-Modifications-Finersubdivisionoftheyear}.

%\item Enlarge the model with elements less similar to those that are
%already there. For instance, it may be desired to limit the
%emission of dust.

 \item Change the model structure. The model structure
consists of the parameters and variables in the model and the
relations between them (see more specifically Section
\ref{DOC-S-Overview}). On this issue it is not possible to
specify  the efforts involved as they depend heavily on the
specific requirements. However, really many modifications to the
structure can be made by an effort which is considerable smaller
than that of acquiring the associated data.
\end{itemize}

The first items have explicitly or implicitly been covered in the
preceding parts. In the sequel the last item is therefore
addressed by way of examples. Section
\ref{DOC-SS-Modifications-Organising} briefly describes aspects of
the organisation of modifications.






\subsection{Organising modifications}
\label{DOC-SS-Modifications-Organising}

Obviously there must be some internal consistency between the data
in the model. Part of this is obtained by maintaining the proper
sequence\index{sequence of statements} of the statements (cf.
Section \ref{DOC-SS-GAMS-intro}).


Observe that there are in this respect the following parts of the
model (excluding auxiliary\index{auxiliary} files):
\begin{itemize}
  \item The first part of the BALMOREL.GMS file,
  down to but excluding the first \$INCLUDE statement
   (not counting the auxiliary files)
  \item The SETS.INC file
  \item The declaration of internal sets in the BALMOREL.GMS file
  \item The other *.inc files (not counting the auxiliary files)
  \item The following part of the BALMOREL.GMS file
\end{itemize}



During execution, the model is composed by including the include
files  into the BALMOREL.GMS file. For this to work properly, the
SETS.INC file must be included  before any of the other include
files.

However,  the other include files  may come in arbitrary order,
since they only contain numerical values,  contained in tables (or
otherwise) directly specified  by the user.  (Possibly there may
immediately after a table be an assignment modifying or
complementing the values given in the TABLE (e.g. to user
specified default values, see Section
\ref{DOC-SS-Parm-defaultvalues}); in such cases the assignments
should always follow immediately after the TABLE.)  Hence, any
table is independent of any other table, and the sequence does
not matter.

All the internal parameters are derived after all the include
files have been included (not counting the auxiliary files).


For this reason modifications in input data (i.e., what is
contained in the *.inc files, not counting the auxiliary files)
may be given quite flexibly.  New static sets should be placed in
the file SETS.INC.  New tables should be placed in the
appropriate include files (cf. Section \ref{DOC-S-Files}),  or
they (and any new sets) may be placed in a new file that is taken
into the BALMOREL.GMS file by an \$INCLUDE statement; this
statement should be placed after the \$INCLUDE SETS.INC and
before the handling of the internal sets and parameters (e.g., it
may be placed immediately after the \$INCLUDE SETS.INC statement).

Changes in the model structure are exemplified in the sequel.




Any modifications made should be in a form that is consistent
with the other parts of the Balmorel model. Here is a checklist:
\begin{itemize}
\item Did you observe the conventions on notation? See Section
\ref{DOC-SS-Naming-conventions} and see 'naming' in the index.
\item Did you observe the conventions on location of sets, parameters,
etc., Section \ref{DOC-S-Files}?
\item Did you remember to give a description of the sets, parameters
introduced (including specification of the units or measurement
wherever relevant, Section \ref{DOC-S-Parms-and-Scalars})?
\item Did you observe the restriction on the use on GAMS version,
Section \ref{DOC-SS-GAMS-intro}? (You may check this by including
a "\$use225"\index{\$use225} at the top of the Balmorel.gms file
(with the \$ in the first column).  Unfortunately, this seems not
to enforce the restriction of identifiers and labels to have a
maximum of ten characters.)

%\item Did you
\end{itemize}






\subsection{No new investments}
\label{DOC-SS-Modifications-Nonewinvestments}

The model is intended for analysis of long term development and
therefore contains as an integrated feature the possibility to
expand generation and electrical transmission capacities
according of economic criteria.

User specified  expansion and
de-commissioning\index{commissioning} of generation capacity is
specified in the parameter  GKFX, Section \ref{GKFX}.

Automatically generated expansion of generation capacity will be
undertaken for those technologies for which a 1 is specified for
GDKVARIABL in GDATA. If investment in new generation capacity
should not be allowed this may be implemented by setting a 0 for
GDKVARIABL  in GDATA.

Another  way, that will  not delete the information contained in
GDKVARIABL in GDATA is the following.  Include   the line
\begin{itemize}
  \item [] VGKN.FX(IA,G) = 0;
\end{itemize}
in the BALMOREL.GMS annual updating section\index{annual updating
section}.



If investment in new electrical transmission  capacity should not
be allowed this may be implemented as follows. Include  the line
\begin{itemize}
  \item [] VXKN.FX(IRE,IRI) = 0;
\end{itemize}
 in the BALMOREL.GMS annual updating section.



\subsection{No electricity transmission}
\label{DOC-SS-Modifications-Notransmission}

Electricity transmission between regions  may be excluded by
including the following statement
\begin{itemize}
  \item []  VX\_T.FX(IRE,IRI,S,T) = 0;
\end{itemize}
in the BALMOREL.GMS annual updating section.

Observe that this will not influence  transmission  to third
regions. Also observe that the above modification will prevent
transmission within a country that has more than one region. To
permit such transmission anyway, the above statement should be
refined to the following form:

\begin{itemize}
  \item []
LOOP(C,VX\_T.FX(IRE,IRI,S,T)\$   (NOT(CCCRRR(C,IRE)  AND
CCCRRR(C,IRI)))=0);
\end{itemize}
[This feature is not properly tested.]




\subsection{Reserve generation capacity}
[This feature is not properly tested.]

It is common to model an electricity system with a constraint
expressing that a certain generation capacity must be available.
One version of this may be achieved as follows. It is assumed that
the requirement is that in every region the available capacity in
each year must exceed the maximum over (S,T) of
nominal\index{nominal} electricity demand DE by a certain
percentage.


\begin{enumerate}
  \item  Make a SET GKRESSET(G) holding those generation
  technologies that may count as providing reserve. E.g., it could
  be identical to the already defined set IGDISPATCH(G).
 \item Declare a PARAMETER GKRES(RRR) to  hold the desired capacities.
\item Assign the appropriate values to GKRES.
Thus, if e.g. 20\% reserve capacity is desired in all regions,
 the  following statement may be made:

  GKRES(IR)= DE(Y,IR)* \index{SMAX}
(SMAX((S,T),DE\_VAR\_T(IR,S,T))

   /(IDE\_SUMST(IR)))* 1.2;
  \item  Declare an EQUATION QGKRES(RRR).
  \item Define the EQUATION as e.g.

     QGKRES(IR)..

   SUM(IA\$(RRRAAA(IR,IA)),  SUM(GKRESSET,

      IGKFX\_Y(GKRESSET,IA) + GKVACCCO\_Y(GKRESSET,IA)

             + VGKN(GKRESSET,IA)       ))
    =G=
    GKRES(IR);

(An alternative version could multiply the capacities by
GKDERATE.)

\item  Include QGKRES in the list of equations specifying the
MODEL, see Section \ref{DOC-S-Model}.
\end{enumerate}





\subsection{Finer subdivision of the year}
\label{DOC-SS-Modifications-Finersubdivisionoftheyear}

The subdivision of time within the year is given by the sets SSS,
S, TTT and T. A number of parameters depend on this subdivision,
see Section \ref{DOC-SS-Parm-AAA-SSS-TTT} and Section
\ref{DOC-SS-Parm-RRR-SSS-TTT}.  All these parameters are found in
the file VAR.INC, cf. Section \ref{DOC-S-Files}.

From a given subdivision it is easy to aggregate such that only
one time segment per year is used, cf. Section
\ref{DOC-SSS-Timewithinyears}.

If another subdivision is desired (with more than one segment per
year), then the mentioned sets and all the parameter values in
Section   \ref{DOC-SS-Parm-AAA-SSS-TTT} and Section
\ref{DOC-SS-Parm-RRR-SSS-TTT} must be changed accordingly.

As a consequence of such revised subdivision it should be
expected that also the calibration must be repeated, cf. Section
\ref{DOC-S-Calibration}.



\subsection{Making more parameters depend on the year}
\label{DOC-SS-Modifications-Moreparametersdependontheyear}

Some  of the parameters have been made dependent on the year,
e.g., demands (parameters DE and DH) and fuel prices (parameter
FUELPRICE). It might for certain analyses  be desirable to have
more parameters depending on the year, e.g. distribution costs and
losses, generating units' efficiencies, etc.

This may   be done by copying the ideas from those parameters
that presently do depend on the year, e.g. DE (see Section
\ref{IDE-T-Y}).



\subsection{Making more parameters depend on geography}
\label{DOC-SS-Modifications-Moreparametersdependongeography}

Description to be entered.


\subsection{Multifuel generation technologies}
\label{DOC-SS-Multifuel}

Description to be entered.




\subsection{More detail on renewable energies}
\label{DOC-SS-Modifications-Moredetailonrenewableenergies}

Description to be entered.


\subsection{Market power}

Description to be entered.

\subsection{Markets for emission,  renewable energy and JI}

See "Co-existence of electricity, TEP and TGC markets in the
Baltic Sea Region" at the Balmorel home page.


\subsection{Investment costs depend on technical life time}
[This feature is not properly tested.]


Investment decisions related to generation technology depend among
other things on the product of the investments cost, given in
GDATA(G,'GDINVCOST'), and ANNUITYC. This choice of representation
is motivated in part by the expectation that the economical life
time of the generation units is not longer than the technical life
time, and therefore the technical life time need not be
represented. However, in some studies this may be inappropriate.
The following describes the introduction of this dependency.

\begin{enumerate}
\item Declare in the file GEOGR.INC the PARAMETER
ANNUITYGC(GGG,CCC) and  enter the appropriate values in a TABLE,
e.g. as
 \begin{itemize}
  \item[]
 TABLE ANNUITYGC(GGG,CCC)  \\
\begin{tabular}{lcccccl}
               & Denmark  &  Estonia & Finland &   \\
 CC-Con00-G    & 0.1627   &  0.1993  & 0.1627  &  \\
 GHydro-res    & INF      &  0.0806  & 0.0651  & ;
\end{tabular}
  \item[] ANNUITYGC(G,C)\$(ANNUITYGC(G,C) EQ 0)=0.1315;
 \end{itemize}
(As seen, also a default value of 0.1315 has been entered
(corresponding to a 10\% interest rate with 15 years of economic
life time, cf. Table \ref{DOC-T-Annuities} and Section
\ref{ANNUITYC}).)

\item In the file BALMOREL.GMS all instances of ANNUITYC(C) should be replaced
by ANNUITYGC(G,C), except where it is multiplied by XINVCOST.

\end{enumerate}

In the above modification the investment on transmission capacity
will be evaluated using the values in ANNUITYC.

If the modification desired is not directed towards
differentiation of the life time within the set of  generation
technologies but rather towards the differentiation between life
times for (all) generation technologies and transmission, then the
following modification  could be used.

\begin{enumerate}
\item Declare in the file GEOGR.INC the PARAMETER
ANNUITYXC(CCC) and enter the appropriate values.
\item In the file BALMOREL.GMS all instances of ANNUITYC(C) that are multiplied by XINVCOST
 should be replaced by ANNUITYXC(C).
\end{enumerate}

It is possible to introduce the two types of modifications
simultaneously.


\subsection{Energy constraints}

Description to be entered.





\subsection{Price dependent electricity exchange with third countries}

Description to be entered.











\subsection{Version numbering and referencing}
\label{DOC-SS-Version-numbering}
 The Balmorel model exists in various
versions.\index{version} We shall here clarify the naming of
these.

A distinction will be made between model structure and data (cf.
also Sections \ref{DOC-SS-Structure-Model-Simulation} and
\ref{DOC-S-Overview}). By model structure\index{model structure}
is mainly  meant identifiers\index{identifier} (i.e., the names)
of the SCALARS, PARAMETERS, VARIABLES, SETS and EQUATIONS in the
model, plus additional information like limitations on variables
(declarations as POSITIVE or FREE, specifications of bounds (.UP,
.LO, or .FX)),\index{UP}\index{LO}\index{FX} and the structure
related to the dynamics\index{dynamics}.
% (Section \ref{DOC-S-Dynamics})
 By the data\index{data} is meant the
actual   labels\index{label} (i.e., members) in the SETS and the
actual numerical values assigned to SCALARS and PARAMETERS. Thus,
the present document deals with model structure and not with data.

The  identification of different versions should  distinguish
between model structure versions and data versions.

If therefore an analysis is performed where the model used
consists of e.g. a model structure called Balmorel version 2.17,
modified to exclude transmission, and using data that mostly
consisted of data called Balmorel version 2.10,  this may be
referred to as " ...  the analysis used the Balmorel model
structure version 2.17, modified to exclude transmission. The data
used in the analysis was based on Balmorel data set version 2.10,
modified as follows: .... ".     \newpage






%\section{Conclusions}


\section{Overview of model structure components}
\label{DOC-S-Overview}

The model structure consists of the sets, parameters and variables
in the model and the relations between them, as described in the
preceding pages. Here an overview of the components will be given.

The table identifies in alphabetical order  all sets, obligatory
set members,
  %(page \pageref{obligatorysetmember})\index{obligatory set member},
scalars, parameters, variables,  and equations in the Balmorel
model, with specification of the units in which they are given
(where relevant, see also page
\pageref{DOC-page-unitsdiscussion}), in which file the component
is declared and at which page in this document the component is
described.

Another way to  get an overview over the model components is the
use the compiler directives
 \index{compiler directive} \index{SYMXREF}\index{SYMLIST}
 \index{UELLIST}\index{UELXREF}
\$ONSYMXREF, \$ONSYMLIST, \$ONUELLIST and \$ONUELXREF (the \$ in
the first position of the line)  which produce maps in the LST
file.


Not identified in the table are the following aspects:
\begin{itemize}
  \item Upper bounds and lower bounds on variables
  \item The internal working of the linking between technology and
  fuel use, cf. Section \ref{FDATA}.
  \item The sequence of the statements. A particular case of this
  is the annual updating parts (linking the individual years).
  \item The constants 8760, 365, 24 and  3.6.
  \end{itemize}

\vspace{2cm}














%\tiny
{\footnotesize
%{\scriptsize
\begin{center}
\begin{tabular}{|l|l|l|l|l|c|} \hline
Name           &    Domain               & Type            & Unit              & Defined in       & Page  \\
 \hline
 1995, 1996,.. & -                       & obl. set member & (none)            & SETS.INC          &  \pageref{YEARS} \\
 AAA           & -                       & set             & -                 & SETS.INC          &  \pageref{AAA} \\
 AAARURH       & -                       & set             & -                 & SETS.INC          &  \pageref{AAARURH} \\
 AAAURBH       & -                       & set             & -                 & SETS.INC          &  \pageref{AAAURBH} \\
 AGKN          &                         & set             & -                 & SETS.INC          &  \pageref{AGKN} \\
 ANNUITYC      & (CCC)                   & parameter       & (none)            & GEOGR.INC         &  \pageref{ANNUITYC} \\
 RRRAAA        & -                       & set             & -                 & SETS.INC          &  \pageref{RRRAAA} \\
 C             & (CCC)                   & set             & -                 & SETS.INC          &  \pageref{C} \\
 CCC           & -                       & set             & -                 & SETS.INC          &  \pageref{CCC} \\
 DE            & (YYY,RRR)               & parameter       & MWh               & DE.INC            &  \pageref{DE} \\
 DE\_VAR\_T    & (RRR,SSS,TTT)           & parameter       & (none$\sim$MW)    & VAR.INC           &  \pageref{DE-VAR-T} \\
 DEF           & -                       & set             & -                 & SETS.INC          &  \pageref{DEF} \\
 DEF\_D        & -                       & set             & -                 & SETS.INC          &  \pageref{DEF-D} \\
 DEF\_DINF     & -                       & obl. set member & (none), Money/MWh & GEOGR.INC         &  \pageref{DEF-DINF} \\
 DEF\_STEPS    & (RRR,SSS,TTT,..         &                 &                   &                   &    \\
               & ..,DF\_QP,DEF)          & parameter       & (none), Money/MWh & GEOGR.INC         &  \pageref{DEF-STEPS} \\

 DEF\_U        & -                       & set             & -                 & SETS.INC          &  \pageref{DEF-U} \\
 DEF\_UINF     & -                       & obl. set member & (none), Money/MWh & GEOGR.INC         &  \pageref{DEF-UINF} \\
 DEFP\_BASE    & (RRR)                   & parameter       & Money/MWh         & GEOGR.INC         &  \pageref{DEFP-BASE} \\
 DEFP\_CALIB   & (RRR,SSS,TTT)           & parameter       & Money/MWh         & GEOGR.INC         &  \pageref{DEFP-CALIB} \\
 DF\_QP        & -                       & set             & -                 & SETS.INC          &  \pageref{DF-QP} \\
 DF\_PRICE     & -                       & obl. set member & Money/MWh         & GEOGR.INC         &  \pageref{DF-PRICE} \\
 DF\_QUANT     & -                       & obl. set member & MW                & GEOGR.INC         &  \pageref{DF-QUANT} \\
 DH            & (YYY,AAA)               & parameter       & MWh               & DH.INC            &  \pageref{DH} \\
 DH\_VAR\_T    & (AAA,SSS,TTT)           & parameter       & (none$\sim$MW)    & VAR.INC           &  \pageref{DH-VAR-T} \\
 DHF           & -                       & set             & -                 & SETS.INC          &  \pageref{DHF} \\
 DHF\_STEPS    & (AAA,SSS,TTT,..         &                 &                   &                   &    \\
               &      ..,DF\_QP,DEF)     & parameter       & (none), Money/MWh & GEOGR.INC         &  \pageref{DHF-STEPS} \\
 DHF\_D        & -                       & set             & -                 & SETS.INC          &  \pageref{DHF-D} \\
 DHF\_DINF     & -                       & obl. set member & (none), Money/MWh & GEOGR.INC         &  \pageref{DHF-DINF} \\
 DHF\_U        & -                       & set             & -                 & SETS.INC          &  \pageref{DHF-U} \\
 DHF\_UINF     & -                       & obl. set member & (none), Money/MWh & GEOGR.INC         &  \pageref{DHF-UINF} \\
 DHFP\_BASE    & (AAA)                   & parameter       & Money/MWh         & GEOGR.INC         &  \pageref{DHFP-BASE} \\
 DHFP\_CALIB   & (AAA,SSS,TTT)           & parameter       & Money/MWh         & GEOGR.INC         &  \pageref{DHFP-CALIB} \\

\hline
\end{tabular}
\end{center}

\newpage

\begin{center}
\begin{tabular}{|l|l|l|l|l|c|} \hline
Name           &   Domain                & Type            & Unit              & Defined in       & Page  \\
 \hline
DISCOST\_E     & (RRR)                   & parameter       &   Money/MWh       & GEOGR.INC         &  \pageref{DISCOST-E} \\
 DISCOST\_H    & (AAA)                   & parameter       &   Money/MWh       & GEOGR.INC         &  \pageref{DISCOST-H} \\
 DISLOSS\_E    & (RRR)                   & parameter       & (none)            & GEOGR.INC         &  \pageref{DISLOSS-E} \\
 DISLOSS\_H    & (AAA)                   & parameter       & (none)            & GEOGR.INC         &  \pageref{DISLOSS-H} \\
 FFF           & -                       &  set            & -                 & SETS.INC          &  \pageref{FFF} \\
 FDCO2          & -                       & obl. set member & kg/GJ             & FUELS.INC         &  \pageref{FDCO2} \\
 FDATASET      & -                       & set             & -                 & SETS.INC          &  \pageref{FDATASET} \\
 FDATA         & (FFF,FDATASET)          & set             & -                 & SETS.INC          &  \pageref{FDATA} \\
 FDNB           & -                       & obl. set member & (none)            & FUELS.INC         &  \pageref{FDNB} \\
 FKPOTA        & (FKPOTSETA,AAA)         & parameter       & MW                & GEOGR.INC         &  \pageref{FKPOTA} \\
 FKPOTC        & (FKPOTSETC,CCC)         & parameter       & MW                & GEOGR.INC         &  \pageref{FKPOTC} \\
 FKPOTR        & (FKPOTSETR,RRR)         & parameter       & MW                & GEOGR.INC         &  \pageref{FKPOTR} \\
 FKPOTSETA     & (FFF)                   & set             & -                 & SETS.INC          &  \pageref{FKPOTSETA} \\
 FKPOTSETC     & (FFF)                   & set             & -                 & SETS.INC          &  \pageref{FKPOTSETC} \\
 FKPOTSETR     & (FFF)                   & set             & -                 & SETS.INC          &  \pageref{FKPOTSETR} \\
 FDSO2          & -                       & obl. set member & kg/GJ             & FUELS.INC         &  \pageref{FDSO2} \\
 FUELPRICE     & (YYY,AAA,FFF)           & parameter       & Money/GJ          & FUELP.INC         &  \pageref{FUELPRICE} \\
 G             & (GGG)                   & set             & -                 & SETS.INC          &  \pageref{G} \\
 %G\_RESCAP    & -                       & parameter       & ?                 & ?                 &  \pageref{G-RESCAP} \\
 %G\_RESCAP\_Y & -                       & parameter       &                   &                   &    \\
 GDCB           & -                      & obl. set member & -                 & SETS.INC          &  \pageref{GDCB} \\
 GDCV           & -                      & obl. set member & -                 & SETS.INC          &  \pageref{GDCV} \\
 GDATASET      & -                       & set             & -                 & SETS.INC          &  \pageref{GDATASET} \\
 GDESO2        & -                       & obl. set member & -                 & SETS.INC          &  \pageref{GDESO2} \\
 GEFFDERATE    & (GGG,AAA)               & parameter       & (none)            & GEOGR.INC         &  \pageref{GEFFDERATE} \\
 GDATA         & (GGG,GDATASET)          & parameter       & -                 & SETS.INC          &  \pageref{GDATA} \\
 GDFE           & -                      & obl. set member & -                 & SETS.INC          &  \pageref{GDFE} \\
 GDFROMYEAR     & -                      & obl. set member & -                 & SETS.INC          &  \pageref{GDFROMYEAR} \\
 GDFUEL         & -                      & obl. set member & -                 & SETS.INC          &  \pageref{GDFUEL} \\
 GGG           & -                       & set             & -                 & SETS.INC          &  \pageref{GGG} \\
 GDINVCOST0     & -                      & obl. set member & -                 & SETS.INC          &  \pageref{GDINVCOST0} \\
 GDINVCOST      & (GGG,AAA)              & parameter       & MMoney/MWh        & GEOGR.INC         &  \pageref{GDINVCOST} \\
 GKDERATE      & (GGG,AAA,SSS)           & parameter       & (none)            &  GEOGR.INC        &  \pageref{GKDERATE} \\
 GKFX          & (YYY,AAA,GGG)           & parameter       & MW                & GKFX.INC          &  \pageref{GKFX} \\
 GKINI         & (GGG,AAA)               & parameter       & MW                & BALMOREL.GMS      &  \pageref{GKINI} \\
 GDKVARIABL    & -                       & obl. set member & -                 & SETS.INC          &  \pageref{GDKVARIABL} \\
 GDNOX          & -                      & obl. set member & -                 & SETS.INC          &  \pageref{GDNOX} \\
 GDOMFCOST0    & -                       & obl. set member & -                 & SETS.INC          & \pageref{GDOMFCOST0} \\
 GDOMVCOST0    & -                       & obl. set member & -                 & SETS.INC          &  \pageref{GDOMVCOST0} \\
 GDOMVCOST     & (GGG,AAA)               & parameter       & Money/MWh         & GEOGR.INC         &  \pageref{GDOMVCOST} \\
 GDOMFCOST     & (GGG,AAA)               & parameter       & MMoney/MW         & GEOGR.INC         &  \pageref{GDOMFCOST} \\
 GDAUXIL        & -                      & obl. set member & -                 & SETS.INC          &  \pageref{GDAUXIL} \\
 GDTYPE         & -                      & obl. set member & -                 & SETS.INC          &  \pageref{GDTYPE} \\
 HYPPROFILS    & (AAA,SSS)               & parameter       & Money/MWh         & VAR.INC           & \pageref{HYPPROFILS}\\
 IA            & (AAA)                   & int. set        & -                 & BALMOREL.GMS      &  \pageref{IA} \\
 ICA           & (XYZ)                   & int. set        & -                 & BALMOREL.GMS      &  \pageref{ICA} \\
 IARURH        & (AAA)                   &int. set         & -                 & BALMOREL.GMS      &  \pageref{IARURH} \\
 IAURBH        & (AAA)                   &int. set         & -                 & BALMOREL.GMS      &  \pageref{IAURBH} \\
 IDAYSIN\_S    & (S)                     &int. scalar      & (none)            & BALMOREL.GMS      &  \pageref{IDAYSIN-S} \\
 IDE\_SUMST    & (RRR)                   & int.  parameter & (none$\sim$MWh)   & BALMOREL.GMS      &  \pageref{IDE-SUMST} \\
 IDE\_T\_Y     & -                       &int.  parameter  & MW                & BALMOREL.GMS      &  \pageref{IDE-T-Y} \\
 IDEFP\_T      & -                       &int.  parameter  & Money/MWh         & BALMOREL.GMS      &  \pageref{IDEFP-T} \\
 IDH\_SUMST    & (AAA)                   &int.  parameter  & (none$\sim$MWh)   & BALMOREL.GMS      &  \pageref{IDH-SUMST} \\
 IDH\_T\_Y     & -                       &int.  parameter  & MW                & BALMOREL.GMS      &  \pageref{IDH-T-Y} \\
 IDHFP\_T      & -                       &int.  parameter  & Money/MWh         & BALMOREL.GMS      &  \pageref{DHFP-T} \\
\hline
\end{tabular}
\end{center}


\newpage
\begin{center}
\begin{tabular}{|l|l|l|l|l|c|} \hline
Name           & Domain                  & Type            & Unit              & Defined in       & Page  \\
 \hline

 IFUELP\_Y     & (AAA,FFF)               & int. parameter  & Money/GJ          & BALMOREL.GMS      &  \pageref{IFUELP-Y} \\
 IGBPR         & (G)                     &int.  set        & -                 & BALMOREL.GMS      &  \pageref{IGBPR} \\
 IGCND         & (G)                     &int.  set        & -                 & BALMOREL.GMS      &  \pageref{IGCND} \\
 IGDISPATCH    & (G)                     & int. set        & -                 & BALMOREL.GMS      &  \pageref{IGDISPATCH} \\
 IGETOH        & (G)                     & int. set        & -                 & BALMOREL.GMS      &  \pageref{IGETOH} \\
% IGEONLY      & -                       & int. set        & -                 & BALMOREL.GMS      &  \pageref{IGEONLY} \\
 IGEOREH       & (G)                     &int. set         & -                 & BALMOREL.GMS      &  \pageref{IGEOREH} \\
 IGEXT         & (G)                     & int. set        & -                 & BALMOREL.GMS      &  \pageref{IGEXT} \\
 IGHOB         & (G)                     & int. set        & -                 & BALMOREL.GMS      &  \pageref{IGHOB} \\
 IGHORHERUR    & (G)                     &int.  set        &-                  & BALMOREL.GMS      &  \pageref{IGHORHERUR} \\
 IGHYRS        & (G)                     & int. set        & -                 & BALMOREL.GMS      &  \pageref{IGHYRS} \\
 IGKFX\_Y      & (GGG,AAA)               & int. parameter  & MW                & BALMOREL.GMS      &  \pageref{IGKFX-Y}  \\
 IGKE          & (G)                     &int.  set        &-                  & BALMOREL.GMS      &  \pageref{IGKE} \\
 IGKH          & (G)                     &int.  set        &-                  & BALMOREL.GMS      &  \pageref{IGKH} \\
 IGKVACCTOY    & (G,AAA)                 &int.  parameter  & MW                & BALMOREL.GMS      &  \pageref{IGKVACCTOY}  \\
 IGNOTETOH     & (G)                     &int.  set        &-                  & BALMOREL.GMS      &  \pageref{IGNOTETOH} \\
 IGSOL         & (G)                     &int.  set        &-                  & BALMOREL.GMS      &  \pageref{IGSOL} \\
 IGWND         & (G)                     &int.  set        &-                  & BALMOREL.GMS      &  \pageref{IGWND} \\
 IHOURSIN24    & (T)                     &int.  parameter  & (none)            & BALMOREL.GMS      &  \pageref{IHOURSIN24} \\
 IHOURSINST    & (S,T)                   &int.  parameter  & (none)            & BALMOREL.GMS      &  \pageref{IHOURSINST} \\
 ILIM\_CO2\_Y  & (C)                     &int.  parameter  & t                 & BALMOREL.GMS      &  \pageref{ILIM-CO2-Y} \\
 ILIM\_NOX\_Y  & (C)                     &int.  parameter  & kg                & BALMOREL.GMS      &  \pageref{ILIM-NOX-Y} \\
 ILIM\_SO2\_Y  & (C)                     & int. parameter  & t                 & BALMOREL.GMS      &  \pageref{ILIM-SO2-Y} \\
 IM\_CO2       & (G)                     &int.  parameter  & kg/GJ             & BALMOREL.GMS      &  \pageref{IM-CO2} \\
 IM\_SO2       & (G)                     &int.  parameter  & kg/GJ             & BALMOREL.GMS      &  \pageref{IM-SO2} \\
 IPLUSMINUS    & -                       & int. set        & -                 & BALMOREL.GMS      & \pageref{IPLUSMINUS} \\
 IPENALTY      & -                       & int. scalar     & -                 & BALMOREL.GMS      & \pageref{IPENALTY} \\
 IR            & (RRR)                   & int. set        & -                 & BALMOREL.GMS      &  \pageref{IR} \\
 ISALIAS       & (S)                     & int. set        &-                  & BALMOREL.GMS      &   \pageref{ISALIAS} \\
 ISOL\_SUMST   & (AAA)                   &int.  parameter  & (none$\sim$MWh)   & BALMOREL.GMS      &  \pageref{ISOL-SUMST} \\
 IST           & (S,T)                   & int. set        &-                  & BALMOREL.GMS      &   \pageref{IST} \\
 ISTS          & (S)                     & int. set        &-                  & BALMOREL.GMS      &   \pageref{ISTS} \\
 ISTT          & (T)                     & int. set        &-                  & BALMOREL.GMS      &   \pageref{ISTT} \\
 ITALIAS       & (T)                     & int. set        &-                  & BALMOREL.GMS      &  \pageref{ITALIAS} \\
 ITAX\_CO2\_Y  & (YYY,CCC)               &int.  parameter  & Money/t           & BALMOREL.GMS      &  \pageref{ITAX-CO2-Y} \\
 ITAX\_NOX\_Y  & (YYY,CCC)               &int.  parameter  & Money/kg          & BALMOREL.GMS      &  \pageref{ITAX-NOX-Y} \\
 ITAX\_SO2\_Y  & (YYY,CCC)               &int.  parameter  & Money/t           & BALMOREL.GMS      &  \pageref{ITAX-SO2-Y} \\
 ITHISYEAR     & -                       &int. scalar      & (none)            & BALMOREL.GMS      &  \pageref{ITHISYEAR} \\
 IWEIGHSUMS    & (S)                     &int.  parameter  & (none)            & BALMOREL.GMS      &  \pageref{IWEIGHSUMS} \\
 IWEIGHSUMT    & (T)                     &int.  parameter  & (none)            & BALMOREL.GMS      &  \pageref{IWEIGHSUMT} \\
 IWND\_SUMST   & (AAA)                   & int. parameter  & (none$\sim$MWh)   & BALMOREL.GMS      &  \pageref{IWND-SUMST} \\
 IXKINI\_Y     & (IRRRC,IRRRE)           &int.  parameter  & MW                & BALMOREL.GMS      &  \pageref{IXKINI-Y} \\
 IX3FX\_T\_Y   & (RRR,S,T)               & int.  parameter & MW                & BALMOREL.GMS      &  \pageref{IX3FX-T-Y}  \\
 IX3FXSUMST    & (RRR)                   & int.  parameter & (none$\sim$MWh)   & BALMOREL.GMS      &  \pageref{IX3FXSUMST}  \\
 LIM\_CO2      & (YYY,CCC)               & obl. set member & t                 & MPOL.INC          &  \pageref{LIM-CO2} \\
 LIM\_NOX      & (YYY,CCC)               & obl. set member & kg                &  MPOL.INC         &  \pageref{LIM-NOX} \\
 LIM\_SO2      & (YYY,CCC)               & obl. set member & t                 &  MPOL.INC         &  \pageref{LIM-SO2} \\
 MPOLSET       & -                       &set              & -                 & SETS.INC          &  \pageref{MPOLSET} \\
   QEEQ        & (RRR,S,T)               & equation        & MW                & BALMOREL.GMS      &  \pageref{QEEQ} \\
   QESTOVOLT        & (AAA,S,T)               & equation        & MW                & BALMOREL.GMS      &  \pageref{QESTOVOLT} \\
   QESTOLOADT       & (AAA,S,T)               & equation        & MW                & BALMOREL.GMS      &  \pageref{QESTOLOADT} \\
   QGCBGBPR    & (AAA,G,S,T)             & equation        & MW                & BALMOREL.GMS      &  \pageref{QGCBGBPR} \\
   QGCBGEXT    & (AAA,G,S,T)             & equation        & MW                & BALMOREL.GMS      &  \pageref{QGCBGEXT} \\
   QGCVGEXT    & (AAA,G,S,T)             & equation        & MW                & BALMOREL.GMS      &  \pageref{QGCVGEXT} \\
   QGGETOH     &  (AAA,G,S,T)                       & equation        & MW                & BALMOREL.GMS      &  \pageref{QGGETOH} \\
   QGNCBGBPR   & (AAA,G,S,T)             & equation        & MW                & BALMOREL.GMS      &  \pageref{QGNCBGBPR} \\
   QGNCBGEXT   & (AAA,G,S,T)             & equation        & MW                & BALMOREL.GMS      &  \pageref{QGNCBGEXT} \\
   QGNCVGEXT   & (AAA,G,S,T)             & equation        & MW                & BALMOREL.GMS      &  \pageref{QGNCVGEXT} \\
   QGNGETOH    & (AAA,G,S,T)             & equation        & MW                & BALMOREL.GMS      &  \pageref{QGNGETOH} \\
   QGEKNT      & (AAA,G,S,T)             & equation        & MW                & BALMOREL.GMS      &  \pageref{QGEKNT} \\
   QGHKNT      & (AAA,G,S,T)             & equation        & MW                & BALMOREL.GMS      &  \pageref{QGHKNT} \\
   QGKNWND     & (RRR,AAA,G,S,T)         & equation        & MW                & BALMOREL.GMS      &  \pageref{QGKNWND} \\
   QGKNSOL     & (RRR,AAA,G,S,T)         & equation        & MW                & BALMOREL.GMS      &  \pageref{QGKNSOL} \\
   QGKNHYRR    &  (AAA,G,S,T)            & equation        & MMoney           & BALMOREL.GMS      &  \pageref{QGKNHYRR} \\
   QHSTOVOLT   & (AAA,S,T)               & equation        & MW                & BALMOREL.GMS      &  \pageref{QHSTOVOLT} \\
   QHSTOLOADT  & (AAA,S,T)               & equation        & MW                & BALMOREL.GMS      &  \pageref{QHSTOLOADT} \\
   QHYRSSEQ    & (AAA,S)                 & equation        & MMoney           & BALMOREL.GMS      &  \pageref{QHYRSSEQ} \\

 \hline
\end{tabular}
\end{center}
\newpage

\begin{center}
\begin{tabular}{|l|l|l|l|l|c|} \hline
Name           &  Domain                 & Type            & Unit              & Defined in       & Page  \\
 \hline
   QHEQURBAN   & (AAA,S,T)               & equation        & MW                & BALMOREL.GMS      &  \pageref{QHEQURBAN} \\
   QHEQRURAL   & (AAA,S,T)               & equation        & MW                & BALMOREL.GMS      &  \pageref{QHEQRURAL} \\
%   QHRURPROP   &                         & equation        & MW                & BALMOREL.GMS      &  \pageref{QHRURPROP} \\
   QHNRURPROP  &  (AAA,G,S,T)            & equation        & MW                & BALMOREL.GMS      &  \pageref{QHNRURPROP} \\
   QKFUELC     & (C,FKPOTSETC)           & equation        & MW                & BALMOREL.GMS      &  \pageref{QKFUELC} \\
   QKFUELR     & (RRR,FKPOTSETR)         & equation        & MW                & BALMOREL.GMS      &  \pageref{QKFUELR} \\
   QKFUELA     & (AAA,FKPOTSETA)         & equation        & MW                & BALMOREL.GMS      &  \pageref{QKFUELA} \\
   QLIMCO2     & (C)                     & equation        & ton               & BALMOREL.GMS      &  \pageref{QLIMCO2} \\
   QLIMSO2     & (C)                     & equation        & ton               & BALMOREL.GMS      &  \pageref{QLIMSO2} \\
   QLIMNOX     & (C)                     & equation        & kg                & BALMOREL.GMS      &  \pageref{QLIMNOX} \\
   QOBJ        &                         & equation        & MMoney           & BALMOREL.GMS      &  \pageref{QOBJ} \\
   QXK         & (IRRRE,IRRRI,S,T)       & equation        & MW                & BALMOREL.GMS      &  \pageref{QXK} \\
 CCCRRR        & -                       & set             & -                 & SETS.INC          &  \pageref{CCCRRR} \\
 RRR           & -                       & set             & -                 & SETS.INC          &  \pageref{RRR} \\
 AAARURH       & (AAA)                   & set             & -                 & SETS.INC          &  \pageref{AAARURH} \\
 S             & (SSS)                   & set             & -                 & SETS.INC          &  \pageref{S} \\
 SOL\_VAR\_T   & (AAA,SSS,TTT)           & parameter       & (none$\sim$MW)    & VAR.INC           &  \pageref{SOL-VAR-T} \\
 SSS           & -                       & set             & -                 & SETS.INC          &  \pageref{SSS} \\
 STARTYEAR     & -                       & scalar          & (none)            & BALMOREL.GMS      &  \pageref{STARTYEAR} \\

 T             & (TTT)                   & set             & -                 & SETS.INC          &  \pageref{T} \\
 TAX\_CO2      & -                       & obl. set member & Money/t           &  MPOL.INC         &  \pageref{TAX-CO2} \\
 TAX\_DE        & (CCC)                   & parameter       & Money/MWh         & GEOGR.INC         &  \pageref{TAX-DE} \\
 TAX\_F        & (FFF,CCC)               & parameter       & Money/MWh         & GEOGR.INC         &  \pageref{TAX-F} \\
 TAX\_DH        & (CCC)                   & parameter       & Money/MWh         & GEOGR.INC         &  \pageref{TAX-DH} \\
 TAX\_NOX      & -                       & obl. set member & Money/kg          &  MPOL.INC         &  \pageref{TAX-NOX} \\
 TAX\_SO2      & -                       & obl. set member & Money/t           &  MPOL.INC         &  \pageref{TAX-SO2} \\
 TTT           & -                       & set             & -                 & SETS.INC          &  \pageref{TTT} \\
 Y             & -                       & set             & -                 & SETS.INC          &  \pageref{YYY} \\
 YVALUE        & (YYY)                   & parameter       & (none)            & SETS.INC          &  \pageref{YVALUE} \\
 YYY           & -                       & set             & -                 & SETS.INC          &  \pageref{YYY} \\
 VDEF\_T       & (RRR,S,T,DET\_STEPS)    & variable        & MW                & BALMOREL.GMS      &  \pageref{VDEF-T} \\
 VDHF\_T       & (AAA,S,T,DHFSTEPS)      & variable        & MW                & BALMOREL.GMS      &  \pageref{VDHF-T} \\
 VESTOLOADT    & (AAA,S,T)               & variable        & MW                & BALMOREL.GMS      &  \pageref{VESTOLOADT} \\
 VESTOVOLT     & (AAA,S,T)               & variable        & MWh               & BALMOREL.GMS      &  \pageref{VESTOVOLT} \\
 VGKN          & (AAA,G)                 & variable        & MW                & BALMOREL.GMS      &  \pageref{VGKN} \\
 VGE\_T        & (AAA,G,S,T)             & variable        & MW                & BALMOREL.GMS      &  \pageref{VGE-T(U,G,S,T)} \\
 VGEN\_T       & (AAA,G,S,T)             &  variable       & MW                & BALMOREL.GMS      &  \pageref{VGEN-T(U,G,S,T)} \\
 VGH\_T        & (AAA,G,S,T)             &  variable       & MW                & BALMOREL.GMS      &  \pageref{VGH-T(U,G,S,T)} \\
 VGHN\_T       & (AAA,G,S,T)             &  variable       & MW                & BALMOREL.GMS      &  \pageref{VGHN-T(U,G,S,T)} \\
 VHSTOLOADT    & (AAA,S,T)               &  variable       & MW                & BALMOREL.GMS      &  \pageref{VHSTOLOADT} \\
 VHSTOVOLT     & (AAA,S,T)               &  variable       & MWh               & BALMOREL.GMS      &  \pageref{VHSTOVOLT} \\
 VOBJ          &                         &  variable       & MMoney            & BALMOREL.GMS      &  \pageref{VOBJ} \\
 VQESTOVOLT    & (AAA,S,T,IPLUSMINUS)    &  variable       & MWh               & BALMOREL.GMS      &  \pageref{VQESTOVOLT} \\
 VQHSTOVOLT    & (AAA,S,T,IPLUSMINUS)    &  variable       & MWh               & BALMOREL.GMS      &  \pageref{VQHSTOVOLT} \\
 VQHYRSSEQ     & (AAA,S)                 &  variable       & MWh               & BALMOREL.GMS      &  \pageref{VQHYRSSEQ} \\
 VXKN          & (IRRRE,IRRRI)           & variable        & MW                & BALMOREL.GMS      &  \pageref{VXKN} \\
 VX\_T         & (IRRRE,IRRRI,S,T)       &  variable       & MW                & BALMOREL.GMS      &  \pageref{VX-T} \\
 WEIGHT\_S     & (SSS)                   & parameter       & (none)            & VAR.INC           &  \pageref{WEIGHT-S} \\
 WEIGHT\_T     & (TTT)                   & parameter       & (none)            & VAR.INC           &  \pageref{WEIGHT-T} \\
 WND\_VAR\_T   & (AAA,SSS,TTT)           & parameter       & (none$\sim$MW)    & VAR.INC           &  \pageref{WND-VAR-T} \\
 WTRRSVARS     & (AAA,SSS)               & parameter       & (none$\sim$MW)    & VAR.INC           &  \pageref{WTRRSVARS} \\
 WTRRRVAR\_T   & (AAA,SSS,TTT)           & parameter       & (none$\sim$MW)    & VAR.INC           &  \pageref{WTRRRVAR-T} \\
 X3FX          & (YYY,RRR)               &  parameter      & MWh               & X3FX.INC          &  \pageref{X3FX} \\
 X3FX\_VAR\_T  & (RRR,SSS,TTT)           &  parameter      & (none$\sim$MW)    & VAR.INC           &  \pageref{X3FX-VAR-T} \\
 XCOST         & (IRRRE,IRRRI)           & parameter       & Money/MWh         & TRANS.INC         &  \pageref{XCOST} \\
 XINVCOST      & (IRRRE,IRRRI)           & parameter       & Money/MWh         & TRANS.INC         &  \pageref{XINVCOST} \\
 XKINI         & (IRRRE,IRRRI)           & parameter       & MW                & TRANS.INC         &  \pageref{XKINI} \\
 XLOSS         & (IRRRE,IRRRI)           &  parameter      & (none)            & TRANS.INC         &  \pageref{XLOSS} \\
 Y             & (YYY)                   & set             & -                 & SETS.INC          &  \pageref{Y} \\
 YEARINC       & -                       & scalar          & (none)            & SETS.INC          &  \pageref{YEARINC} \\

\hline
\end{tabular}

\end{center}
}


\newpage
 \index{OFF: see next letter}
 \index{ON: see next letter}
 \index{ON-/OFF-: see next letter}
\input{bms210a.ind}






\end{document}
















The technical characteristics of generation technologies are given
in the table GDATA. In addition to this information, there is a
need to specify how the generation units are used in the model.

There are three major aspects of this, viz.,
\begin{itemize}
  \item Is the capacity of the generation technology specified
  exogenously, or found endogeneouly during the simulation?
  \item Can the technology for which the capacity  is found endogeneouly
  during the simulation be used in urban areas, rural areas or
  both?
      \item  Is the generation in the individual time segment specified
  exogenously, or found endogeneouly by optimal dispatch during the simulation?
\end{itemize}


As concerns possibilities of exogeneuously specifying generation
capacities and generation amounts, there are the following.

The set GGG of generation technologies is divided into two subsets
TGCAPFIXD(GGG) and TGCAPFREE(GGG), holding those technologies that
have their capacities specified exogeneuously and  endogenously,
respectively. The two subsets need not be mutually exclusive, and
their sum need not constitute GGG.

By specification of P as a proper subset of GGG ... ?


As concerns the dispatch of generation among the units with
capacities that are available at a particular time, their are two
classes, those that have fixed generation and those that can
adjust their generation according to economic criteria, i.e.,
engage in optimal  dispatch.

Some types of technologies automatically belong to the first
type. These are wind power, solar power and hydro run-of-river.
For other types of technologies  belonging to the first type this
much be explicitly declared. This is done by including these
technologies in the set TGPRODFIXD(TG).

The sets TGPRODFIXD(GGG) and TGPRODFREE(GGG) hold the technologies
that have fixed and free dispatch, respectively. The two subsets
must be mutually exclusive, but their sum need not constitute GGG.
(Alternativt, hvis summen er lig med GGG: only TGPRODFIXD(GGG) is
specified, and technologies not declared members of
TGPRODFIXD(GGG) are automatically considered dispatchable.)



An example of a generation technology that may be in TGCAPFIXD
and  TGPRODFIXD could be an inceneration plant.  An example of a
generation technology that may be in TGCAPFREE and  TGPRODFIXD
could be a local chp plant in a rural area?. An example of a
generation technology that may be in TGCAPFIXD and  TGPRODFREE
could be an existing condensing unit. An example of a generation
technology that may be in TGCAPFREE and TGPRODFREE could be a new
chp or combined cycle unit considered for investment.


The generation technologies are further subdivided into subsets
IARURH and IAURBH ......





Introduction of a new technology:


Upon introduction of a new technology into the model, the
following should be done.


In the  file SETS.INC the name of the unit is included  in the
set TG. (And, upon simulation, the set may be included in the set
G, if relevant.) Further, the technology is included  in the set
TGCAPFIXD and/or TGCAPFREE, and in at most one of the sets
TGPRODFIXD and TGPRODFREE.

In the file TECH.INC the data of the unit is entered in  TABLE
GDATA.

If the unit is in set TGCAPFIXD then in the  file PCAPFX\_Y.INC
the capacity must be specified for each year.

If the unit is in set TGPRODFIXD then in the  file VAR.INC the
.........












: [not completed]
\begin{itemize}
  \item   IGCND(G):
  \item IGBPR(G): equations  G\_CMBP,  GN\_CMBP,
  and the bounds
  VGE\_T.LO(A,IGBPR,S,T) = 0; VGEN\_T.LO(A,IGBPR,S,T) = 0;
  VGE\_T.UP(A,IGBPR,S,T) = IGKVACCTOY(IGBPR,A)*GKDERATE(IGBPR,A);
   VGH\_T.UP(A,IGBPR,S,T) = (IGKVACCTOY(IGBPR,A)*GKDERATE(IGBPR,A))
     /GDATA(IGBPR,'PCM'); by the specification POSITIVE VARIABLE
     VGH\_T,   VGH\_T.LO(A,IGBPR,S,T) is zero.

  \item   IGEXT(G):
  \item  IGHONLY(G):
  \item  IGETOH(G):  (Observe that the heat pump consumes
  electricity, i.e., the associated electricity variable VGE\_T is negative, and generates heat,
  i.e., the associated heat variable VGH\_T is positive.) (Observe
  that the heat pump is assumed located such that there is no
  electricity distribution loss associated with the electricity
  consumed, and that there is a  distribution loss associated
  with the heat generated.)
  \item IGHYRS(G): equations G\_HP   and the bounds  VGEN\_T.FX(IAURBH,IGHYRS,S,T) = 0;
 \item IGWND(G)
  \item IGSOL(G)

\end{itemize}







\subsubsection{WATER\_IN\_R }

PARAMETER WATER\_IN\_R  \index{WATER \_IN\_R}\label{WATER-IN-R}
contains the description of the seasonal variation of the amount
of water inflow to the reservoirs for each installed MW capacity
of hydropower with storage. Unit: MWh/MW.









\subsubsection{HRURALMAXH}
The  internal PARAMETER
HRURALMAXH\index{HRURALMAXH}\label{HRURALMAXH} holds maximal
demand for heat in rural areas as expressed in the units of the
weights and demands used in IDAYSIN\_S, IHOURSIN24 and DH\_VAR\_T.

See also Section \ref{HRURALMAXS}   and \ref{HRURALMAXT}.


\subsubsection{HRURALMAXS}
The  internal PARAMETER
HRURALMAXS\index{HRURALMAXS}\label{HRURALMAXS} holds
ORD\index{ORD} (i.e., the number in the set SSS)  for the season
with maximal demand for heat in rural areas. Unit: (none).


See also Section \ref{HRURALMAXH}   and \ref{HRURALMAXT}.

\subsubsection{HRURALMAXT}
The  internal PARAMETER
HRURALMAXT\index{HRURALMAXT}\label{HRURALMAXT} holds
ORD\index{ORD} (i.e., the number in the set TTT)  for the time
segment with maximal demand for heat in rural areas. Unit: (none).

See also Section \ref{HRURALMAXH}   and \ref{HRURALMAXS}.

 HRURALMAXH  & int. parameter  & (none) & BALMOREL.GMS &  \pageref{HRURALMAXH} \\
 HRURALMAXS  & int. parameter  & (none) & BALMOREL.GMS &  \pageref{HRURALMAXS} \\
 HRURALMAXT  & int. parameter  & (none) & BALMOREL.GMS &  \pageref{HRURALMAXT} \\










\subsection{More elaborate representation  of third regions}

The intention of the incorporation of the representation of third
regions in the model is to cover in an easy way import/export that
is easily    foreseen, e.g.  because contracts exist for a few
years ahead. This poses limitations on what can be modeled.  If
more elaborate representations are desired, the following can be
suggested.

[...]

\subsection{Maximum heat capacity on extraction units} [...]

\subsection{Modeling of individual generation units}[...]



\subsection{Multifuel units }[...]

\subsection{Making  FUELTAX depend on the year }[...]

\subsection{Making GKDERATE depend on the season}[...]

Revision planning ...

\subsection{Non-dispatchable technologies other than wind etc.}

The three generation technology types, hydro without reservoir,
wind and  solar  (GDTYPE= 7, 8, 9, respectively) are considered to
be non-dispatchable\index{dispatch}\index{economic dispatch}, and
hence their generation profile is given by the relevant VAR\_T
parameters. However, there may be a need to model also that other
types of  technologies   have their generation given by a profile,
e.g. incineration plants.

This may be done as follows.

Introduce the element GFXVAR in set GDATA (e.g. between ... and
...) in the file SETS.INC. Enter the necessary information into
TABLE GDATA.  The information to be filled into this table
relative to GFXVAR is a zero, if the generation of the technology
is dispatchable, or if the technology is by nature considered
non-dispatchable (hydro without reservoir, wind and solar, GDTYPE=
7, 8, 9, respectively). The information is a positive integer, if
the technology is otherwise  considered non-dispatchable.
Moreover, this integer will be interpreted to point to a specific
profile, see below.

(not IGEXT, not HPRES)

Introduce the SET GFXSET(G) as
\begin{itemize}
  \item [] GFXSET(G) = YES\$(GDATA(G,'GFXVAR') GT 0) and not wind,sol, hydroROR;
\end{itemize}



Introduce the SET GFX\_TSET /GFX1\_T * GFX5\_T /; ( - where the 5
indicates that  5 different profiles are needed). It should be
placed in the file SETS.INC.

Introduce the PARAMETER GFX (SSS,TTT,GFX\_TSET) and give the
numerical values in a TABLE, e.g. as follows
\begin{itemize}
  \item[]
  TABLE(SSS,TTT,GFX\_TSET)\\
\begin{tabular}{lcccccl}
   & GFX1\_T  & GFX2\_T & GFX3FX\_T & GFX4\_T & GFX5\_T  \\
 S1.T1 &  20  & 40  & 70 & 80 &  30 &\\
S1.T2  &  30  &  55 & 75 & 80  &  50 & \\
 S2.T1 &  20  & 45  & 70 & 80 &  40 & \\
S2.T2  &  30  &  60 & 80 & 80  &  55 & ;
\end{tabular}
\end{itemize}


Der mangler noget om �rsenergier - m�ske kan det blot gives i
forhold til kapaciteten? og husk, at det kan modificeres af
GKDERATE

(Behov for at afklare om IGETOH angvies for el eller varme ...

Then introduce in the file BALMOREL.GMS section ...
\begin{itemize}
  \item[] VGE\_T.FX( GFXSET , S,T)\$ alt andet end HOB (IGETOH??)=  ...
  \item[] VGE\_T.FX( GFXSET , S,T)\$ IGBPR ...
  \item[] VGE\_T.FX( GFXSET , S,T)\$ IGETOH ...
  \item[] VGE\_T.FX( GFXSET , S,T)\$  HOB (IGETOH??)=  ...
 \end{itemize}


(der mangler  svarende til WNDSUM\_Y ...)









\subsubsection{DHFP\_VAR\_T}

PARAMETER DHFP\_VAR\_T \index{DHFP\_VAR\_T}\label{DHFP-VAR-T}
contains the description of seasonal and daily variation of the
heat price in the base year. Unit: (none).

See the comments in relation to PARAMETER DHFP\_VAR\_T, Section
\ref{DEFP-VAR-T}.

\subsubsection{DEFP\_VAR\_T}

PARAMETER DEFP\_VAR\_T \index{DEFP\_VAR\_T}\label{DEFP-VAR-T}
contains the description of seasonal and daily variation of the
electricity price in the base year. Unit: (none) (see description
in relation to DE\_VAR\_T, Section \ref{DE-VAR-T}).



\inputdata Observe that the differentiation of prices by time of the day
and time of the year has not been all that extended historically,
and therefore it is quite possible that in the base year there has
been no variation. If there has been none at all, all entries in
the table should be identical and positive, e.g. 1.









\subsection{New investments: mapping between AAA and G}
\label{IAGKN---}


The specification of where  endogenously determined new
technology capacity of a particular type  can be placed was
described in Section \ref{DOC-SSS-SubsetsofAAA}. The further
implementation consists in the declaration of sets corresponding
to the subsets described in that section: SETS
IAGKNEONLY(AAA,G),\index{IAGKNEONLY}\label{IAGKNEONLY}
IAGKNRURH(AAA,G),\index{IAGKNRURH}\label{IAGKNRURH}
IAGKNURBH(AAA,G),\index{IAGKNURBH}\label{IAGKNURBH} and the
definitions:
\begin{itemize}
\item[] IAGKNEONLY(AGKNEONLY,GAKNEONLY)=YES;
\item[] IAGKNRURH(AGKNRURHY,GAKNRURH)=YES;
\item[] IAGKNURBH(AGKNURBHY,GAKNURBH)=YES;
\end{itemize}


For those area and technology combinations  that do not allow
investment the upper bounds on new capacity and generation are set
to zero. The set IAGKNNOT(AAA,G)\index{IAGKNNOT}\label{IAGKNNOT}
is used for this in the following way:
\begin{itemize}
\item[] IAGKNNOT(IA,G)=NOT(); ?????????
\item[] GKN.FX(IAGKNNOT)=0;
\item[] GEN\_T.FX(IAGKNNOT,S,T)=0;
\item[] GHN\_T.FX(IAGKNNOT,S,T)=0;
\end{itemize}
Then IAGKNNOT is specified as  ...., and the last three
statements  are repeated - and so.
