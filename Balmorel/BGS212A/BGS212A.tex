
\documentclass[twoside,10pt]{article}
% \makeindex

%\newcommand{\gmssec}{the BALMOREL.GMS file section [...]}
%\newcommand{\MONEY}{Money}
%\newcommand{\checks}{The input is checket for reasonable values. See file errors.inc.}
%\newcommand{\inputdata}{Comment on input data: }
%\newcommand{\naming}{Comment on naming conventions: }
\newcommand{\BMS}{'The Balmorel Model Structure' }
\newcommand{\Versionname}{2.12 Alpha   (July 2005) }
\newcommand{\VersionGAMS}{2.25}
% DER HENVISES TIL OG KOPIERES FLITTIGT FRA BMS. DERFOR   DER V�RE KONSISTENS!
% DETTE G�RES I H�NDEN!!!!
% alle s�danne steder er markeret med !!!ZZZ!!!, s�g p� dette

% DER KOPIERES F�LGENDE FRA bms:
% NOGET fra introduction om Balmorel generelt


% DER HENVISES TIL F�LGENDE I bms:
% Afsnittet om filstrukter


% DER SKAL V�RE KONSISTENS med mbs P� F�LGENDE punkter :
% henvisning til versionsnummer




\usepackage{graphicx}


 \setlength{\textheight}{240mm}
 \setlength{\textwidth}{125mm}
 \setlength{\oddsidemargin}{0mm}
 \setlength{\evensidemargin}{30mm}
\setlength{\topmargin}{-15mm}


\begin{document}


%\section*{The Balmorel Model Structure}


%\newpage


\section*{{\large Hans F. Ravn}}

\vspace{5cm}
%\section*{}\section*{}\section*{}

\section*{{\LARGE Balmorel: Getting Started}}

\section*{}\section*{}\section*{}



\vspace{12cm}
 %!!!ZZZ!!!

\subsubsection*{To be used with Balmorel Version \Versionname}
\subsubsection*{This document revised 2005.07.31}

%{\tiny Filename: b\_strdoc.tex  Author: Hans F. Ravn  Date:
%2001.02.26 }
%\setcounter{page}{??}

\newpage

%{\tiny .}
%\newpage




\tableofcontents

\newpage

%{\tiny .}
%\newpage

\section{Introduction}
\label{BGS-S-Introduction}

This paper  describes how to get started using the Balmorel model.
%!!!ZZZ!!! starter her:
 %!!!ZZZ!!!
The  document is part of a series that together documents the
Balmorel model:
\begin{itemize}
  \item[] Balmorel: A  Model for Analyses of the Electricity and
  CHP Markets in the Baltic Sea Region (Main Report)
  \item[]The Balmorel Model: Theoretical Background
  \item[]The Balmorel Model Structure
  \item[]Balmorel: Data and Calibration
  \item[]Balmorel: Getting  Started (this document)
 % \item[]Balmorel: Observations on modelling and results
\end{itemize}

These  documents and further information may be found at the
Balmorel homepage\index{homepage, Balmorel}: www.Balmorel.com.

\subsection{This version}

 %!!!ZZZ!!!
The description given here is for  version \Versionname of the
model.

 %!!!ZZZ!!!
 It is intended for use with the document  \BMS  for the same
version, so please get that document from the Balmorel homepage.




\section{Getting the Model}
 %!!!ZZZ!!!
The model may be downloaded from the homepage www.Balmorel.com.
There may be different versions, make sure that you take the
latest version and in particular that there is consistency between
the model version and the documentation version.

The model comes in a zipped version. Unzip it and place it in a
convenient directory on your computer.

 %!!!ZZZ!!!
Now the file structure described in
 %!!!ZZZ!!!  % \ref{DOC-SS-GAMS-intro}.
 \BMS  should appear,
please consult that document and verify that the necessary files
are present and properly located.

The model is written in the GAMS modeling language, see Section
\ref{BGS-S-GAMS}.



\section{The GAMS Modeling System}
\label{BGS-S-GAMS} GAMS is the acronym for General Algebraic
Modeling System. The system is suitable for formulation,
documentation and solution of large  mathematical models.

The user must have  GAMS installed at the computer, and it is
assumed that the reader is sufficiently familiar with the GAMS
language and the operation of the GAMS system. Conditions for
acquiring GAMS, a User's Guide,
 %\index{User's Guide} \index{GAMS User's Guide}
  a Tutorial,
and other relevant information about the GAMS modeling system may
be found at GAMS homepage  www.gams.com.
 % \index{homepage, GAMS}

Version  \Versionname of the Balmorel model     may be used with
GAMS  version \VersionGAMS { }     or later.

\BMS contains a short description of GAMS. If you are not
familiar with GAMS you may  find a tutorial and  other relevant
introductory  material at the GAMS homepage.
 %!!!ZZZ!!!
% the tutorial by JHU ...
 %!!!ZZZ!!!


The basic GAMS system is file-oriented, and  any word processor
or editor that operates on text files may be used.

User interface facilities using Excel is under development for
the Balmorel model, however it is not necessary to use this.



\section{The Solver}

In order to use the model, a solver is required. The Balmorel
model is a linear model and therefore a linear programming solver
is required. The solver must have an interface with the GAMS
system, see the GAMS User's Guide and the GAMS homepage
www.gams.com.

\section{The Computer}

The GAMS system including solvers with interface to GAMS may be
used on a wide variety of computers. Consult www.gams.com.

The time needed to run the model will depend on the model, the
solver and the computer. The model, in particular its size, may
to some extent be chosen according to circumstances, however, you
will probably soon like to run a large model, and therefore you
will also want a powerful computer.

The standard user interface operates on text files, so an editor
to handle this is necessary.  User interface facilities using
Excel is under development. It is not necessary to use it,  if
you want to, you must have Excel of course.




\section{Running the Model}
\label{BGS-S-Running(text)}

Once the model, the GAMS system and the solver are installed on
your computer  it should be possible to run the model.

There are various ways to operate the GAMS system, cf.
www.gams.com,   use any.

In the following it will be assumed that you can run the installed
model, and that you can do so without getting any error messages.

The description in this section is aimed at working  with the
text files. In Section \ref{BGS-S-Excel} the Excel files will be
described.


The model comes in a small version where e.g.  only one country,
one year and one time segment per year are represented, where
limited information is printed out, and otherwise.  In Section
\ref{BGS-SS-Fullmodel} it will be described how to extend the
model and the output to full capabilities.

\subsection{First Experiments}


\subsubsection{The output}
%\label{BGS-SS-Outputfiles)}

In \BMS  you may find a general description of the output
produced by a successful run of the Balmorel model. In particular
note the distinction between the output automatically generated
by GAMS and the output originating specifically from the Balmorel
model. In the sequel only the latter type will be dealt with.


\subsubsection{Log and Error files}
\label{BGS-SS-LogandErrorfiles}

 Run the model. It is assumed that
this occurs without any problems being reported by the GAMS
system.

However, this does not necessarily mean that the output from the
model was meaningful or as expected. Therefore the Balmorel model
comes with additional features that serves to identify if the
input, the model and the solution were reasonable.

Open the file 'logfile.out' in the directory 'logerror'. Here you
will find an indication if suspicious things were discovered. If
this is the case, the contents of the file  will instruct you on
how find more details. Observe that the mentioned file may not be
updated if the GAMS system reported errors, therefore always check
the date and time in the file.

Remember:
\begin{itemize}
\item After any model run which was somehow declared successful
by GAMS, check the file 'logfile.out'.
\end{itemize}



\subsubsection{Other output files}
\label{BGS-SS-Outputfiles}

Output describing the energy system results of a simulation will
appear in various files in the directory 'printout'.


Open one file with the extension 'out' in the   directory
'printout' and study the contents. You may also study the contents
of other files  with the extension 'out' in the same directory.

See the description of the various output files in a document
located in the 'documentation' directory.



Observe that the output files are overwritten each time the model
is run. Therefore you will have to move or rename the files after
each run if you want to save them for later use.


\subsubsection{The input}
\label{BGS-SS-TheInput)}

 In \BMS  you may find a  description of the
file structure of the model, the input to the model, and where
this is located in the different files.

In the following you will be directed to make elementary changes
in the input that is found in some of the files that are included
into the main file 'balmorel.gms'.



\subsubsection{Change demand}
\label{BGS-SS-ChangeDemand}

Open the file 'de.inc' in the directory 'model'. For the country
and the year simulated (search one or more of the output files,
cf. Section \ref{BGS-SS-Outputfiles}, to identify this) change the
electricity demand in  one of the regions (see \BMS for the
definition of region).  Rerun the model and study the consequences
for the output in various files (Section
\ref{BGS-SS-Outputfiles}).

Remember  that the output files are overwritten each time the
model is run, Section \ref{BGS-SS-Outputfiles}.



You may similarly change heat demand in the file 'dh.inc' for one
heat area in  the country and year simulated  and study the
consequences (see \BMS for the definition of area). Do not forget
the advice given in Section \ref{BGS-SS-LogandErrorfiles}!

You may experiment with other  changes of numerical input data
found in e.g. 'geogr.inc' or 'fuelp.inc'.

Make sure that after these experiments you reenter the original
numerical values.

Do not change anything in the file 'balmorel.gms', and be very
careful if you make changes  in 'sets.inc'!


\subsection{Carrying On}


\subsubsection{The inputout file}
\label{BGS-SS-Inputout}

 The file 'inputout.out' in the directory
'printout' contains a summary of the input to the model that is
simulated. The file is useful for getting an overview of the model
and data, for error finding and otherwise.

The simulated model need not represent all the countries and all
the years that are indicated in the include files (eg. 'de.inc',
cf. Section \ref{BGS-SS-ChangeDemand}).
 %!!!ZZZ!!!
See also the description on the difference between data structure
and simulation in \BMS .
 %!!!ZZZ!!!


The following section illustrates how the simulated model is
specified, see also Section \ref{BGS-SS-Fullmodel}.





\subsubsection{Countries}
\label{BGS-S-C(CCC)}


The countries in the data structure are specified in  SET CCC
which is defined in the file 'sets.inc'. Open this file, find CCC
and study the definition.

The  countries in the simulation are specified in  SET C(CCC)
(which is a subset of SET CCC), also in the file 'sets.inc'. Find
C(CCC) and study the definition. Observe in particular that an
asterix (*) in the first position of a line turns that line into a
comment. Only one country from the set CCC is selected for
inclusion in C(CCC), since all other countries have an asterix in
the first position of the line.

Include one more country. Run the model. Study the consequences
(Section \ref{BGS-SS-Outputfiles}, Section \ref{BGS-SS-Inputout}).


You end this section by redefining  SET C(CCC) to contain only
one country (the  one originally included or any other).


\subsubsection{The years}
\label{BGS-S-Y(YYY)}

The years in the data structure are given as SET YYY and the years
simulated are given in SET Y(YYY), both located in the file
'sets.inc'.

Find YYY and study the definition. Observe that a notation like
"/1995*2030/" means that all years from 1995 to 2030 are
included, and "/2000*2000/" hence means that only 2000 is
selected. Find Y(YYY) and change it to contain two consecutive
years, rerun the model and study the consequences.

You end this section by redefining  SET Y(YYY) to contain only
one  year.



\subsubsection{Time segments within the year}
\label{BGS-S-ST(SSSTTT)}

The simulated  year  may be subdivided into time segments (S,T),
cf. \BMS .

The seasons (e.g. months of the year, or summer/winter) in the
data structure are given as SET SSS and the seasons  simulated are
given in SET S(SSS), both located in the file 'sets.inc'.

The subdivision of the seasons (e.g., 4-hours intervals)  are
given in the data structure by  SET TTT and the ones simulated are
given in SET T(TTT), both located in the file 'sets.inc'.

Change S(SSS) and T(TTT) to contain two labels each, rerun the
model and study the consequences.


You may experiment with various selections of time segments by
changing the definitions of S(SSS) and T(TTT). Observe how the
solution and the solution time depend on this. You will probably
find that the number of time segments per year is a crucial
determinant for the results as well as for the solution time.
Intuitively, one would prefer many time segments to obtain more
accuracy, however, this implies a more heavy computational burden.

This is one of the reasons why the model has been constructed
with a flexible subdivision of the year. For any problem to be
analysed the determination of the necessary and sufficient number
of time segments of the year is part of the problem.

You end this section by defining the sets S(SSS) and T(TTT) to
contain only one or two elements  each.


\subsubsection{The full model}
\label{BGS-SS-Fullmodel}

The model installed  was a small version where e.g. only one
country, one year and one time segment per year are represented.
In the previous sections some of this has been illustrated, and
it has been explained how to extend the model.

A list of the main intended limitations in the supplied model is
given below along with an indication of how they are implemented,
and hence what should be changed in order to  eliminate them.
 \begin{itemize}
 \item Only one country: SET C(CCC), cf. Section  \ref{BGS-S-C(CCC)}.
 \item One time segment per year: SET T(TTT) and SET S(SSS), cf.
    Section  \ref{BGS-S-ST(SSSTTT)}.
 \item Only one year: SET Y(YYY), cf. Section  \ref{BGS-S-Y(YYY)}.
 \item Limited output: files 'print1.inc',   'print2.inc',   'prt3-bb1.inc',   'prt4-bb1.inc',
   cf. Section \ref{BGS-S-Output} and Section \ref{BGS-S-Display}.
 \item No investments in new production capacity: consult \BMS .
 \item No investments in  new transmission capacity: consult \BMS .
%  \item ...: consult \BMS .
%  \item ...: consult \BMS .
%  \item ...: consult \BMS .
 \end{itemize}

 %!!!ZZZ!!!
%At this stage you should consult \BMS  which also provides a
%description of how the model may be further modified.





\subsubsection{Print files}
\label{BGS-S-Output}

The output observed above was produced by instructions written
specifically for the Balmorel model. Typically, one file of
output was produced by one file containing the printing
specifications. These files are located in the subdirectory
 %!!!ZZZ!!!
 'printinc' and have the extension '.inc'. The file 'prt4-bb1.inc'
specifies which files that are selected.

In 'prt4-bb1.inc' you may comment in some more print files
(provided of course that they exist and are properly located,
defined and specified, note in particular 'print1.inc'). Comment
in two or three more files, run the model and observe the new
files in the subdirectory
 %!!!ZZZ!!!
'printout'. Open them and study the contents.


Observe that you computer may have a limit on the number of files
that may be open at the same time. Therefore be careful not to
use too may print files in the same simulation. If you encountered
no problems with this, so far, you may proceed, otherwise make a
smaller selection of print files that you want to have open by
commenting out those that are not necessary. Alternatively you
may change the setting of the computer.


See Section \ref{BGS-S-Display} for more on output.


\subsubsection{Designing and controlling the output}
\label{BGS-S-Display}

The output facilities described in Section \ref{BGS-S-Output}
were prepared specifically for the Balmorel model. Although there
are many print files, you will probably soon like to have more
simulation output. You may get this by copying and modifying the
print files to suit your purpose.

The mentioned print files rely on the PUT command.  You may also
use the DISPLAY command which you may find more easy to use. This
produces output to the file 'balmorel.lst'.

Place the statement
 %!!!ZZZ!!!
 "DISPLAY 'MyOwnChoice: ', Y, C, G, DE, VGE\_T.L , 'End MyOwnChoice' ;" at the very end of the file 'balmorel.gms'. Run the model
and find  the result in 'balmorel.lst' (near the end of the file;
you may search for the text string 'MyOwnChoice').

Various options control the type, amount and format of the output
in the file 'balmorel.lst'. The options along with a short
description is given in the file
 %!!!ZZZ!!!
 'balgams.gms'.  See further the GAMS User's Manual (Section
\ref{BGS-S-GAMS}).


\section{Using Excel Interface}
\label{BGS-S-Excel}

The description in Section \ref{BGS-S-Running(text)} was based on
working with the text file version of the model.

However, it is also possible to work with an Excel interface
 %!!!ZZZ!!!
(presently under development).
 %!!!ZZZ!!!
 Basically, this consists of two parts, an input part containing
the model and additional material, and an output part providing
graphs based on some of the output files.

The advantage of using a spreadsheet environment like Excel as
interface is that it provides convenient facilities for
preparation of input. In  few cases the input to a model can be
taken directly from data sources, in  most cases it must in some
way be manipulated. Some kinds of such manipulation may
conveniently be done in a spreadsheet environment. Further,
through  graphical presentation and otherwise, easy overview and
control of the input is possible. Similarly the output may
advantageously be presented and further processed in a spreadsheet
environment. Some Excel spreadsheet facilities are provided for
this purpose.

The text files described in Section \ref{BGS-S-Running(text)}
serve as intermediate files. Thus, it will always be possible to
observe directly in those files the input to and output from
simulation.

Since the Excel user interface is under construction it will not
be explained here.

You may get a preliminary version by contacting us as
www.Balmorel.com. 


\subsection{The input system}

Awaiting description.
 %(You may try to find an Excel spreadsheet in the directory 'data-pre-inc'.)

\subsection{The output system}
Awaiting description.
%  (You may try to find an Excel spreadsheet in the directory 'printout'.)



\section{And then}

The present document provides a basic introduction on how to get
started. To proceed, you will have to experiment yourself. You may
benefit from studying other documents mentioned in Section
\ref{BGS-S-Introduction},
 %!!!ZZZ!!!
 in particular \BMS.

In case of serious problems you may contact us, see
www.Balmorel.com for contact address.

Any comments to this document or any other aspects related to
Balmorel are welcome.


\end{document}
